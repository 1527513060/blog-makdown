\documentclass{book}

\usepackage[CJKspace]{xeCJK} 
\usepackage{indentfirst}
\usepackage[bindingoffset=0.5cm, margin=1.5cm, includeheadfoot]{geometry}
\setlength{\parindent}{2em}
\renewcommand{\contentsname}{目录}
\usepackage{tocbibind}
\usepackage{booktabs}
\usepackage[titletoc]{appendix} % this is used for \appendices
\usepackage[
	%urlbordercolor = {1 1 1},
	%linkbordercolor = {1 1 1},
	%citebordercolor = {1 1 1},
	bookmarksnumbered, % add bookmark number in pdf output
	urlcolor = blue,
	colorlinks = true,
	citecolor = black,
	linkcolor = black]{hyperref}

\setCJKmainfont    [
      %BoldFont   = FandolSong-Bold,
      BoldFont = 文泉驿微米黑, % uncomment for release build
      ItalicFont = FandolKai,
      %ItalicFont = AR PL KaitiM GB,
    ]{FandolSong}

\linespread{1.3}    
\begin{document}
\date{1976年6月第1版\\
1976年6月北京第1次印刷}
\title{\textbf{
政治经济学讲话(社会主义部分)
}}
\author{《政治经济学讲话(社会主义部分)》编写组}
\maketitle

\tableofcontents

\chapter{社会主义社会是从资本主义社会到共产主义社会的过渡时期}

社会主义制度代替资本主义制度,必须经过无产阶级革命和无产阶级专政才能实现。无产阶级专政的建立,开始了从资本主义社会到共产主义社会的过渡时期。在生产资料所有制的社会主义改造基本完成以后,社会主义社会仍然存在着生产关系和生产力、上层建筑和经济基础之间的矛盾,仍然存在着阶级、阶级矛盾和阶级斗争。正确地认识社会主义社会的性质和特征,对于我们坚持以阶级斗争为纲,坚持无产阶级专政,把无产阶级专政下的继续革命进行到底,加快社会主义建设,有着巨大的现实意义。

\section{无产阶级革命和无产阶级专政是社会主义经济制度产生的前提}

\subsection{社会主义制度代替资本主义制度,必须经过无产阶级革命}

人类社会是由低级阶段向高级阶段发展的。是什么力量推动人类社会不断地向前发展的呢?马克思主义认为,推动人类社会不断地向前发展的,既不是神仙皇帝,也不是英雄豪杰,而是社会的基本矛盾,即生产关系和生产力之间、上层建筑和经济基础之间的矛盾。这一基本矛盾,在存在阶级的社会里,表现为阶级之间的矛盾和斗争。因而,阶级斗争是推动阶级社会发展的根本动力。

在资本主义制度下,资本家占有生产资料,而劳动者则一无所有,不得不出卖自己的劳动力,遭受资本家的残酷剥削。资本主义的发展,一方面使生产日益社会化,另一方面又使生产资料越来越集中在人数很少的大资本家手中。这样就使资本主义社会的基本矛盾即生产的社会性和资本主义私人占有制之间的矛盾日益尖锐化。特别是到了帝国主义阶段,这种矛盾达到空前尖锐的程度。日益频繁地爆发经济危机,资本主义生产关系成为生产力发展的最大障碍。资本主义生产关系和生产力之间的对抗性矛盾,决不能在资本主义制度内部得到解决,只有用社会主义生产关系来代替已经过时的资本主义生产关系,才能推动生产力进一步发展。马克思指出:“生产资料的集中和劳动的社会化,达到了同它们的资本主义外壳不能相容的地步。这个外壳就要炸毁了。资本主义私有制的丧钟就要响了。剥夺者就要被剥夺了。”

资本主义的发展,不仅使生产日益社会化,为社会主义生产关系代替资本主义生产关系准备了必要的物质条件,而且也造成了实现这个革命转变的社会力量即无产阶级,准备了自己的掘墓人。毛主席指出:“社会主义制度终究要代替资本主义制度,这是一个不以人们自己的意志为转移的客观规律。不管反动派怎样企图阻止历史车轮的前进,革命或迟或早总会发生,并且将必然取得胜利。”

社会主义制度代替资本主义制度,必须经过无产阶级革命和无产阶级专政才能实现。我们知道,在人类历史上,以往各种社会经济制度的更替,除了奴隶制度代替原始公社制度以外,都是用一种新的剥削制度代替另一种已经陈旧的剥削制度。由于一切剥削制度的基础都是生产资料的私有制,因而新的生产关系有可能在旧社会内部逐渐产生出来。但是,即使是这样,不论是封建制度代替奴隶制度,还是资本主义制度代替封建制度,也还必须由代表新的生产关系的剥削阶级发动革命,夺取政权,运用政权的力量为新的生产关系的进一步发展扫清道路,使它在整个社会中占统治地位。至于社会主义制度代替资本主义制度,则更是只能如此。因为,社会主义生产关系是以生产资料的社会主义公有制为基础的,同以生产资料的资本主义私有制为基础的资本主义生产关系是根本对立的,它根本不可能在资本主义社会内部产生。资产阶级的国家机器及其全部上层建筑,都在维护资本主义私有制,维护资本主义剥削制度。因此,无产阶级要以社会主义公有制代替资本主义私有制,用社会主义制度代替资本主义制度,必须进行革命斗争,夺取政权,建立无产阶级专政。

无产阶级取得政权,建立无产阶级专政的唯一道路,是进行暴力革命。这是因为,资产阶级决不会自动放弃自己的政治统治和经济剥削,他们总是首先使用反革命的暴力,使用国家这个阶级压迫的工具,来残酷地镇压无产阶级革命运动。因此无产阶级只有用革命的暴力粉碎反革命的暴力,才能推翻资产阶级的反动统治,彻底打碎资产阶级的国家机器,建立起无产阶级专政,为消灭资本主义制度和建立社会主义制度开辟道路。

“既要革命,就要有一个革命党。”无产阶级要取得革命的胜利,必须有自己的按马克思主义理论武装起来的政党,执行一条马克思主义的革命路线,采取正确的战略和策略。

坚持无产阶级暴力革命,打碎资产阶级的国家机器,还是反对无产阶级暴力革命,维护资产阶级的国家机器,是马克思主义同修正主义反复斗争的焦点。

马克思和恩格斯在《共产党宣言》中指出:“共产党人不屑于隐瞒自己的观点和意图。他们公开宣布:他们的目的只有用暴力推翻全部现行的社会制度才能达到。让统治阶级在共产主义革命面前发抖吧。”一八七一年,巴黎公社是无产阶级拿起武器,用革命暴力推翻资本主义制度的具有世界历史意义的第一次演习。巴黎公社虽然失败了,但是它的原则是永存的。列宁指出;“不用暴力破坏资产阶级的国家机器,不用新的国家机器代替它,无产阶级革命是不可能的。”一九一七年俄历十月二十五日,俄国无产阶级在列宁和布尔什维克党的领导下举行武装起义,夺取了政权,第一次胜利地实现了无产阶级的社会主义革命,开辟了人类历史的新纪元。毛主席继承、捍卫和发展了马克思列宁主义的理论,把马克思列宁主义的普遍真理同中国的具体实践相结合,深刻地分析了旧中国的社会性质,制定了领导新民主主义革命和社会主义革命的完整理论,提出了“枪杆子里面出政权”的著名论断。中国人民在毛主席和中国共产党的领导下,沿着十月革命的道路,经过长期的、艰苦的武装斗争,终于推翻了帝国主义、封建主义和官僚资本主义的反动统治,取得了新民主主义革命的伟大胜利,开始了社会主义革命和无产阶级专政的新的历史阶段。俄国十月革命和中国革命的胜利,是马克思列宁主义关于无产阶级革命和无产阶级专政的理论的伟大胜利。

修正主义者为了维护资本主义的统治,疯狂地反对马克思列宁主义关于无产阶级革命和无产阶级专政的理论,大肆鼓吹“和平过渡”的谬论。第二国际修正主义者伯恩施坦、考茨基之流叫嚷什么,对资本主义制度“用不着炸毁它们,只需要继续发展它们”,就可以“和平地长入”社会主义,主张“从取得议会中多数的办法来夺取政权”。苏修叛徒集团竭力宣扬“通过议会的道路向社会主义过渡”。刘少奇一类在我国民主主义革命时期,也抛出了“和平民主新阶段”的投降主义路线,说什么“中国革命主要斗争形式要由武装斗争转变到非武装的、群众的、议会斗争形式”。我们知道,资产阶级国家机器的主要部分是军队,而不是议会。议会只是资产阶级专政的装饰品。资产阶级实行议会制还是取消议会制,给予议会较大的权力还是较小的权力,采取这种选举法还是那种选举法,总是按照资产阶级统治的需要来决定的。而在无产阶级革命斗争的面前,资产阶级总是“首先使用暴力,发动内战,‘把刺刀提到议事日程上来’”的。因此,通过“议会道路”来实现社会主义,完全是不可能的,是骗人的鬼活。正如列宁指出:“为了通过选举和各种党派在议会中的斗争达到教育群众的目的,参加资产阶级的议会活动,对革命无产阶级的政党来说是必要的。但是,把阶级斗争局限于议会内的斗争,或者认为议会内的斗争是最高的、决定性的、支配着其余一切斗争形式的斗争,那就是实际上转到资产阶级方面去而反对无产阶级。”

目前,继续受资产阶级压迫和剥削的无产阶级,要用社会主义制度代替资本主义制度,必须清除“和平过渡”修正主义路线的影响,在自己的马克思列宁主义政党的领导下,开展武装斗争,通过暴力革命推翻资产阶级的统治,建立无产阶级专政。

\subsection{无产阶级专政是无产阶级达到自己的阶级目的的工具。}

无产阶级通过暴力革命建立的无产阶级专政,同其他一切国家政权一样,也是一个阶级压迫另一个阶级的工具。但是,它不是由少数剥削者阶级压迫多数被剥削者阶级,而是多数被剥削者阶级压迫少数剥削者阶级。因此,它和剥削阶级的国家政权有本质的区别,是人类历史上最进步的,也是最后的一次阶级专政。

无产阶级夺取政权,使自己上升为统治阶级,这只是无产阶级社会主义革命的开始。因为,无产阶级革命的最终目的,是要消灭剥削,消灭阶级,实现共产主义。而实现无产阶级革命的最终日的,是一场广泛而深刻的革命,存在着无产阶级和资产阶级之间、社会主义和资本主义之间的尖锐斗争。因而,无产阶级专政不是阶级斗争的结束,而是阶级斗争在新形式中的继续。这个专政是“对旧社会的势力和传统进行的顽强斗争,流血的和不流血的,暴力的和和平的,军事的和经济的,教育的和行政的斗争”。

无产阶级专政是社会主义的上层建筑,是无产阶级达到自己的阶级目的的工具。它有三个方面的职能:首先,无产阶级利用自己的政治统治,对阶级敌人实行专政,特别是对党内资产阶级,走资本主义道路的当权派实行专政,并防御国家外部敌人的颠覆活动和可能的侵略,以保证社会主义建设事业的顺利进行,巩固和发展社会主义经济制度。

无产阶级专政在人民内部实行民主,采用民主的方法,用马克思主义、列宁主义、毛泽东思想教育和改造劳动人民,引导劳动人民清除资产阶级和内外反动派的影响,同旧的传统观念实行彻底决裂,坚决走社会主义道路,从而促进社会主义经济基础的巩固和发展。

无产阶级运用无产阶级专政这个国家政权,有计划、有步骤地进行社会主义革命和社会主义建设,逐步地消灭旧社会的痕迹,建立、巩固和发展社会主义经济基础,保证社会主义经济战胜资本主义经济,为实现共产主义创造条件。

无产阶级专政,是工人阶级领导的以工农联盟为基础的国家政权。工人阶级是通过自己的政党实现对国家的领导的。所以,坚持无产阶级专政,就必须维护共产党的领导,巩固无产阶级领导下的工农联盟。

“只有承认阶级斗争,同时也承认无产阶级专政的人,才是马克思主义者。”一切新老修正主义者,都竭力地歪曲无产阶级专政的阶级实质,反对无产阶级专政。苏修叛徒集团曾经公开宣扬用所谓“全民国家”,来代替无产阶级专政;林彪一类也竭力鼓吹孔老二的什么“仁政”,咒骂“恃力者亡”,其实质就是要取消无产阶级专政的压迫和镇压阶级敌人的职能,掩护地主资产阶级猖狂进攻,破坏社会主义革命和社会主义建设,颠覆无产阶级专政,复辟资本主义。现代修正主义者勃列日涅夫之流口头上也在那里讲什么无产阶级专政,但他们只是以承认这个名称为幌子,实际上却改变了专政的阶级内容,把它变成官僚垄断资产阶级的法酉斯专政,并以此为掩护,镇压苏联人民的反抗。我们必须彻底批判什么“全民国家”、“仁政”之类的谬论,坚持和加强无产阶级专政。无产阶级专政“对于胜利了的人民,这是如同布帛菽藏粟一样地不可以须臾离开的东西。这是一个很好的东西,是一个护身的法宝,是一个传家的法宝,直到国外的帝国主义和国内的阶级被彻底地干净地消灭之日,这个法宝是万万不可以弃置不用的”。

\section{社会主义经济制度的建立}

\subsection{收大资本,掌握国家的经济命脉}

无产阶级专政建立后,开始了从资本主义社会到共产主义社会的过渡时期。无产阶级为建立自己的经济基础必须消灭生产资料的私有制,代之以生产资料的社会主义公有制。

无产阶级在夺得政权以后,必须依靠无产阶级专政国家政权的力量,立即将资产阶级借以垄断国民经济命脉的大工厂、大银行和交通运输等大企业,收归社会主义国家所有,从而掌握国家的经济命脉。否则,不甘心失败的资产阶级就会凭借他们仍然占有的经济实力,配合其政治上、军事上的反扑,来窒息和破坏无产阶级专政,复辟资本主义。巴黎公社失败的历史教训之一,就在于无产阶级在革命的中途停了下来,没有把象银行这样的关系国民经济命脉的大企业掌握在自己的手中。

旧中国是一个半殖民地、半封建的社会。帝国主义在中国开办企业,不但在重工业方面处于垄断地位,而且在轻工业方面也占有一定的比重。官僚资本在中国资本主义工业的固定资本中占百分之八十,而且还垄断了整个金融和贸易。帝国主义、封建主义、官僚资本主义的生产关系,是旧中国最落后、最反动的生产关系,是生产力发展的严重障碍。在我国人民解放战争的胜利发展过程中,党和政府根据毛主席制定的无产阶级革命路线,分别地接管了帝国主义国家在中国的企业,没收了官僚资本主义企业,归社会主义国家所有,并进行了一系列的改革,使之变为社会主义国营企业,从而使我国无产阶级掌握了国家的经济命脉。对于封建土地所有制,则是根据我国的具体情况,采取深入发动群众,进行土地改革,没收封建地主阶级的土地分给无地和少地的农民所有。这样,就逐步地消灭了帝国主义所有制、官僚资本主义所有制和封建主义所有制。

由于无产阶级取得政权以后,立即没收大资本(官僚资本),产生了社会主义国营经济,社会经济成分发生了根本的变化。但是,没收大资本,并不意味着一切私有经济都已被消灭,还存在大量的中小资本和个体所有制。因此,在过渡时期的初期,国民经济中一般地存在着五种经济成分:社会主义国营经济、合作社经济、农业和手工业个体经济、国家资本主义经济、资本主义经济。其中社会主义国营经济、农业和手工业个体经济、资本主义经济是基本的经济成分。与此相应地存在工人阶级、小资产阶级(主要是农民)、资产阶级三个基本阶级。一九五二年,在我国国民收入中,社会主义国营经济占百分之十九点一,合作社经济占百分之一点五,个体经济占百分之七十一点八,国家资本主义经济占百分之零点七,资本主义经济占百分之六点九。

社会主义经济同资本主义经济、农业和手工业个体经济,是不可能互不干扰地平衡发展的。无产阶级要求变资本主义经济和个体经济为社会主义经济,以生产资料的公有制代替私有制,使生产资料公有制成为社会唯一的经济基础。而资产阶级则力图发展资本主义,抗拒社会主义改造。斗争的焦点,是走社会主义道路还是走资本主义道路,是无产阶级专政还是资产阶级专政。

在我国革命取得全国胜利的前夕,毛主席在党的七届二中全会的报告中,全面地分析了新民主主义革命胜利以后,国内外阶级斗争的新形势和社会各阶级相互关系的变动情况,指出了国内的主要矛盾是“工人阶级和资产阶级的矛盾”,并制定了进行社会主义革命、建立和巩固无产阶级专政的路线。在一九五三年,当恢复国民经济的任务基本完成,在开展“三反”“五反”运动,打退资产阶级的猖狂进攻以后,毛主席根据列宁关于过渡时期的学说和我国的实践,制定了党在过渡时期的总路线:“在一个相当长的时期内,逐步实现国家的社会主义工业化,并逐步实现国家对农业、对手工业和对资本主义工商业的社会主义改造”。这条总路线的实质,是解决生产资料所有制的问题,变多种经济成分为单一的社会主义经济,使社会主义公有制成为我们国家唯一的经济基础。

刘少奇一类疯狂地反对毛主席制定的党在过渡时期的总路线,提出了“巩固新民主主义秩序”的反动纲领,竭力鼓吹“剥削有功”,“雇工单干,应该放任自流”的谬论,反对社会主义革命,妄图保留生产资料私有制,发展资本主义。毛主席对此进行了深刻的批判,指出:“过渡时期充满着矛盾和斗争。我们现在的革命斗争,甚至比过去的武装革命斗争还要深刻。这是要把资本主义制度和一切剥削制度彻底埋葬的一场革命。‘确立新民主主义社会秩序’的想法,是不符合实际斗争情况的,是妨碍社会主义事业的发展的。”

\subsection{改造资本主义工商业,使之逐步转变为社会主义全民所有制}

消灭资本主义经济,把生产资料的资本主义所有制转变为社会主义全民所有制,是社会主义革命的一项基本任务。因为,首先,资本主义经济是以追求剩余价值为目的,生产和经营是无政府状态的。它的存在和发展,不仅会妨碍社会主义经济的发展,而且还必然力图夺取国民经济的领导权,把社会拉向资本主义道路。其次,资本主义经济本身存在着生产的社会性和资本主义私人占有制之间的矛盾,严重地阻碍着社会生产力的发展。第三,无产阶级成为统治阶级以后,决不能长期容忍资本主义的剥削。而资产阶级也不甘心于失败,必然要从政治、思想意识到经济各个方面,向无产阶级发动猖狂进攻,妄图搞垮社会主义经济,颠覆无产阶级专政。

无产阶级用什么办法来消灭这部分中小资本呢?马克思主义认为,消灭资本主义经济有两种不同的方法,一种是没收,另一种是赎买。列宁曾经指出:“在一定条件下,工人决不拒绝向资产阶级赎买。”我们知道,在无产阶级掌握国家政权和国民经济命脉的条件下,迫使一部分资本家接受赎买政策,给予一定的赎金,将他们经营的资本主义企业逐步地改造为社会主义企业,避免或减少突然变革中可能引起的损失和混乱,对无产阶级是有利的。但是,能否用赎买的办法来消灭中小资本,还要取决于各国当时的国内和国际条件,取决于资产阶级的态度。俄国十月革命胜利以后,列宁曾试图通过国家资本主义的方式,对一部分资本家进行赎买。而由于资产阶级的疯狂反抗,这个设想没有得到实现。

毛主席在领导中国革命的过程中,对我国民族资产阶级进行了深刻的分析,规定了具体的政策,发展了马克思列宁主义关于在一定条件下可以对资产阶级进行赎买的思想。

我国占有中小资本的民族资产阶级具有两面性,在民主革命时期,它有革命性的一面,又有妥协性的一面;在社会主义革命时期,它有被迫接受社会主义改造的可能性,又有强烈发展资本主义的反动性,他们所经营的民族资本主义工商业,在我国国民经济恢复时期也具有两方面的作用,既有有利于国计民生的积极作用,又有不利于国计民生的消极作用。毛主席根据民族资产阶级的两面性和民族资本主义经济的两重作用,制定了对民族资本主义工商业实行利用、限制和改造的政策,即通过行政机关的管理、国营经济的领导和工人群众的监督,利用资本主义工商业有利于国计民生的积极作用,限制其不利于国计民生的消极作用,逐步地将民族资本主义所有制改造为社会主义全民所有制。

同时,在我国的具体条件下,民族资产阶级除了接受社会主义改造,没有别的更好的出路。首先,无产阶级掌握了政权,建立了巩固的无产阶级专政。“人民手里有强大的国家机器,不怕民族资产阶级造反。”其次,无产阶级没收了官僚资本,建立了强大的社会主义国营经济,掌握了国家的经济命脉。第三,工人阶级和农民在政治上经济上结成了巩固的联盟,无产阶级占领了广大农村阵地,资产阶级极为孤立。第四,国际形势有利于无产阶级而不利于资产阶级。

我国对资本主义工商业的社会主义改造,是通过各种形式的国家资本主义进行的:在无产阶级专政条件下,国家资本主义是无产阶级专政的国家“能够加以限制,能够规定其活动范围的资本主义”。资本主义经济纳入国家资本主义的形式,就和社会主义经济发生联系,而程度不同地改变自己的性质。在我国,国家资本主义经历了一个由工业方面的加工、订货、统购、包销和商业方面的经销、代销的初级形式到个别企业公私合营和全行业公私合营的高级形式的发展过程。这样,社会主义经济和资本主义经济的联系,也由流通领域进入生产领域,逐步地改变企业的性质。当发展到全行业公私合营时,资本家对生产资料的所有权,只是表现在按私股份额取得一定的股息即定息上,而同生产资料的支配权完全分离。这样的公私合营企业,已经基本上是社会主义企业,当国家规定的支付定息的年限期满而停止支付定息时,就完全成为社会主义全民所有制的企业了。

我国对资本主义工商业社会主义改造所采取的赎买的形式,在全行业公私合营以前是分配利润的制度,以后则是定息制度。资本家拿取的定息,与利润一样,也是对工人劳动的无偿占有,实质上还是剥削。

无产阶级用赎买的办法,来改造私人资本主义经济,并不是什么“和平过渡”,而是阶级斗争的一种形式。事实上,正是经过反复的阶级较量,沉重地打击了资产阶级反对限制、抗拒改造的行为,才迫使他们不得不接受社会主义改造。我国一九五○年反对投机倒把活动的斗争,一九五二年反对资产阶级行贿、偷税漏税、盗窃国家资财、偷工减料、盗窃国家经济情报的斗争,一九五六年公私合营高潮时期反对资本家抽逃资金、抬高私股估价的斗争,以及一九五七年反对资产阶级右派猖狂进攻的斗争,就是无产阶级和资产阶级之间的限制与反限制、改造与反改造的阶级斗争。资产阶级在党内的代理人刘少奇一类疯狂反对毛主席的对民族资本利用、限制和改造的政策,鼓吹“剥削有功”论,美化资本主义制度,说什么“中国不是资本主义太多了,而是资本主义太少了”,“要无限制的大发展”,妄图阻碍社会主义革命,使中国走发展资本主义的道路。我国正是在毛主席的正确领导下,粉碎了刘少奇一类的破坏,打退了资产阶级的猖狂进攻,经过反复的斗争,在农业合作化高潮的推动下,于一九五六年基本上完成了对资本主义工商业的社会主义改造。

\subsection{改造个体经济,建立社会主义劳动群众集体所有制}

无产阶级为了彻底战胜资本主义,建设社会主义,必须对以生产资料私有制为基础的个体农业进行社会主义改造,把小农经济改造成为社会主义经济。“没有农业社会化,就没有全部的巩固的社会主义。”

我国在新民主主义革命胜利后,开展了大规模的土地改革运动,消灭了封建的土地所有制,实现了农民的土地所有制。土地改革的结果,在很大程度上解放了农业生产力,促进了农业生产的发展。但是,小农经济是以一家一户为单位的分散经营的小私有经济,它无力采用新式农具和先进技术,无力抗御自然灾害,不能合理使用土地和劳力。因此,小农经济仍然限制着农业生产力的进一步发展,不仅使农民不能摆脱落后和贫困,而且与国家有计划的经济建设、社会主义工业化的需要也相矛盾。同时,“小生产是经常地、每日每时地、自发地和大批地产生着资本主义和资产阶级的”。土地改革以后,“农村中的资本主义自发势力一天一天地在发展,新富农已经到处出现,许多富裕中农力求把自己变为富农。许多贫农,则因为生产资料不足,仍然处于贫困地位,有些人欠了债,有些人出卖土地,或者出租土地。这种情况如果让它发展下去,农村中向两极分化的现象必然一天一天地严重起来”。这样,在土地改革基础上建立起来的工农联盟,就有被破坏的危险。因此,土地改革以后,必须“趋热打铁”,不停顿地、及时地对农业进行社会主义改造。

通过什么道路来改造小农经济呢?恩格斯指出:“当我们掌握了国家权力的时候,我们绝不会用暴力去剥夺小农(不论有无报偿,都是一样),象我们将不得不如此对待大土地占有者那样。我们对于小农的任务,首先是把他们的私人生产和私人占有变为合作社的生产和占有,但不是采用暴力,而是通过示范和为此提供社会帮助。”由于个体农民和手工业者,既是小私有者又是劳动者,是依靠自己的劳动,而不是依靠剥削他人为生的,是无产阶级的同盟军,因而在任何情况下,对他们私有的生产资料是不能剥夺的,只能引导他们自愿联合起来,走合作化的道路。对农民采取任何强制的措施,都会破坏工农联盟,而给社会主义革命和社会主义建设带来不可弥补的损失。

通过合作化的道路对农业进行社会主义改造,不仅是必要的,而且也是可能的。首先,无产阶级政党的正确的路线和政策,是实现农业合作化的根本保证。其次,广大的劳动农民特别是贫下中农,“有一种走社会主义道路的积极性”,他们完全有可能在无产阶级的领导下,走上社会主义道路。

农业合作化运动是一场用社会主义战胜资本主义的伟大的革命斗争,必须有一条正确的马克思主义的阶级路线。列宁根据当时俄国农民中的各个阶层对农业合作化的不同态度,制定了依靠贫农、团结中农、反对富农的阶级路线。我国土地改革以后,农村的阶级关系发生了新的变化,许多贫农上升为中农,中农的队伍扩大了。毛主席针对这一新情况,对占农村人口大多数的中农作了深刻的阶级分析,指出必须把中农区分为下中农和上中农。下中农,特别是新下中农,在经济上还不富裕,生活水平还比较低,一般地具有走社会主义道路的积极性,同贫农一起是工人阶级在农村中的依靠力量。而上中农,即富裕中农,在经济上比较富裕,一般地具有严重的资本主义自发倾向,对农业合作化运动是动摇和抵触的,必须对他们加强社会主义教育,同他们的资本主义倾向作斗争才能巩固地团结他们,争取他们走社会主义道路。毛主席在科学地分析我国土地改革后农村阶级状况的基础上,制定了党在农业合作化运动中的阶级路线:依靠贫农和下中农,巩固地团结其他中农,通过由限制富农剥削到最后消灭富农剥削。从而,丰富和发展了马克思列宁主义关于农业合作化运动中阶级路线的理论。

由于农民长期处于个体生产的地位,私有观念和旧习惯势力的影响较深,对于社会主义要有一个逐步认识的过程。因而,在农业合作化运动中,必须坚持自愿和互利的原则,采取逐步前进的方针。我国的农业合作化,在党的领导下,根据生产发展的需要和农民的觉悟程度,采取了三个互相衔接的步骤和形式:第一步,成立带有某些社会主义萌芽的互助组;第二步,在互助组的基础上,组织以土地入股和统一经营为特点的小型的、带有半社会主义性质的初级农业生产合作社;第三步,在初级农业生产合作社的基础上,组织大型的、完全社会主义性质的高级农业生产合作社。采取这种逐步前进的步骤,不仅逐步地把农民的个体劳动者所有制改造成为社会主义劳动群众集体所有制,而且也避免了由于突然变化而可能引起的种种损失。我国在这个大变动中,农业生产不仅没有受到损失,而且得到了发展。农业总产值以一九五二年为一百,一九五六年则为百分之一百二十点四,一九五七年为百分之一百二十四点七。

农业合作化运动既然是一场深刻的社会革命,因此,它自始至终都是在尖锐而复杂的两个阶级、两条道路、两条路线的斗争中进行的。地主、富农利用一些富裕中农的资本主义自发倾向,用造谣、破坏,或者混入合作社篡夺领导权,打击和排斥贫下中农,来阻碍和破坏农业合作化运动的开展。刘少奇一类抛出了一条所谓“先机械化,后合作化”的修正主义路线,竭力宣扬反动的唯生产力论,以没有农业机械化就不能实行农业合作化为借口,诬蔑农业合作化是一种“错误的、危险的、空想的农业社会主义思想”,疯狂地反对进行农业社会主义改造,并且策划了“反冒进”的罪恶活动,大砍合作社,妄图把个体经济引向资本主义的邪道。毛主席在《关于农业合作化问题》中,阐明了党对农业社会主义改造的理论、路线和政策。毛主席指出:“在农业方面,在我国的条件下(在资本主义国家内是使农业资本主义化),则必须先有合作化,然后才能使用大机器。”“我们对于工业和农业、社会主义的工业化和社会主义的农业改造这样两件事,决不可以分割起来和互相孤立起来去看,决不可以只强调一方面,减弱另一方面。”从而,有力地批判了刘少奇一类的“先机械化,后合作化”的修正主义路线,粉碎了他们大砍合作社的罪恶活动,极大地激发了广大农民特别是贫下中农的社会主义积极性。到一九五六年,我国就基本上实现了农业合作化,五亿农民兴高采烈地走上了社会主义金光大道。

个体手工业,同小农经济一样,也是产生资本主义的温床。因而,对它也必须实行社会主义改造。个体手工业的社会主义改造,同样必须走合作化的道路,采取逐步前进的办法。由于个体手工业是一种比较纯粹的小商品经济,因而对它采取“从供销着手,实行由小到大,由低到高”的方针,先组织供销小组、手工业供销合作社,分散生产、联合经营,再发展到生产集中、规模较大、自负盈亏的手工业生产合作社。我国在一九五六年,在农业合作化高潮的推动下,基本上完成了手工业社会主义改造的任务。

我国无产阶级和劳动人民在毛主席的无产阶级革命路线指引下,凭借着无产阶级专政的政治力量和社会主义国营经济的经济实力,逐步地改造了民族资本主义所有制和个体劳动者所有制,到一九五六年基本上完成了生产资料所有制方面的社会主义改造,基本上挣脱了私有制的锁链,建立起社会主义的经济制度。在一九五六年,我国国民收入中,社会主义国营经济占32.2\%,合作社经济占53.4\%,国家资本主义经济占7.3\%,个体经济占7.1\%。

\section{社会主义社会是一个相当长的历史阶段}

\subsection{社会主义社会的基本矛盾仍然是生产关系和生产力、上层建筑和经济基础之间的矛盾}

在社会主义社会,无产阶级专政代替了资产阶级专政社会主义生产关系代替了资本主义生产关系和一切以私有制为基础的生产关系。它是一个什么性质的社会,具有什么特征呢?马克思指出:“在资本主义社会和共产主义社会之间,有一个从前者变为后者的革命转变时期。同这个时期相适应的也有一个政治上的过渡时期,这个时期的国家只能是无产阶级的革命专政。”这就是说,社会主义社会是从资本主义社会到共产主义社会的过渡时期,是通过无产阶级的革命专政实现从阶级社会向无阶级社会过渡的重要历史时期。

我们知道,在社会主义社会,生产资料的公有制代替了私有制,马克思主义已成为社会的指导思想。这表明社会主义社会已根本不同于资本主义社会,具有共产主义社会的因素。但是,它还不是完全的共产主义社会,只是共产主义社会的初级阶段,“我们这里所说的是这样的共产主义社会,它不是在它自身基础上已经发展了的,恰好相反,是刚刚从资本主义社会中产生出来的,因此它在各方面,在经济、道德和精神方面都还带着它脱胎出来的那个旧社会的痕迹。”由于社会主义社会在各个方面都还存在着资本主义的残余和痕迹,因此,它“不能不是衰亡着的资本主义与生长着的共产主义彼此斗争的时期。”无产阶级为了最后消灭阶级,实现共产主义,必须促进共产主义因素的生长,加速资本主义因素的衰亡,必须造成使资本主义既不能存在也不能再产生的经济、道德、精神的条件。而这是需要时间的。正如列宁曾经指出的:“为了完全消灭阶级,不仅要推翻剥削者即地主和资本家,不仅要废除他们的所有制,而且要废除任何生产资料私有制,要消灭城乡之间、体力劳动者和脑力劳动者之间的差别。这是很长时期才能实现的事业。”所以,社会主义社会是一个相当长的历史阶段。

为了清楚地认识社会主义社会的这种性质和特征,首先需要明确地认识社会主义社会的基本矛盾。

在基本上完成了生产资料所有制方面的社会主义改造,生产资料的社会主义公有制代替了私有制以后,社会主义社会的生产关系和生产力之间、上层建筑和经济基础之间是否还存在矛盾呢?在这个原则问题上,马克思主义和修正主义存在着根本的分歧。修正主义者竭力否认社会主义社会始终存在的生产关系和生产力之间、上层建筑和经济基础之间的矛盾,宣扬阶级斗争熄灭论,借以推行其复辟资本主义的修正主义路线。毛主席依据马克思主义的基本原理,总结了列宁以后无产阶级专政的新经验,提出了社会主义社会基本矛盾的理论,批判了修正主义。

毛主席指出:“对立统一规律是宇宙的根本规律。这个规律,不论在自然界、人类社会和人们的思想中,都是普遍存在的。”“在社会主义社会中,基本的矛盾仍然是生产关系和生产力之间的矛盾,上层建筑和经济基础之间的矛盾。”

在社会主义社会,生产资料所有制方面的社会主义改造基本完成以后,建立了以生产资料公有制为基础的社会主义生产关系。生产资料的社会主义公有制代替了私有制,无产阶级和劳动人民摆脱了被剥削、被奴役的地位,成为生产资料的主人;社会产品按照有利于劳动者的原则进行分配,这就消除了资本主义所固有的生产的社会性和资本主义私人占有制之间的对抗性矛盾,解放了生产力。所以,社会主义生产关系是和生产力的发展相适应的,能够容许生产力以旧社会所没有的速度迅速发展。

但是,在社会主义社会,生产关系和生产力之间除了有相适应的一面外,还有相矛盾的一面。这主要表现在以下两个方面。

在社会主义社会中,在一定时期内,还存在着非社会主义的生产关系。从所有制方面来看,在生产资料所有制的社会主义改造基本完成以后的一个相当时期内,资本家还凭借着对生产资料的所有权领取定息,城乡还不可避免地存在部分的私有制。在人和人的相互关系上,还存在着被推翻的地主资产阶级和新产生的资产阶级分子同劳动人民之间的阶级对立关系。在个人消费品的分配形式方面,一定时期内还向留用的资本家和资产阶级专家支付高薪。显然,这些非社会主义的生产关系不仅同生产力的发展相矛盾,而且同社会主义生产关系也是相矛盾的。

就社会主义生产关系来讲,“社会主义生产关系已经建立起来,它是和生产力的发展相适应的;但是,它又还很不完善,这些不完善的方面和生产力的发展又是相矛盾的”。这些“不完善的方面”表现在哪里呢?我们可以看到,在所有制方面,社会主义公有制的建立,是对私有制的否定,但是所有制问题还没有完全解决,社会主义公有制内部还存在着全民所有制和集体所有制的差别,资产阶级法权还没有完全取消,公有制还要巩固和发展。在人们的相互关系上,由于生产资料的私有制变为公有制,劳动人民之间建立起在根本利益上一致的互助合作关系,但是还存在着工业和农业之间、城市和乡村之间、脑力劳动和体力劳动之间的差别,资产阶级法权还严重地存在着。在消费品分配方面,社会主义的“各尽所能,按劳分配”,是对人剥削人的分配制度的否定,但是按劳动分配消费品这个平等的权利按照原则仍然是资产阶级法权,资产阶级法权在这里还占着统治地位。上述社会主义生产关系各方面存在的资产阶级法权,是社会主义生产关系的不完善的方面,是旧社会的痕迹。正如毛主席指出:“现在还实行八级工资制,按劳分配,货币交换,这些跟旧社会没有多少差别。”对于社会主义生产关系各方面存在的资产阶级法权,必须作历史的全面的分析,既要看到它存在的不可避免性,又要看到限制它的必要性。由于它以形式上的平等掩盖着事实上的不平等,是同生产力的发展相矛盾的,因此必须加以限制,并逐步创造条件以便将来最后加以消灭,从而使生产关系的不完善方面逐步完善起来,以适应生产力发展的需要。

在社会主义社会“除了生产关系和生产力发展的这种又相适应又相矛盾的情况以外,还有上层建筑和经济基础的又相适应又相矛盾的情况”。社会主义生产关系的总和构成社会主义的经济基础。社会主义的经济基础需要有同它相适应的上层建筑去维护它、巩固它。无产阶级专政的国家政权和用马克思列宁主义武装起来的无产阶级政党的领导,以马克思主义、列宁主义、毛泽东思想为指导的社会主义意识形态,这些社会主义的上层建筑是同社会主义经济基础相适应的,有力地促进社会主义经济基础的形成、巩固和发展。但是,社会主义社会的上层建筑和经济基础之间还有相矛盾的一面,表现在资产阶级和其他剥削阶级的意识形态、旧社会的习惯势力还严重地存在,代表党内外的新旧资产阶级的党内走资本主义道路当权派的存在;国家机关工作人员的官僚主义作风和由于资产阶级思想的腐蚀而产生的资产阶级生活作风的存在;以及国家制度中某些环节上缺陷的存在,这些都是和社会主义经济基础相矛盾的。必须不断地解决这些矛盾,使上层建筑进一步适应社会主义经济基础的巩固和发展的需要。

社会主义社会生产关系和生产力之间、上层建筑和经济基础之间的又相适应又相矛盾的情况,正表明了社会主义社会和资本主义社会有着本质的差别,但它们之间在有些方面又“没有多少差别”,在经济基础和上层建筑等方面都带有“旧社会的痕迹”。

社会主义社会的基本矛盾,同资本主义社会的基本矛盾,在性质上是根本不同的。因为,资本主义社会的基本矛盾,表现为剧烈的对抗和冲突,因而这种矛盾只有通过无产阶级暴力革命,用无产阶级专政代替资产阶级专政,用社会主义制度代替资本主义制度才能得到解决。而社会主义社会的基本矛盾,则是在确立了无产阶级专政和实现了生产资料社会主义公有制条件下的矛盾,它可以在党和国家的领导下,在马克思主义、列宁主义、毛泽东思想的指引下,以阶级斗争为纲,坚持党的基本路线,放手发动群众,自觉地调整和变革那些不适应的部分而得到解决。毛主席指出:“它不是对抗性的矛盾,它可以经过社会主义制度本身,不断地得到解决。”社会主义社会也正是在生产关系和生产力之间、上层建筑和经济基础之间的矛盾运动的推动下,不断地向前发展的。但是,这决不是说它不会转化为对抗性的矛盾。如社会主义公有制某些单位的蜕化变质,无产阶级专政机构某些部分被资产阶级,被党内的走资派所篡夺,矛盾就变为对抗性的了。当然,只要主要领导权还掌握在无产阶级革命家的手中,马克思主义路线在党和国家中还占据统治地位,这些局部范围内的对抗性矛盾可以通过社会主义制度本身来解决。如果一旦党和国家的主要领导权被资产阶级特别是党内的资产阶级,党内走资本主义道路的当权派所篡夺,修正主义路线占据了统治地位,社会主义制度就会演变为资本主义制度,社会的基本矛盾也就由非对抗性转化为对抗性,到那时候,只有再通过无产阶级起来革命,推翻修正主义的统治,才能得到解决。

\subsection{社会主义社会仍然是一个有阶级的社会}

社会主义社会的基本矛盾表明,社会主义社会仍然是一个有阶级的社会。这可以从以下两个方面来认识:

剥削阶级虽然被剥夺了生产资料,但作为阶级它还存在着。我们知道,人们对生产资料的不同关系,和在物质生产中的不同地位,是划分阶级的经济依据。但是,阶级不仅是一个经济范畴,而且是一个更广泛的社会范畴,必须从经济、政治和思想的统一来认识阶级的存在。在社会主义社会,虽然剥削阶级被剥夺了生产资料,但它的反动政治观点和思想意识还会长期存在,因而它作为阶级还始终存在着。事实上,在我国土地改革和生产资料所有制的社会主义改造基本完成以后,地主、资产阶级人还在,心不死,还继续同无产阶级较量,千方百计地企图恢复他们被夺去的“天堂”。

其次,社会主义社会还存在着产生资本主义的土壤和条件,会自发地产生出新的资产阶级分子。我们知道,在社会主义社会,除了国外的资本主义的阴风还会不时地吹来以外,在国内,商品生产、货币交换、按劳分配等方面资产阶级法权的存在,资产阶级思想意识的存在,是产生新的资产阶级分子的重要的土壤和条件。由于存在着部分私有制和两种社会主义公有制,农民还不可避免地保留着原来小生产者某些固有的特点,一部分富裕农民的资本主义自发倾向,必然导致产生新的资产阶级分子。同时,由于三大差别和资产阶级法权这些旧社会的痕迹的存在,资产阶级腐朽作风也会侵入到无产阶级队伍中,破坏社会主义的政治和经济关系,从而在工人阶级中、党员中、机关工作人员中产生出新的资产阶级分子。正如毛主席指出:“列宁说,‘小生产是经常地、每日每时地、自发地和大批地产生着资本主义和资产阶级的。’工人阶级一部分,党员一部分,也有这种情况。无产阶级中,机关工作人员中,都有发生资产阶级生活作风的。”事实上,随着城乡资本主义因素的发展,新资产阶级分子是经常不断地产生出来的。

由此可见,社会主义社会是一个有阶级的社会,不仅在于被打倒的地主资产阶级还依然存在;而且更重要的是在于社会主义社会还存在旧社会的痕迹,由于商品制度、货币交换、按劳分配等方面资产阶级法权的存在,还会产生出新的资产阶级分子。

社会主义社会既然是一个有阶级的社会,因而社会主义社会基本矛盾必然要集中表现为阶级矛盾和阶级斗争。从上述分析中我们知道,社会主义社会生产关系和生产力之间、上层建筑和经济基础之间存在不相适应的矛盾的一面,主要由于社会主义社会还存在旧社会的痕迹。社会主义经济基础还不稳固,在上层建筑各个领城里旧思想、旧习惯势力还顽强地阻碍着社会主义新生事物的成长。无产阶级是新的生产关系的代表,坚持走社会主义道路,要求逐步清除旧社会的痕迹,积极扶持社会主义新生事物,限制资产阶级法权,促进社会主义生产关系和上层建筑的巩固和不断完善。而资产阶级则坚持走资本主义道路,总是竭力维护和利用旧社会的痕迹,疯狂反对社会主义新生事物,破坏社会主义生产关系和上层建筑,妄图复辟资本主义。因而,在整个社会主义历史阶段,无产阶级和资产阶级的斗争、社会主义道路和资本主义道路的斗争就是客观存在的。无产阶级同资产阶级的矛盾是我们国内的主要矛盾。

“从资本主义过渡到共产主义是一整个历史时代。只要这个时代没有结束,剥削者就必然存在着复辟希望,并把这种希望变为复辟行动。”在整个社会主义历史阶段,无产阶级和资产阶级之间的阶级斗争,是长期的、曲折的,有时甚至是很激烈的。党内走资本主义道路的当权派,是被推翻的地主资产阶级和新产生的资产阶级分子的代表,是社会主义革命时期党内的资产阶级。由于社会主义社会还存在阶级、阶级矛盾和阶级斗争,还存在产生资本主义和资产阶级的土壤和条件,就总会在党内出现走资派,出现新的资产阶级的代表,“走资派还在走”的现象必将长期存在。毛主席最近指出,资产阶级“就在共产党内,党内走资本主义道路的当权派。走资派还在走。”因此我们必须清醒地认识到:整个社会主义历史阶段的主要矛盾是无产阶级同资产阶级的矛盾,主要危险是修正主义,最危险的是党内走资本主义道路的当权派。

毛主席指出:“天下大乱,达到天下大治。过七八年又来一次。牛鬼蛇神自己跳出来。他们为自己的阶级本性所决定,非跳出来不可。”阶级敌人、一切牛鬼蛇神是非跳出来不可的,每过若干年就有一次大的斗争。每经过一次大的斗争,就清除一些旧社会的痕迹,使社会主义经济基础和上层建筑得以巩固和发展,社会主义社会也就向前发展一步。只有经过多次的阶级较量,无产阶级才能完成最后消灭阶级、实现共产主义的伟大历史使命。

\subsection{无产阶级要逐步造成使资产阶级既不能存在也不能再产生的条件}

通过上面的分析,我们可以看到,在社会主义社会这个历史阶段中,始终存在着阶级、阶级矛盾和阶级斗争,存在着社会主义同资本主义两条道路的斗争,存在着资本主义复辟的危险性,以及存在着帝国主义、社会帝国主义进行颠覆和侵略的威胁。这就决定了社会主义社会有着两种发展可能性:一是前进到共产主义,一是倒退为资本主义。

社会主义革命,就是无产阶级反对资产阶级和一切剥削阶级的革命。革命的对象是资产阶级,重点是党内走资本主义道路的当权派。它的任务是要用无产阶级专政代替资产阶级专政,用社会主义战胜资本主义,并在长期的阶级斗争中逐步造成使资产阶级既不能存在又不能再产生的条件,最后消灭阶级,实现共产主义。

无产阶级为了造成使资产阶级既不能存在也不能再产生的条件,必须坚持经济领域里的继续革命,逐步完成所有制改造方面尚未完成的那一部分任务,逐步缩小以至最后消灭与阶级划分相联系的旧的分工,即工农之间、城乡之间、脑力劳动和体力劳动之间的差别,限制并逐步缩小各个方面存在的资产阶级法权的范围和作用,并积极创造条件以达到将来最后予以取消。还必须坚持上层建筑领域里的革命,深入批判修正主义,批判资产阶级,批判资产阶级法权思想,同传统的私有观念实行最彻底的决裂。正如马克思指出:社会主义“就是宣布不断革命,就是无产阶级的阶级专政,这种专政是达到消灭一切阶级差别,达到消灭这些差别所由产生的一切生产关系,达到消灭和这些生产关系相适应的一切社会关系,达到改变由这些社会关系产生出来的一切观念的必然的过渡阶段”。

马克思在论述阶级斗争和无产阶级专政的关系问题时,曾指出:“

\begin{enumerate}

\item 阶级的存在仅仅同生产发展的一定历史阶段相联系;
\item 阶级斗争必然要导致无产阶级专政,
\item 这个专政不过是达到消灭一切阶级和进入无阶级社会的过渡”。无产阶级为了不断削弱产生资本主义的基础,逐步造成使资产阶级既不能存在也不能再产生的条体,最后消灭阶级,就必须以阶级斗争为纲,实行无产阶级对资产阶级的全面专政,把无产阶级专政下的继续革命进行到底。

\end{enumerate}

无产阶级对资产阶级的全面专政,必须是在一切领域和革命发展的一切阶段里专政,决不能只在某些领域和某个阶段专政。如果丧失警惕,在“达到消灭一切阶级和进入无阶级社会的过渡”的路上停留下来,资本主义复辟将随时是可能的。

毛主席在揭示社会主义社会的基本矛盾和阶级斗争规律的基础上,提出了无产阶级专政下继续革命的理论,为我们党制定了一条整个社会主义历史阶段的基本路线。“社会主义社会是一个相当长的历史阶段。在社会主义这个历史阶段中,还存在着阶级、阶级矛盾和阶级斗争,存在着社会主义同资本主义两条道路的斗争,存在着资本主义复辟的危险性。要认识这种斗争的长期性和复杂性。要提高警惕。要进行社会主义教育。要正确理解和处理阶级矛盾和阶级斗争问题,正确区别和处理敌我矛盾和人民内部矛盾。不然的话,我们这样的社会主义国家,就会走向反面,就会变质,就会出现复辟。我们从现在起,必须年年讲,月月讲,天天讲,使我们对这个问题,有比较清醒的认识,有一条马克思列宁主义的路线。”党的基本路线明确地指出了社会主义社会还存在着阶级、阶级矛盾和阶级斗争,国内的主要矛盾是无产阶级和资产阶级两个阶级、社会主义道路和资本主义道路两条道路的斗争,这种斗争是长期的、复杂的,要警惕出修正主义,特别要警惕中央出修正主义。党的基本路线指明了无产阶级专政的历史任务,是要彻底战胜资本主义,消灭阶级,最终实现共产主义,因而必须对资产阶级实行全面专政。党的基本路线规定了必须严格区别和正确处理敌我矛盾和人民内部矛盾这两类不同性质的矛盾的总政策,团结一切可以团结的力量,同阶级敌人进行斗争,巩固无产阶级专政。毛主席制定的党在整个社会主义历史阶段的基本路线,正确地反映了社会主义社会发展的客观规律,是实现党的基本纲领的唯一正确的路线,是我们党的生命线,是照耀我们各项工作胜利前进的灯塔。

在毛主席关于无产阶级专政下继续革命理论的指导下,在党的基本路线的指引下,我国在一九五六年生产资料所有制方面的社会主义革命取得基本胜利后,在思想、政治、经济领域进行了一次又一次的社会主义革命,同资产阶级、同党内走资本主义道路的当权派开展了多次的斗争,使旧社会遗留下来的东西受到了猛烈的冲击,促进了社会主义生产关系和上层建筑的巩固和不断发展,推动了社会主义事业的前进。一九六六年开始的无产阶级文化大革命,是在社会主义条件下无产阶级反对资产阶级和一切剥削阶级的政治大革命。无产阶级文化大革命和批林批孔运动的伟大胜利,粉碎了以刘少奇、林彪为头子的两个资产阶级司令部,冲击了腐朽的意识形态和上层建筑的其他不适应的部分,对于巩固无产阶级专政,防止资本主义复辟,建设社会主义具有重大意义。毛主席指出:“现在的文化大革命,仅仅是第一次,以后还必然要进行多次。”只有这样,才能坚持以阶级斗争为纲,不断巩固文化大革命的胜利成果,加强无产阶级对资产阶级的全面专政,促进社会主义经济基础的巩固和发展,夺取社会主义事业的新胜利。

承认不承认在整个社会主义历史阶段中存在阶级和阶级斗争,要不要以阶级斗争为纲,要不要坚持无产阶级专政,要不要坚持无产阶级专政下的继续革命,这是马克思主义同修正主义斗争的焦点。修正主义者为反对马克思主义的无产阶级革命和无产阶级专政理论,炮制了反动的阶级斗争熄灭论和唯生产力论,片面地夸大生产力的决定作用,把社会变革单纯地理解为生产力发展的问题,根本抹煞了生产关系对生产力、上层建筑对经济基础的反作用和在一定条件下的决定作用,否认阶级斗争是社会发展的根本动力。老修正主义者伯恩施坦胡说什么“社会主义的胜利,……取决于社会财富或社会生产力的增长”,认为一个国家如果没有生产力的高度发展,资本主义不发达,无产阶级就不应该也不可能进行社会主义革命;竭力否认阶级斗争必然导致无产阶级专政,反对无产阶级暴力革命,反对彻底打碎旧的资产阶级国家机器。俄国十月革命胜利以后,考茨基之流宣扬“我们并没有达到甚至还没有走近建立社会主义的阶段……只有在生产力发展和繁荣的基础上,社会主义才是可能的”,竭力反对无产阶级社会主义革命,维护生产资料的私有制。刘少奇一类在我国由新民主主义革命转入社会主义革命的时刻,也大肆叫嚷什么我国工业不发达,资本主义还“在青年时代”,有“积极作用”,“社会主义是将来的事情,现在提得过早”,疯狂反对我国的社会主义改造。在生产资料所有制方面的社会主义革命基本完成以后,现代修正主义者又祭起唯生产力论这一破烂武器,否认无产阶级专政的必要性,反对无产阶级对资产阶级的全面专政。苏修叛徒集团竭力篡改马克思主义关于社会主义社会是从资本主义社会到共产主义社会的过渡时期的理论,说什么在社会主义生产关系确立以后,“社会主义就开始在自身的基础上发展起来”,“无产阶级专政国家在发达的社会主义社会建立过程中变成全民国家”,“党机关工作中主要的东西就是生产”,以此掩盖他们在苏联全面复辟资本主义、用资产阶级专政代替无产阶级专政的罪恶活动。刘少奇、林彪一类也宣扬什么“我国社会主义和资本主义谁战胜谁的问题,现在已经解决了”,国内的主要矛盾“是先进的社会主义制度同落后的社会生产力之间的矛盾”,“以后的主要任务是发展生产”,“国家基层政权在开始消亡”。他们反对以阶级斗争为纲,反对党的基本路线,反对无产阶级专政下的继续革命,竭力阻挠和破坏我们党在无产阶级专政下限制资产阶级法权的各种革命措施,攻击社会主义新生事物,竭力维护和扩大那些旧的东西,妄图利用这块产生资本主义的土壤培植一批新的资产阶级分子,同被打倒的地主资产阶级一起,来推翻无产阶级专政,复辟资本主义。国际共产主义运动和我们党的历史告诉我们:一场大革命开始的时候,修正主义者总是利用唯生产力论来反对革命,阻挡革命洪流前进;而当革命取得伟大胜利的时候,他们又总是搬出唯生产力论来否定这场革命,反对将革命进行到底。

恩格斯指出:“根据唯物史观,历史过程中的决定性因素归根到底是现实生活的生产和再生产。无论马克思和我都从来没有肯定过比这更多的东西。如果有人在这里加以歪曲,说经济因素是唯一决定性的因素,那末他就是把这个命题变成毫无内容的、抽象的、荒诞无稽的空话。”毛主席也指出:“诚然,生产力、实践、经济基础,一般地表现为主要的决定的作用,谁不承认这一点,谁就不是唯物论者。然而,生产关系、理论、上层建筑这些方面,在一定条件之下,又转过来表现其为主要的决定的作用,这也是必须承认的。”

批判反动的唯生产力论,决不是不要发展生产力。无产阶级为建设社会主义、实现共产主义,需要“尽可能快地增加生产力的总量”。但是,社会主义社会还存在着生产关系和生产力、上层建筑和经济基础之间的矛盾,存在着阶级和阶级斗争,存在着资本主义复辟的危险性。因此无产阶级必须遵循马克思主义的生产关系和生产力、上层建筑和经济基础的辩证统一论,以阶级斗争为纲,坚持无产阶级专政下的继续革命,以巩固和不断发展社会主义生产关系和上层建筑。“革命就是解放生产力,革命就是促进生产力的发展”,只有把革命搞好了,才能促进生产力的迅速发展。否则,社会主义建设就将被引入歧路,生产或者搞不上去,或者在一个时期暂时搞上去了,但最终还要垮下来,或者即使搞上去了,也会重蹈“卫星上天,红旗落地”的覆辙。

毛主席在关于无产阶级专政理论的重要指示中指出:“列宁为什么说对资产阶级专政,这个问题要搞清楚。这个问题不搞清楚,就会变修正主义。要使全国知道。”我们必须遵照毛主席的教导,“要多看点马列主义的书”,认真学习和弄通马克思主义关于无产阶级专政的理论,深入批判反动的唯生产力论,以阶级斗争为纲,更加自觉地贯彻执行党的基本路线,保卫文化大革命的胜利成果,坚持在一切领域和一切阶段上对资产阶级实行全面专政,把无产阶级专政下的继续革命进行到底。

\chapter{生产资料的社会主义公有制}

社会主义公有制的建立,是对私有制的根本否定,引起了生产关系的根本变革。但是,在社会主义社会的一个相当长的时期内,社会主义公有制内部还存在全民所有制和集体所有制的差别,存在着资产阶级法权,而且生产资料还有实际上归哪个阶级所有的问题,所有制的问题并没有完全解决。必须正确处理两种社会主义公有制之间及其内部的关系,自觉地为巩固和发展社会主义公有制而斗争。

\section{社会主义公有制是社会主义生产关系的基础}

\subsection{社会主义公有制存在全民所有制和集体所有制两种形式}

生产资料的所有制,是指劳动者与生产资料结合的方式,也就是生产资料归谁所有、由谁支配的问题。劳动者和生产资科是社会生产的基本要素,它们必须按一定的方式结合起来,社会生产才能正常地进行。它们结合的方式不同,生产资料所有制的性质也就不同。在人类社会发展的进程中,经历了两种不同类型的所有制,一种是私有制,一种是公有制。自奴隶社会开始,直到无产阶级革命取得胜利以前,生产资料基本上为少数剥削阶级分子所占有,他们利用所占有的生产资料,对劳动人民进行剥削和压迫。无产阶级夺取政权以后,有步骤地变生产资料私有制为社会主义公有制,把生产资料掌握在劳动人民的手里。

在社会主义社会的一个相当长的时期内,生产资料的社会主义公有制不可避免地存在两种形式即社会主义全民所有制和社会主义劳动群众集体所有制。与此相适应的,有两种社会主义经济:即国营经济和集体经济。

为什么在一个相当长的时期内,两种社会主义公有制的并存是不可避免的呢?这是由客现的历史经济条件和现阶段的生产力发展水平所决定的。我们知道,无产阶级夺取政权以后,遇到两种不同的私有制,即以剥削雇佣劳动为基础的资本主义所有制和不剥削他人的个体劳动者所有制。正如在上一讲中所讲的,根据马克思列宁主义的基本原理对这两种私有制必须采取不同的政策,通过没收和赎买的办法,变资本主义所有制为社会主义全民所有制;经过合作化的道路把个体劳动者所有制改造为社会主义劳动群众集体所有制。同时,这也是同工业和农业的生产力的不同发展水平相适应的。现代工业生产的特点是高度社会化,客观上要求各部门、各行业分工协作,协调地进行生产。要实现这一点只有社会直接占有生产资料,即建立社会主义全民所有制。而农业和手工业的生产力水平还比较低,建立社会主义劳动群众集体所有制则有利于组织生产和加强管理,有利于调动广大劳动者的社会主义积极性。因而在社会主义社会一个相当长的时期内,必然并存着两种社会主义公有制。

社会主义全民所有制和社会主义劳动群众集体所有制,就其经济性质来看,是属于同一类型的。首先,它们都是废除了生产资料私有制,消除了人剥削人的不合理现象,生产资料和产品由代表全体劳动人民的国家或集体所占有。其次,在集体劳动过程中,人们不论职位高低都是普通劳动者,互相帮助,互相促进。第三,它们都是为了满足国家和人民不断增长的需要,而有计划地进行生产和经营的。可见全民所有制和集体所有制,都是社会主义的公有制,实现了劳动者与生产资料的直接结合,是对私有制的革命否定。

但是,它们之间又存在着重大差别。它们之间的差别,最根本的在于公有化的程度不同。我们知道,社会主义全民所有制是由无产阶级专政的国家代表全体劳动人民占有生产资料的一种所有制形式,是全社会范围内的公有化,而社会主义劳动群众集体所有制是由一个集体单位内的劳动群众共同占有生产资料的一种所有制形式,是一个集体范围内的公有化。由于生产资料公有化程度的不同,这两种社会主义公有制的差别,具体表现在以下几个方面:首先,国营经济的生产资料和产品属于全体劳动人民所有,可以由国家根据社会和人民的需要进行统一调度;而集体经济的生产资料和产品则归该集体的劳动者所有,国家对集体经济的生产资料和产品不能进行无偿调拨,就是各个集体经济之间也不能进行无偿调拨。其次,国营经济必须严格按照国家规定的统一计划进行生产和经营,全面完成和超额完成国家规定的各项计划指标,而集体经济虽然也必须服从国家的计划指导,但它的生产和经营则是在国家计划的指导下,按照党的方针、政策,因地制宜地安排的。第三,国营经济的利润全部上缴给国家,职工的工资收入也由国家根据整个社会生产发展和政治经济情况统一规定的;而集体经济的积累和消费的比例,以及劳动报酬的标准,则是由各个集体经济按照国家规定的原则,结合自己的条例来规定的,劳动者的收入水平同所在的集体经济的生产发展水平有直接的联系。

社会主义的全民所有制和劳动群众集体所有制都是社会主义公有制,它们的根本利益是一致的。在社会主义国营经济的领导下,两种社会主义经济紧密地联系起来,互相支援,互相促进,共同发展。但是,由于两种社会主义所有制有着重大的差别,它们之间还存在着矛盾的一面。这种矛盾是国家与集体、工业与农业以及工农两个劳动阶级,在根本利益一致基础上的人民内部矛盾。

正确处理全民所有制与集体所有制的矛盾,必须把国家利益和集体利益、整体利益和局部利益正确地结合起来。既要把国家利益、整体利益放在第一位,同时又要兼顾集体利益、局部利益;既要坚持社会主义方向和国营经济的领民,又要尊重集体经济的特点,坚持自愿互利、等价交换的原则。总之,我们必须以阶级斗争为纲,坚持党的基本路线,从加强工农联盟和巩固无产阶级专政出发,采取正确的政策,认真处理全民所有制与集体所有制之间的矛盾,以密切城乡经济联系,促进工农业生产及整个国民经济的迅速发展,使社会主义公有制巩固和不断发展。

\subsection{社会主义公有制代替私有制,但所有制问题并没有完全解决}

无产阶级通过社会主义革命,把生产资料私有制转变为社会主义公有制,引起了生产关系的根本变革。

马克思主义认为,生产关系“包括:

  \begin{enumerate}

\item 生产资料的所有制形式;
\item 由此产生的各种不同社会集团在生产中的地位以及他们的相互关系,或如马克思所说的,‘互相交换其活动’;
\item 完全以它们为转移的产品分配形式。”

    \end{enumerate}

在生产关系的这三个方面中,所有制形式是生产关系的基础,它决定生产关系的性质,决定人们的相互关系和分配形式。同时,人们的相互关系和分配形式又反作用于所有制,而且在一定条件下它们还要起决定作用。

由于生产资料实现了社会主义公有制代替私有制,无产阶级和劳动人民成了生产资料的主人,消除了资本主义制度下直接生产者与生产资料相分离的现象,从根本上消灭了工农之间、城乡之间、脑力劳动和体力劳动之间的对立。他们为着建设社会主义、巩固无产阶级专政而共同劳动,在根本利益一致的基础上建立起社会主义的相互关系。同时,随着社会主义公有制和社会主义相互关系的建立,在个人消费品分配方面,消灭了资本主义的人剥削人的分配制度,而按照有利于劳动者的方式进行。因此生产资料的社会主义公有制是社会主义生产关系的基础。

建立在社会主义公有制基础上的社会主义生产关系,比资本主义生产关系具有无比的优越性。毛主席指出:“我国现在的社会制度比较旧时代的社会制度要优胜得多。如果不优胜,旧制度就不会被推翻,新制度就不可能建立。所谓社会主义生产关系比较旧时代生产关系更能够适合生产力发展的性质,就是指能够容许生产力以旧社会所没有的速度迅速发展,因而生产不断扩大,因而使人民不断增长的需要能够逐步得到满足的这样一种情况。”

但是,社会主义公有制的建立,并不意味着所有制问题已经完全解决。我们可以从两个方面来认识这个问题。

在所有制范围内资产阶级法权还没有完全取消。具体表现在:首先在社会主义社会的一定时期内,生产资料所有制方面除了占统治地位的社会主义公有制外,还存在着部分的私有制。在我国现阶段,根据生产力的发展水平和人们的思想觉悟程度,在保证人民公社集体经济的发展和占绝对优势的条件下,人民公社社员可以经营少量的自留地和家庭副业,牧区社员可以有少量的自留畜。它对于充分利用劳动力,增加社会产品,满足人民需要具有一定的作用。另外,国家还允许非农业的个体劳动者在城镇街道组织、农村人民公社的生产队统一安排下,从事在法律许可范围内的,不剥削他人的个体劳动。

其次,在社会主义社会的一个相当长的时期内,社会主义公有制也不都是全民所有制,而是两种所有制。集体所有制只是在一个集体范围内的公有化,每一个集体经济单位占有自己的生产资料和产品,都是独立的所有者。它同私有制相比较,那种私人占有生产资料的不平等是取消了。但是,各个集体单位占有生产资料的数量和质量(如土地的数量和肥沃程度等)还存在着差别,在其他条件相同的情况下,所能获得的产品和收益却是不同的。因此,各个集体单位在生产资料的占有关系上,仍然存在着事实上的不平等。这正说明了在小生产基础上建立起来的集体所有制本身,还存在着资产阶级法权。

另一方面,不论是全民所有制还是集体所有制,即有一个领导权的问题,即生产资料实际上归哪个阶级所有的问题。我们知道,所有制问题,不能只看它的形式,还要看它的实际内容,即要看领导权掌握在什么人的手里,执行的是什么路线,代表着哪个阶级的利益。领导权掌握在真正的马克思主义者和劳动群众的手里,就能够贯彻执行马克思主义路线,坚持无产阶级政治挂帅,批判修正主义,批判资产阶级,限制资产阶级法权,就能够代表无产阶级和劳动人民的利益,社会主义公有制就能够巩固和不断发展。反之,如果领导权被走资派和资本主义思想严重的人所把持,不把劳动人民当作社会主义企业的主人,维护、扩大和强化资产阶级法权,大搞修正主义的管、卡、压、罚,推行“利润挂帅”和“物质刺激”,把资本主义原则搬到社会主义经济关系中来,势必会损害、削弱社会主义公有制,甚至蜕变为资本主义私有制。苏修叛徒集团正是凭借其篡夺的党和国家的领导大权,推行“利润挂帅”、“物质刺激”等一整套修正主义路线,变社会主义公有制为官僚垄断资产阶级所有制。刘少奇、林彪一类也是这样程度不同地改变了一批工厂企业的社会主义性质。正如毛主席指出:“看来,无产阶级文化大革命不搞是不行的,我们这个基础不稳固。据我观察,不讲全体,也不讲绝大多数,恐怕是相当大的一个多数的工厂里头,领导权不在真正的马克思主义者、不在工人群众手里。过去领导工厂的,不是没有好人。有好人,党委书记、副书记、委员,都有好人,支部书记有好人。但是,他是跟着过去刘少奇那种路线走,无非是搞什么物质刺激,利润挂帅,不提倡无产阶级政治,搞什么奖金,等等。”“但是,工厂里确有坏人。”“就是说明革命没有完”。

因此,社会主义公有制建立以后,革命并不能就此结束。巩固和发展社会主义公有制,是无产阶级和劳动人民的根本利益所在。我们必须以阶级斗争为纲,牢记党的基本路线,坚持无产阶级专政下的继续革命,继续在较长时期内逐步完成所有制改造方面尚未完成的那一部分任务,把社会主义公有制经济的领导权牢牢地掌握在马克思主义者和劳动群众的手里,限制资产阶级法权,批判资产阶级,批判修正主义,巩固和完善社会主义全民所有制,促进社会主义集体所有制的巩固和发展,在一个比较长的时间内,随着生产力的发展和人们觉悟程度的提高,逐步地实现向社会主义全民所有制过渡。

\subsection{社会主义公有制的建立,为人们自觉地认识和运用经济规律提供了可能性}

在生产资料公有制的基础上,产生了社会主义经济规律。

经济规律是不以人们的意志为转移的社会经济发展过程的客观规律。我们知道,社会经济不是孤立的、静止的,而是相互联系、彼此制约,并按着一定的规律运动的。这个反映社会的经济现象和经济过程的本质,表明社会经济发展过程内在的因果联系和依赖关系的规律,就是经济规律。

经济规律既然是反映社会经济现象和经济过程本质的,因而它是在一定的经济条件的基础上产生和发生作用的,并随着产生这种经济规律的经济条件的消失而失去作用。正是这样,有的经济规律是在各个社会形态都起作用的,如生产关系一定要适合生产力性质的规律。有的经济规律是在几个社会形态中起作用的,如与商品生产相联系的价值规律,凡是有商品生产存在的地方,也就有价值规律的存在并发生作用。而多数经济规律只是在某一社会形态中起作用,如剩余价值规律、竞争和生产无政府状态规律、资本积累和无产阶级贫困化规律等,只能在资本主义条件下起作用。随着生产资料社会主义公有制的建立,社会主义生产关系代替资本主义生产关系,体现资本主义生产关系的资本主义经济规律也就失去作用,而体现社会主义生产关系的社会主义经济规律,如社会主义基本经济规律、国民经济有计划按比例发展规律、国民经济高速度发展规律等,也就产生并发生作用。

社会主义经济规律是反映社会主义经济发展过程的客观规律。它和其他经济规律一样,都具有客观性,是不以人们的意志为转移的。如果以为人们可以“创造”、“改造”规律,那就陷入了唯心沦的泥坑。无产阶级政党和国家制定路线、方针和政策,做计划,必须以客观规律为依据,按照客观规律办事。

客观规律是不能“改造”或“创造”的,但是,人们在客观规律面前也决不是无能为力的,而是可以认识和利用它们的。在社会主义制度下,生产资料公有制的建立,为人们自觉地认识和利用经济规律提供了广泛的可能性。正如恩格斯指出:“人们自己的社会行动的规律,这些直到现在都如同异己的、统治着人们的自然规律一样而与人们相对立的规律,那时就将被人们熟练地运用起来,因而将服从他们的统治。”“这是人类从必然王国进入自由王国的飞跃。”

可是,要把这种可能变为现实,还要经过一个反复实践和认识的过程。“人的正确思想,只能从社会实践中来,只能从社会的生产斗争、阶级斗争和科学实赊这三项实践中来。”人们必须在三大革命斗争实践中,经过由实践到认识、认识到实践的多次反复,逐步克服盲目性,提高自觉性,以达到对客观经济规律的正确认识。这种认识过程,是同人们世界观的改造过程分不开的。

在存在阶级的社会里,经济规律是在激烈的阶级斗争中贯彻和发挥作用的。我们知道,经济规律作用的结果,必将直接关系到社会各阶级的阶级利益,因而利用经济规律无论何时何地都有阶级背景的。在社会主义社会中,无产阶级给予社会主义经济规律以充分发生作用的广阔场所,限制某些经济规律发生作用的范围,来加速社会主义建设,巩固和发展社会主义生产关系,加强无产阶级专政。这就必然要遭到资产阶级和其他社会腐朽势力的反抗,特别是修正主义路线的干扰和破坏。因此,无产阶级自觉运用经济规律的过程,也就是无产阶级和资产阶级、马克思主义路线和修正主义路线斗争的过程。

刘少奇、林彪一类竭力宣扬主观唯心主义的反动的“天才论”,胡说什么“我的话就是真理”,“打破旧规律,创造新规律”,其反革命用心在于使无产阶级不能自觉地来认识和运用经济规律,而由他们任意违反社会主义经济发展规律,推行修正主义路线,以实现其复辟资本主义的迷梦。我们必须深入批判刘少奇、林彪一类的谬论,在改造客观世界的过程中改造主观世界,经过反复的实践和认识,更加自觉地认识和运用经济规律,来促进社会主义社会向着共产主义社会的方向前进。

\chapter{社会主义生产的基本特征}

在社会主义制度下,由于生产资料公有制的建立,社会生产的目的是为了满足整个社会和人民群众不断增长的需要,国民经济是有计划按比例和高速度发展的。这是社会主义生产的基本特征。但是,社会主义社会是生长着的共产主义和衰亡着的资本主义彼此斗争的时期,社会主义生产是在生产关系和生产力、上层建筑和经济基础的矛盾运动中进行的。因而,我们必须以阶级斗争为纲,贯彻执行毛主席的革命路线,坚持无产阶级专政下的继续革命,为实现国民经济有计划按比例、高速度的发展而努力。

\section{社会主义生产的目的}

\subsection{社会再生产过程是物质资料再生产和生产关系再生产的统一过程}

物质资料的生产是人类社会存在和发展的基础。马克思指出:“一个社会不能停止消费,同样,它也不能停止生产。”任何一个社会要存在和发展,都必须使物质资料的生产过程不间断地进行下去。当我们把物质资料的生产过程作为一个连续的、不断更新的过程来看,就是再生产过程。

物质资料的生产和再生产,总是在一定的生产关系下进行的。生产资料的所有制不同,生产和再生产的目的及其实现形式也就不同。在资本主义社会,由于生产资料为资本家私人占有,生产和再生产的目的是为了保证资本家摄取最大限度的利润;是在竞争和无政府状态中,通过经常的波动和周期性的经济危机而盲目地自发地进行的。而在社会主义社会,由于生产资料的公有制代替了私有制,废除了人剥削人的制度,生产和再生产的目的是为了满足整个社会和人民群众不断增长的需要,是有计划按比例、高速度地进行的。

在社会再生产的过程中,除了社会产品的再生产以外,还进行着劳动力的再生产。在资本主义社会,“工人阶级的不断维持和再生产始终是资本再生产的条件。资本家可以放心地让工人维持自己和繁殖后代的本能去实现这个条件。他所操心的只是把工人的个人消费尽量限制在必要的范围之内”。资本主义社会的劳动力再生产是服从于资本家榨取剩余价值的需要的,失业、饥饿和过早衰亡是它固有的特点。而在社会主义社会,劳动人民成了社会的主人,不仅能够有计划地进行社会产品的再生产;而且可以通过计划生育,自觉地培养和造就一支有高度无产阶级思想觉悟的、掌握现代科学技术的又红又专的产业大军,有计划地进行劳动力的再生产。

在社会再生产的过程中,社会的生产关系也不断地被再生产着。在资本主义社会,“把资本主义生产过程联系起来考察,或作为再生产过程来考察,它不仅生产商品,不仅生产剩余价值,而且还生产和再生产资本关系本身:一方面是资本家,另一方面是雇佣工人。”随着资本主义再生产的进行,一方面一小撮剥削者占有越来越多的生产资料,另一方面广大劳功者则日益贫困化,资本主义制度所固有的矛盾不断尖锐化,导致资本主义走向灭亡。而在社会主义社会,随着再生产的进行,社会主义生产关系适应生产力的发展不断变革而更加完善,共产主义因素逐步地成长起来。

由此可见,社会再生产过积是物质资料再生产和生产关系再生产的统一过程。

但是,社会主义再生产虽有巨大的优越性,但它是不能自发地实现的。由于社会主义社会存在着阶级和阶级斗争,在各个方面存在着旧社会的痕迹,因而它要在党和国家依据对社会主义社会基本矛盾的深刻认识,自觉地解决生产关系和生产力、上层建筑和经济基础之间的矛盾,通过无产阶级战胜资产阶级、马克思主义路线战胜修正主义路线来实现的。

\subsection{社会主义生产的目的是满足整个社会和人民群众的需要}

任何社会的生产,都不是为生产而生产,而具有一定的目的。生产的目的不是由人们的主观意志决定的,而是取决于生产资料所有制的性质。生产资料所有制的性质不同,生产目的也就不同。

在资本主义社会,生产资料为资本家私人占有。这就决定了资本主义生产是服从于资本家的私人利益,以从工人身上榨取剩余价值,追求最大限度的利润为目的。马克思指出:“资本主义生产的始终不变的目的,是用最小限度的预付资本生产最大限度的剩余价值和剩余产品”。

在社会主义社会,由于消灭了生产资料的资本主义私有制,建立了社会主义公有制,无产阶级和劳动人民成了社会的主人。因而,社会主义生产是为劳动人民的利益服务,是以满足整个社会和人民群众不断增长的需要为目的的。这种需要主要有以下几个方面的内容:

是无产阶级和广大人民群众的物质和文化生活方面的需要。正如恩格斯指出的,在社会主义制度代替资本主义制度以后,“通过社会生产,不仅可能保证一切社会成员有富足的和一天比一天充裕的物质生活,而且还可能保证他们的体力和智力获得充分的自由的发展和运用”。

是巩固无产阶级专政、巩固国防的需要。由于社会主义社会还存在着阶级和阶级斗争,存在着资本主义复辟的危险性,存在着帝国主义和社会帝国主义进行颠覆和侵略的威胁,因而,社会主义国家必须巩固无产阶级专政和巩固国防。满足这方面的需要,是满足无产阶级和广大人民群众物质和文化生活方面需要的重要前提。

是支援世界人民革命斗争的需要。已经获得革命胜利的社会主义国家的人民,应当援助正在争取解放的世界各国人民的革命斗争,这是自己应尽的国际主义义务。

由此可见,社会主义生产的目的,既包括满足劳动人民的个人需要,又包括满足体现着劳动人民的长远利益和整体利益的整个社会的需要。

苏修叛徒集团拼命歪曲社会主义生产的目的,竭力宣扬什么“一切为了人,为了人的幸福”。他们之所以用资产阶级的经济主义和个人主义偷换社会主义生产的目的,是为了欺骗苏联人民,用个人主义的幸福观来麻痹无产阶级和劳动人民的革命意志,从而忘记阶级斗争、忘记革命,以便他们搞资本主义复辟。事实完全揭露了他们的谎言,正是这伙叛徒在苏联全面复辟了资本主义,社会主义公有制变成了新型的官僚垄断资产阶级所有制,给苏联劳动人民带来了沉重的苦难。林彪一类也大肆鼓吹这套黑货,攻击社会主义制度是“国富民穷”,实质上就是要步苏修的后尘,妄图在中国复辟资本主义。

毛主席早在我国解放战争时期,提出革命根据地财政经济工作的第一个原则是;“发展生产,保障供给”。在我国进入社会主义革命时期以后,毛主席又进一步总结了社会主义革命和社会主义建设的历史经验,提出了“备战、备荒、为人民”的伟大战略方针。毛主席的这些重要指示,全面地反映了社会主义生产的客观目的,深刻地揭露了修正主义的谎言和欺骗。它把劳动人民的当前利益和长远利益,把劳动人民的个人需要和巩固无产阶级专政的整体需要,把搞好国内的社会主义建设和支援世界人民的革命斗争紧密地结合起来,为社会主义生产建设指出了明确的方向。

\subsection{革命是生产发展的动力}

为了实现社会生产的目的,需要有相应的手段。社会主义生产的目的是满足整个社会和人民群众不断增长的需要,为了实现这一目的,决不能象资本主义制度那样地采取剥削本国人民和掠夺其他国家人民的办法,而只能在无产阶级专政国家的领导下,依靠无产阶级和劳动人民的努力,不断地发展社会主义生产。

怎样才能发展社会主义生产呢?毛主席指出:“革命就是解放生产力,革命就是促进生产力的发展”。

我们知道,新的科学技术和新的生产工具的采用,对于生产的发展具有重要的作用。但是,科学技术、生产工具是由劳动者创造和掌握使用的,只有在一定的生产关系下面才能得到保护、使用和发展。经过社会主义革命,以生产资料公有制为基础的社会主义生产关系的建立,为生产力的发展开辟了广阔的道路。但是,社会主义生产关系和生产力之间是又相适应又相矛盾的,因而社会主义生产关系建立以后,还需要适时地调整生产关系中那些不完善的、同生产力发展相矛盾的方面,以促进生产力的进一步发展。

同时,社会主义社会的上层建筑和经济基础之间也是又相适应又相矛盾的,只有及时地调整上层建筑中的不完善和不适应的部分,以巩固和发展社会主义经济基础,从根本上提高广大群众的觉悟,才能促进社会主义生产的发展。

由于社会主义社会是一个有阶级的社会,无产阶级和劳动人民调整或变革生产关系同生产力发展不相适应的部分、调整或变革上层建筑同经济基础不相适应的部分,增强和发展共产主义因素,逐步削弱和铲除资本主义因素,必然要引起一切新老资产阶级的反抗和破坏。因而,调整或变革生产关系和上层建筑,是在两个阶级、两条道路、两条路线的激烈斗争中进行的,是一场极为深刻的社会革命。

由此可见,在社会主义社会只有以阶级斗争为纲,坚持无产阶级专政下的继续革命,深入开展政治、思想、经济战线上的社会主义革命,及时地调整或变革生产关系和上层建筑中不完善、不适应的部分,巩团和发展社会主义的生产关系和上层建筑,以保证生产的社会主义方向,调动一切积极因素,才能促进社会主义生产不断的发展。否则,如果不抓革命,只埋头搞生产、搞技术,不仅生产上不去,而且会偏离社会主义方向,资本主义就会复辟。所以,“革命是历史的火车头”,革命是生产发展的动力。

现代修正主义者竭力歪曲马克思主义关于生产关系和生产力、上层建筑和经济基础相互关系的基本原理,把社会主义生产的发展仅仅归结为生产技术和生产工具的改进,否认阶级斗争是社会主义社会发展的动力,也是社会生产力发展的动力,否认劳动群众在生产中的决定作用,鼓吹“生产第一”、“技术第一”、“专家治厂”,贩卖反动的唯生产力论。这种谬论的实质,是抹煞社会主义时期客观存在的阶级斗争,反对无产阶级专政下的继续革命,否定革命统帅生产,以达到颠覆无产阶级专政、复辟资本主义的罪恶目的。

毛主席彻底批判了他们的反动谬论,为我们党制订了“抓革命,促生产”的伟大方针。这一方针科学地阐明了革命和生产,精神和物质,上层建筑和经济基础,生产关系和生产力之间的辨证关系,正确地反映了社会主义生产发展的客观规律,指引着我们取得社会主义革命和建设事业的伟大胜利。

我们在上面分析了社会主义生产的目的和实现这一目的的手段,实际上也是分折了社会主义基本经济规律的主要内容。

我们知道,任何一个社会,除了有反映自己生产、分配、交换和消费等方面和过程的特点的各个经济规律以外,还有一个基本经济规律。这个基本经济规律的主要内容包括社会生产的目的和达到这一目的的手段两个方面。由于社会生产的每一个方面和过程不是彼此孤立的,而是统一于社会生产的整体之中,是在一定的生产目的和相应的手段的支配下进行的。因而,反映社会生产的目的和达到这一目的的手段这样两方面内容的基本经济规律,就具有决定社会生产的一切主要方面和一切主要过程的作用。

社会主义社会同其它社会一样,在其经济运动过程中有一个基本经济规律存在并发生作用。它要求社会主义生产的目的必须是满足整个社会和人民群众不断增长的需要;达到这一目的的手段必须是深入开展社会主义革命,及时地调整或变革生产关系和上层建筑,加速技术改造,多快好省地发展社会主义生产。

社会主义基本经济规律决定着社会主义经济发展的方向,决定着生产、分配、交换和消费等方面的活动。在进行生产时,决定生产什么,生产多少,生产怎样布局等,都必须服从这一规律的要求;在安排分配时,规定积累和消费的比例及其内部的构成,也必须服从这一规律的要求;在组织交换时,内销与外销、供应城市与供应乡村的比例的规定,各类商品比价的确定,也都必须服从这一规律的要求。总之,在组织社会主义经济活动的整个过程中,都必须遵循社会主义基本经济规律的要求。而一切“利润挂帅”、“生产第一”、“自由经营”、“自由价格”、“物质刺激”、“分光吃尽”等修正主义的邪门歪道,都是违背社会主义基本经济规律的要求的,都是对社会主义生产目的和与之相适应的实现目的的手段的一种反动,其结果必然会瓦解社会主义的经济基础,导致资本主义的复辟。

社会主义基本经济规律是在社会主义经济条件的基础上产生并发生作用的,但在社会主义社会的一个相当长的时期内,它的作用程度还有一定的局限性。这是因为,生产资料的社会主义公有制还存在着全民所有制和集体所有制两种形式。这两种社会主义经济的生产目的,虽然都是为了满足整个社会和人民群众不断增长的需要,但由于公有化程度的不同,满足需要的具体情况是有所不同的。农村集体经济除了考虑整个社会的需要之外,还要考虑本集体的需要。两者之间既有统一的一面,又有矛盾的一面。这说明社会主义生产在满足整个社会和人民群众不断增长的需要方面,还受到社会主义公有制存在两种形式的影响或局限。随着社会主义革命的深入,社会主义集体所有制将由小而大、由低到高地向社会主义全民所有制逐步发展,社会主义基本经济规律的作用将日益充分地得到发挥。

正确地认识社会主义基本经济规律,可以使我们把握住社会主义经济发展的方向,分清什么是社会主义、什么是资本主义,分清什么是马克思主义、什么是修正主义,更好地执行毛主席的革命路线、方针和政策,加速社会主义经济的发展。

\section{社会主义国民经济的有计划按比例发展}

\subsection{社会主义经济是计划经济}

国民经济是由许多生产部门、行业和经济单位组成的统一体。它们之间存在着精细的社会分工,存在着密切的联系和依赖关系。社会再生产的正常进行,要求按照一定的比例在国民经济的各个部门、行业之间分配生产资料和劳动力,也就是说要求在国民经济的各个部门、行业之间保持一定的比例关系,否则社会生产就要陷入混乱而无法顺利进行。马克思指出:“要想得到和各种不同的需要量相适应的产品量,就要付出各种不同的和一定数量的社会总劳动量。这种按一定比例分配社会劳动的必要性,决不可能被社会生产的一定形式所取消,而可能改变的只是它的表现形式,这是不言而喻的。”

在社会主义条件下,由于以下几个方面的原因,国民经济有计划按比例的发展就显得更为必要。首先,社会主义制度使生产的社会化获得了进一步的发展,如果没有一个保证国民经济按比例发展的统一计划,社会主义的社会化大生产就将陷入混乱之中,而不能顺利地发展。其次,只有国民经济有计划按比例的发展,使人力、物力、财力得到最充分最有效的利用,才能保证国民经济的高速度发展,才能使社会主义生产的目的得到完满地实现。第三,各个部门、企业也只有在国家的统一计划下,才能使自己的经济活动有利于社会主义经济的发展,才能实现彼此之间互相支援、互相协作的社会主义关系。第四,实行国家的统一计划,是限制资产阶级法权,克服资本主义自发倾向,从而保证社会主义经济沿着正确方向前进的必要条件之一。正如列宁指出的:“没有一个使千百万人在产品的生产和分配中最严格遵守统一标准的有计划的国家组织,社会主义就无从设想。”“如果对于产品的生产和分配不实行全面的国家计算和监督,那末劳动者的政权,劳动者的自由,就不能维持下去,资本主义压迫制度的复辟,就不可避免。”

社会主义制度下,国民经济有计划按比例的发展不仅是必要的,而且也是完全可能的。我们知道,在资本主义社会,生产资料的资本主义私有制把整个国民经济分割成各个资本家所有的私人企业,他们为了追逐利润而彼此进行着激烈的竞争,生产什么、生产多少,完全取决于价格的高低和利润的大小。生产资料和劳动力在国民经济各个部门、行业之间的分配,是由价值规律自发地调节的。因而,社会再生产所需要的比例关系,只能在自发的波动中,通过经常性的比例失调和周期性的经济危机来强制地实现。正如列宁指出的:“资本主义必须经过危机来建立经常被破坏的平衡”。而在社会主义制度下,由于生产资料公有制的建立,消除了资本主义所固有的生产的社会性和资本主义私人占有制之间的矛盾,国民经济的各个部门、行业在根本利益一致的基础上结成一个统一的有机整体,共同为满足整个社会和人民群众的需要而进行生产。因而,社会再生产所需要的比例关系,完全可以由无产阶级专政的国家用统一的计划自觉地进行调节。恩格斯早就指出过:“一旦社会占有了生产资料,……社会生产内部的无政府状态将为有计划的自觉的组织所代替。”

由此可见国民经济有计划按比例的发展是社会主义经济运动的客观规律,它是在生产资科公有制的基础上,作为竞争和无政府状态规律的对立物而产生的社会主义的经济规律。这一规律在客观上要求:社会主义国民经济必须由整个社会进行有计划的统一领导而按比例地协调发展。

社会主义经济是计划经济。正是这一规律的作用,使得社会主义国家有可能实行计划经济,有计划地在国民经济中分配生产资料和劳动力,组织和指导社会主义经济按比例的发展。实行计划经济,是社会主义制度区别于资本主义制度的一个显著标志,是社会主义制度优越性的一个重要方面。

资产阶级经济学家编造的“有计划的资本主义”和现代修正主义者鼓吹的国家垄断资本主义可以实行“计划化”,都是否认国民经济计划化同社会制度之间的必然联系,抹煞资本主义制度所固有的矛盾,混淆社会主义和资本主义两种社会制度的根本区别,妄图取消无产阶级革命。

\subsection{要从国民经济复杂的比例关系中,找出主要的比例关系}

国民经济有计划按比例的发展,要求有计划地在国民经济各个部门、行业之间,按照一定的比例关系分配生产资料和劳动力。可是,社会主义国民经济包括生产、分配、交换和消费的各个环节;包括物质资料生产部门加工业、农业、建筑业、货物运输业等;也包括非物质资料生产部门如商业、物资供应、财政金融和文化、教育、卫生、科研等,上述各个部门又是由许多企业或单位所组成而分布在全国各个地区,它们之间存在着复杂的比例关系。为了实现国民经济有计划按比例的发展,就必须从复杂的比例关系中找出主要的比例关系。

根据马克思主义关于再生产的基本原理,结合我国社会主义的实践经验,在国民经济复杂的比例关系中,有以下一些主要的比例:

生产资料生产和消费资料生产之间的比例关系。社会生产分为两大部类:生产资料生产和消费资料生产。这是根据社会产品最终用于生产消费还是用于生活消费来划分的。社会主义再生产的特点,是扩大再生产。为了实现扩大再生产,必须使生产资料生产较快地增长。但是,生产资料生产的较快增长,不能离开消费资料生产的相应发展。否则,不仅生产资料生产的较快增长得不到保证,迟早会遇到困难而不能继续增长下去,而且也是不符合社会主义制度的要求的。因此,生产资料生产的较快增长,要有消费资料生产的相应发展。

社会生产两大部类之间的比例关系基本上可以由重工业、轻工业和农业这三个部门之间的比例关系体现出来。重工业、轻工业、农业是最基本的物质资料生产部门,它们是根据各个物质资料生产部门在劳动对象和生产方法等方面的不同特点来划分的。这三个部门虽然同时都进行着两个部类的生产,但重工业主要是生产生产资料的,而轻工业和农业则主要是生产消费资料的。因此,农、轻、重之间的比例关系,基本上反映了社会生产两大部类之间的比例关系,又补充了社会生产两大部类的划分所不能表示出来的复杂的不同生产部门之间的比例关系。所以,坚持“以农业为基础、工业为主导”的方针,正确处理农业、轻工业和重工业之间的比例关系,在较快发展重工业的条件下,加速农业和轻工业的发展,实行发展工业和发展农业同时井举,就基本上可以使社会生产两大部类之间保持相适应的比例关系,从而保证整个国民经济的迅速发展。

在正确处理农业、轻工业、重工业比例关系的基础上,还必须正确处理农业内部的农、林、牧、副、渔各业之间,以及工业内部的采掘、原料、动力、冶炼等基础工业与加工工业之间的比例关系,实行“以粮为纲”和“以钢为纲”的方针,带动整个农业和工业的全面发展,才能促进国民经济的全面跃进。

积累和消费之间的比例关系。社会总产品扣除用来补偿生产中消耗掉的生产资料的补偿基金以后,剩下的新创造的国民收入通过分配和再分配最终形成积累基金和消费基金。正确处理积累和消费之间以及它们内部的比例关系,使积累的增长与人民生活的适当地改善相适应,使基本建设与当前生产相适应,使生产建设与文教、卫生、科研事业的发展相适应,使经济建设与国防建设相适应,对于社会主义再生产的顺利进行和社会主义建设事业的迅速发展是极为重要的。

交通运输业、商业、物资供应等部门同生产建设和人民生活之间的比例关系。交通运输业是生产建设的先行。它同商业和物资供应部门都是联结生产与生产、生产与消费的纽带。正确地处理交通运输业同生产建设之间的比例关系,正确地安排商品购销计划同工农业生产发展和人民生活需要之间的比例关系,是保证国民经济顺利发展的重要条件。

生产建设、文教卫生事业同人口增长之间的比例关系。物质资料的有计划生产和文教卫生事业的有计划发展,在客观上要求人口的增长必须消灭无政府状态,实行计划生育,计划增长。这是有计划地发展社会主义经济建设,不断提高人民的健康水平,安排好人民生活的必要条件。

在全国范围内,各地区之间和各地区内部合理地配置生产力,也是保证国民经济有计划按比例发展的一个重要方面。正确处理各个部门在地区分布上保持协调的比例关系,合理地配置生产力,对于巩固国防,加强各民族的团结,充分利用资源,节约社会劳动耗费,加快社会主义建设的步伐和逐步缩小三大差别,都具有非常重要的意义。在我国,为了使各个部门在地区分布上保持协调的比例关系,使各地区的经济得到协调的发展,必须正确处理沿海工业和内地工业的关系,把工业建设的重点转向内地,首先注意内地工业的发展,同时也要适当地发展沿海工业,充分发挥沿海老工业基地的作用。在各个地区内部,也应当保持工业和农业全面协调地发展。工业发展的地区,必须抓紧发展农业,否则农业上不去,就会拖住工业的后腿,不能充分发挥工业基地的作用。原来的农业地区,必须积极发展工业,否则就不能迅速地改变落后面貌。各地区经济只有全面地协调发展,才能更好地发挥本地区的特点。

上面只是对国民经济中主要比例关系作了简要的概述,有的在以下各讲里还要进一步加以分析。

安排国民经济发展中的各种比例关系,必须抓住主要矛盾。实践证明,农业、轻工业和重工业之间的比例关系就是其中带有战略性和决定意义的比例关系。只要处理好这三者之间的比例关系,就为正确安排整个国民经济的比例关系奠定了牢固的基础。

\subsection{加强计划工作,搞好综合平衡}

国民经济有计划按比例发展规律的作用,是要通过党和国家的自觉作用,通过国民经济计划工作来实现的。

国民经济计划工作的基本内容和主要任务,是社会主义国家在认识和掌握社会主义客观经济规律的基础上,根据党的路线、方针和政策,通过编制、执行、检查和调整国民经济计划,提出发展国民经济的具体任务,安排好国民经济的比例关系,实现对国民经济的计划领导,组织和动员广大群众多快好省地建设社会主义。

社会主义国民经济的比例关系究竟应当服从于什么样的任务,并不是由国民经济有计划按比例发展规律决定的,而是决定于社会主义基本经济规律的要求,以及各个时期的政治经济条件。因此,国民经济计划工作不仅要正确地反映国民经济有计划按比例发展规律的要求,而且要反映社会主义基本经济规律的要求,执行党的路线、方针和政策。

国民经济有计划按比例发展的规律,为社会主义国家对国民经济实行计划领导提供了可能性。但是,要把这种可能性变为现实性,就必须充分发挥人们的主观能动作用,调查研究,弄清情况,在客观条件许可的范围内尽可能地使编制的国民经济计划比较正确地反映客观经济规律的要求和符合客观的实际情况。

同时,在社会主义社会,由于还存在着商品生产,价值规律也就必然存在并发生作用。在社会主义制度下,虽然社会生产是由国民经济计划调节的,价值规律也可以被人们所认识和自觉地利用来作为完成国家计划的工具,但价值规律毕竞是商品生产的经济规律,不可避免地还要发生一定的消极作用,使一些企业热衷于生产产值大、利润高的产品,冲击国民经济计划。

因此,国民经济计划工作,必须以阶级斗争为纲,坚持党的基本路线,坚持无产阶级政治挂帅,加强党的一元化领导,执行群众路线,充分发挥人们的主观能动作用,限制价值规律的消极作用,热情支持社会主义新生事物,不断总结经验,搞好国民经济的综合平衡。

综合平衡,是国民经济计划工作的一个基本方法。它的主要任务,就是依据社会主义客观经济规律的要求,依据党的路线、方针和政策以及当时的政治经济任务,对国民经济各个部门和各个方面进行统筹兼顾、适当安排,正确地安排国民经济的发展速度和比例关系,组织好社会生产和社会需要之间的平衡,以保证国民经济有计划按比例、高速度地发展。

我们知道,社会主义国民经济的发展,要求各个部门、各个方面经常保持一定的比例关系。但是,在国民经济发展过程中,各个部门、各个方面的速度是不可能完全一样的,这就必然会不断地打破原有的比例关系,出现不平衡的情况。因而,不平衡是经常的、绝对的。但是,国民经济的发展,又要求必须保持相适应的比例关系,以暂时的、相对的平衡作为发展的必要条件。综合平衡就是要克服国民经济发展过程中经常出现的不平衡,自觉地组织起相对的平衡,以促进国民经济的发展。从不平衡到平衡,再出现新的不平衡,矛盾不断出现,又不断解决,正是国民经济有计划按比例发展规律的表现形式。列宁指出:“经常的、自觉地保持的平衡,实际上就是计划性”。

综合平衡的过程,就是揭露矛盾、分析矛盾和解决矛盾,推动社会主义经济前进的过程。在进行综合平衡时,对于国民经济发展过程中的薄弱环节和先进环节之间的矛盾,应当采取积极态度,克服薄弱环节,使它向先进环节看齐,由不平衡转化为新的平衡,推动国民经济持续地高速度发展。但是,克服薄弱环节,必须使它能够得到必要的人力、物力、财力,充分发挥人们的主观能动作用,进行一系列艰苦细致的工作。积极平衡和消极平衡,不是一个单纯的计划工作方法问题,而是两个阶级、两条路线斗争在计划工作中的具体体现。

“我们的方针是统筹兼顾、适当安排。”进行综合平衡,必须正确处理需要和可能的关系,从实际出发,充分发挥人们的主观能动作用;必须正确处理全局和局部的关系,发挥中央和地方两个积极性,在中央的统一领导下使地方能办更多的事情;必须正确处理重点和一般的关系,保证重点环节的发展,切忌分散力量,同时也要照顾到一般的需要,为其相应发展提供必要的条件;必须正确处理长远和当前的关系,计划安排不能只顾一时,要瞻前顾后,远近结合,计划指标既要先进,又要留有余地。

有计划地领导国民经济的发展,是社会主义经济的根本原则,是巩团和发展社会主义经济基础、加强无产阶级专政、防止资本主义复辟和建设社会主义的重要措施。因此,一切修正主义者为了复辟资本主义,都力图瓦解和取消社会主义计划经济。刘少奇一伙竭力鼓吹“经济自由化”,恶毒地诬蔑计划经济“只有呆板的计划性,而没有灵活性和多样性”,而要搞什么“自由市场、自由价格、自由竞争”;林彪一类也胡说什么“需要就是计划”,“打仗就是比例”,他们的目的是妄图取消社会主义计划经济,破坏社会主义革命和社会主义建设,复辟资本主义。可见,在要不要坚持社会主义计划经济的问题上,一直存在着马克思主义路线和修正主义路线的激烈斗争。我们必须批判修正主义,坚持社会主义计划经济,使国民经济计划真正成为巩固无产阶级专政和建设社会主义的工具。

\section{社会主义国民经济的高速度发展}

\subsection{社会主义制度使生产以旧社会所没有的速度迅速发展}

国民经济的高速度发展,是社会主义经济的基本特征之一,是社会主义制度优越于资本主义制度的显著标志。

高速度地发展社会主义经济,这是无产阶级所必须解决的一项重要历史任务。由于帝国主义时期发展不平衡规律的作用,社会主义只能首先在一国或几国内取得胜利。已经取得社会主义胜利的国家,只有高速度地发展社会主义经济,增强经济实力和国防威力,才能有效地战胜国内外阶级敌人的反抗、侵略和颠覆,巩固无产阶级专政;才能更好地支援世界人民的革命斗争。同时,只有高速度地发展社会主义经济,才能逐步提高人民群众的物质和文化生活水平,促进社会主义公有制的巩固和发展;才能为逐步缩小和最后消灭工农之间、城乡之间、脑力劳动和体力劳动之间的差别,限制和最后消灭资产阶级法权,最终战胜资产阶级,过渡到共产主义创造必要的物质条件。

列宁在十月革命前夕,曾经就夺取政权后的无产阶级要不要高速度地发展社会主义经济的问题尖锐地指出:“或是灭亡,或是开足马力奋勇前进。历史就是这样提出问题的。”我国是一个发展中的社会主义国家,这个问题更尖锐地摆在我们面前。因而,必须以阶级斗争为纲,坚持党的基本路线,充分利用通过斗争赢得的时间和条件,自力更生,奋发图强,尽快地把我国国民经济搞上去。

在社会主义社会,国民经济的高速度发展不仅是必要的,而且也是可能的。因为,社会主义制度的建立,为国民经济的高速度发展提供了这种可能性。

社会主义制度使生产力中最重要的因素广大的劳动者获得了解放。劳动人民开始掌握了自己的命运,以主人翁的态度从事社会生产,充分发挥出建设社会主义的积极性和创造性。这是推动国民经济高速度发展的最强大力量。

社会主义制度使无产阶级专政的国家能够用统一的计划指导整个国民经济按比例地发展。国民经济有计划按比例地发展,消灭了资本主义的竞争和生产无政府状态所造成的人力、物力、财力巨大浪费,使社会资源得到合理的安排和充分的利用。这就使国民经济得以高速度地发展。

社会主义制度使科学技术能得到迅速的发展和广泛的应用。在资本主义制度下,新的科学技术由于被资本家所垄断,服务于资本家攫取剩余价值的目的,而不能得到广泛的应用,并且新的科学技术的采用也必将给劳动者带来各种各样的灾难。社会主义制度打破了资本主义制度的种种局限,为科学技术的迅速发展和广泛应用开辟了广阔的道路。这就使国民经济能够在现代化科学技术的基础上高速度地发展。

社会主义制度使社会生产和社会需要之间形成相互促进的关系。社会主义生产的目的是为了满足整个社会和人民群众不断增长的需要,因此摆脱了资本主义所特有的那种生产无限扩大的趋势同劳动群众购买力相对缩小的对抗性矛盾,摆脱了资本主义条件下所不可避免的生产过剩的经济危机,为国民经济的高速度发展扫除了障碍。

由此可见,国民经济的高速度发展是社会主义制度优越性的表现,是社会主义经济运动的客观规律。

社会主义国民经济的高速度发展是一种必然的趋势,但这并不意味着发展速度每年都一样,而没有任何的起伏。这是因为,社会主义社会还存在着阶级和阶级斗争,存在着旧社会的痕迹,存在着限制或破坏国民经济高速度发展的因素。例如,生产关系方面存在的资产阶级法权对生产发展的限制,资产阶级及在党内的走资派的干扰和破坏,资本主义自发倾向和旧习惯势力的阻挠,严重自然灾害的袭击,以及人们对客观规律认识能力的限制等因素的制约,因而社会主义国民经济的发展不可能是一帆风顺的,而是在激烈的两个阶级、两条道路、两条路线的斗争中,不断地克服各种矛盾的斗争中进行的。这就决定了国民经济的发展速度不可能每年都是一样,而是有的年份增长多一些,有的年份增长少一些,形成波浪式的前进运动。这种起伏是高速度发展中的起伏,它同资本主义制度下由于周期性地爆发经济危机所造成的生产严重倒退和生产增长相交错的情况,是根本不同的。

\subsection{鼓足干劲,力争上游,多快好省地建设社会主义}

社会主义制度使整个国民经济有可能有计划按比例,高速度地发展,但要把这一可能性变为现实,必须经过我们主观上的努力,必须有一条马克思主义的路战。

毛主席总结了国内外社会主义建设的经验和教训,在一九五八年制定了“鼓足干劲,力争上游,多快好省地建设社会主义”的总路线。

这条总路线坚持了无产阶级政治挂帅。“政治是统帅,是灵魂。”“政治工作是一切经济工作的生命线。”在社会主义建设中,如果离开了党的基本路线,不以阶级斗争为纲,不坚持无产阶级政治挂帅,一切经济、技术工作就会失掉灵魂,就不能沿着正确的方向前进。列宁指出:“政治同经济相比不能不占首位。不肯定这一点,就是忘记了马克思主义的最起码的常识。”“如果说(或者只是间接地表达了这种思想)从政治上看问题和‘从经济上’看问题有同等的价值,‘二者’都可以采用,这就是忘记了马克思主义的最起码的常识。”“一个阶级如果不从政治上正确地处理问题,就不能维持它的统治,因而也就不熊解决它的生产任务。”党的社会主义建设总路线要求我们在社会主义建设中,必须以阶级斗争为纲,坚持党的基本路线,加强党的领导,加强政治思想工作,批判修正主义,批判资产阶级,正确地处理革命和生产的关系,用革命统帅生产,使社会主义建设沿着正确的方向迅速发展。

这条总路线发扬了党的群众路线。人民群众是历史的创造者。在党的正确领导下,充分发挥人民群众的主观能动作用,这是马克思主义的一条基本原理。社会主义建设是人类历史上空前宏伟的事业,是亿万人民群众自觉的事业,如果不充分发挥广大人民群众建设社会主义的积极性和创造性,就不可能有生气勃勃的社会主义建设。“鼓足干劲,力争上游”,讲的是人的精神状态、主观能动性,是要充分调动广大人民群众建设社会主义的积极性。党的社会主义建设总路线要求我们在社会主义建设中必须坚定地相信群众,依靠群众,尊重群众的首创精神,在一切工作中都必须大搞群众运动,要善于从本质上发现群众的社会主义积极性,满腔热情地支持群众中涌现的社会主义新生事物,集中群众的智慧和经验,把群众最大限度地组织和动员起来,使社会主义建设真正成为亿万人民群众生气勃勃地自觉创造历史的伟大事业。

这条总路线强调了独立自主、自力更生的方针。“独立自主、自力更生”,是社会主义革命和社会主义建设的战咯方针。一个国家没有政治独立,就不可能获得经济独立,而没有经济独立,政治独立也是不完全、不巩固的。只有建立起独立的强大的社会主义经济,才能巩固无产阶级专政,永远立于不败之地。党的社会主义建设总路线要求我们进行社会主义建设,必须依靠本国人民的辛勤劳动和艰苦奋斗,充分利用本国的各种资源,挖掘国内的一切潜力,把争取社会主义建设的胜利放在自己力量的基点上;要破除迷信,解放思想,打破洋框框,走独立自主地发展社会主义工业、农业、科学技术事业的道路,对外国的好经验可以借鉴,先进技术可以引进,但决不能代替自己的创造。

这条总路线正确地反映了客观经济规律的要求。党的社会主义建设总路线要求我们在社会主义建设中把多快好省几个方面统一起来,这正是社会主义基本经济规律、国民经济有计划按比例发展规律和国民经济高速度发展规律的要求。多是对数量的要求,快是对时间的要求,好是对质量的要求,省是对节约劳动耗费的要求。多快好省是一个辩证的统一整体,这几个方面是互相制约的,不切实际的“多快”,既不可能多快,也不能好省,不能搞;但反过来,好是好,省是省,就是那么一点点,慢的要死,那也不行。为了实现多快好省,就要求我们在社会主义建设中,必须实行“以农业为基础、工业为主导”,发挥中央和地方两个积极性,利用沿海工业,加速建设内地工业,工农业并举,轻重工业并举,土法生产和洋法生产并举,大、中、小并举等一整套“两条腿走路”的方针和政策,彻底克服社会主义建设中一条腿走路的片面性。

显然,毛主席制定的党的社会主义建没总路线,是充分发挥人们的主观能动作用,充分利用社会主义制度的优越性,多快好省地建设社会主义的一条马克思主义路线,是高速度地发展社会主义建设事业的可靠保证。

在党的“鼓足干劲,力争上游,多快好省地建设社会主义”的总路线的指引下,一九五八年我国国民经济的发展,出现了蓬蓬勃勃的大跃进局面。无产阶级文化大革命和批林批孔运动的胜利,社会主义革命的不断深入,更进一步地促进了国民经济持续的全面跃进。我国的农业已连续获得了第十四个丰收年,粮食总产量比解放初期增长了一倍多,农业机械化水平不断提高,农业的生产条件正在逐步改变。工业生产突飞猛进,钢的产量比解放初期增长了一百多倍,为建成一个独立的比较完整的工业体系打下了牢固的基础。交通运输、商业、财政金融以及文化、教育、卫生、科学技术等事业也都获得了迅速的发展。我国依靠自己的力量,成功地进行了氢弹试验,发射了人造地球卫星,自行设计和制造了许多大型精密设备和现代化的重要工程项目。在我们这样一个近八亿人口的国家里,不仅保证了人民吃穿的基本需要,而且广大人民群众的物质文化生活逐步得到了改善。我国已成为世界上少有的财政收支平衡,既无外债,又无内债,市场繁荣,物价稳定的国家。我国国民经济全面跃进的客观事实,雄辩地证明了党的社会主义建设总路线是完全正确的。

社会主义革命和社会主义建设的伟大胜利,使我国由原来的“一穷二白”变成一个初步繁荣昌盛的社会主义国家。这就引起了国内外阶级敌人的惊慌和仇恨,他们对总路线和在总路线光辉照耀下出现的大跃进进行了恶毒的攻击和破坏。刘少奇一伙疯狂叫嚷“总路线可以讨论,可以推翻”,恶毒攻击大跃进是“小资产阶级狂热性”;林彪一类大肆咒骂总路线、大跃进“过极了”,破坏了“个人积极性”,竭力诬蔑我国国民经济是“停滞不前”。他们妄图抹煞我国国民经济在社会主义建设总路线指引下所取得的成就,否定社会主义制度的优越性,取消党的总路线,为资本主义的复辟开辟道路。我国社会主义建设欣欣向荣、蒸蒸日上的景象,是对刘少奇、林彪一类的无耻谰言最有力的驳斥。

我们已经取得了伟大的成就,但我国国民经济的发展水平还是不高的,在许多方面还比较落后。我们不能走世界各国技术发展的老路,跟在别人后面一步一步地爬行。我们必须打破常规,尽量采用先进技术,在一个不太长的历史时期内,把我国建设成为一个社会主义的现代化的强国。我们要在以毛主席为首的党中央的正确领导下,以阶级斗争为纲,坚持党的基本路线,坚持无产阶级政治挂帅,批判修正主义,批判资产阶级,贯彻执行“鼓足干劲,力争上游,多快好省地建设社会主义”的总路线,大搞群众运动,加快社会主义建设的步伐,争取提前实现在本世纪内把我国建设成为社会主义的现代化强国的宏伟目标。

\chapter{社会主义生产中人与人的相互关系}

人们在生产中的相互关系,是生产关系的一个重要组成部分。它是由生产资料的所有制决定的,但又具有很大的能动作用。在生产资料社会主义公有制的基础上,以阶级斗争为纲,坚持无产阶级专政下的继续革命,正确处理人们的相互关系,使之不断完善,对于巩固和完善社会主义公有制,推动社会生产力的发展,加强无产阶级专政,防止资本主义复辟,具有很大的意义。

\section{人们在社会主义生产中相互关系的形成及其阶级性质}

\subsection{人们的相互关系既决定于所有制又反作用于所有制}

人们在生产中的相互关系,就是人们在生产中所处的地位,以及他们之间互相交换活动的关系。它是生产关系的一个重要组成都分。

马克思主义认为,生产资料所有制形式是生产关系的基础,生产资料所有制的性质,决定着人们在生产中的相互关系。

在资本主义社会,资本家占有生产资料,工人则一无所有而不得不出卖劳动力来维持生存。这就决定了资本家在生产中处于统治、剥削的地位,而雇佣工人则处于被统治、被剥削的地位。马克思指出:“经济关系的无声的强制保证资本家对工人的统治。”在生产资料资本主义所有制的基础上,形成了资产阶级和无产阶级之间的剥削和被剥削、统治和被统治的阶级对立的关系。各个资本家之间,他们为着瓜分剩余价值,则存在着竞争和冲突的关系。

在社会主义社会,由于生产资料的社会主义公有制代替了私有制,人们在生产中的相互关系开始发生深刻的变化。社会主义公有制的建立,工人阶级和广大劳动人民成为生产资料的占有者,从根本上摆脱了被统治、被剥削的地位。剥削阶级则丧失了剥削劳动人民的手段,迫使他们不得不接受工人阶级和广大劳动人民对他们的统治和改造。旧社会里的那种关系被颠倒过来了,工人阶级和广大劳动人民同剥削阶级之间的关系成为统治和被统治、改造和被改造的关系。同时,工人阶级和广大劳动人民成了社会的主人,“在改变了的环境下,除了自己的劳动,谁都不能提供其他任何东西”。他们互相交换劳动,为巩固无产阶级专政、建设社会主义和共产主义的革命目标而共同斗争。工业和农业两个基本物质资料生产部门之间,也就是工人和农民两个劳动阶级之间,工人生产各种农业机器、化肥、农药和日用工业品,支援农业生产的发展和满足农民生活的需要,农民生产粮食、原料和其他农副产品,支援工业生产的发展和满足城市居民生活的需要;各个企业、各个部门、各个地区之间,在生产协作、产品交换、先进经验和先进技术等方面相互支援、相互促进;企业内部的领导和群众之间,管理人员、技术人员和直接生产者之间,在不同岗位上,共同管理企业。劳动人民正是在这根本利益一致的基础上,通过各种形式互相交换劳动,彼此之间形成了同志式的互助合作关系。

人们在生产中的相互关系虽然决定于生产资料所有制的性质,但却具有很大的能动作用,它反作用于所有制,并且在一定条件下还会起着决定的作用。例如,在资本主义社会,资本家为了巩固和发展资本主义所有制,就必须维护和发展资本家统治和奴役工人的关系。如果资本家不能强制工人按照自己的意志来行动,资本主义的剥削就不能实现,资本主义所有制也就不能巩固和发展。在社会主义社会,只有按照社会主义原则建立和发展人们的相互关系,企业的社会主义方向才能得到保证,社会主义公有制才能巩固和发展。因此,一定的生产资料所有制形式建立以后,它的进一步巩固和发展,都是同建立和完善与其相适应的相互关系,紧密联系在一起的。人们在生产中的相互关系如果处理得好,可以促进所有制的巩固和发展;如果处理得不好,就会对所有制起瓦解和破坏的作用。

但是,人们的相互关系并不会随着所有制的建立就能自然地巩固和发展,它还必须依靠上层建筑的巨大反作用。这是因为,人们的相互关系固然是由一定的生产资料所有制形式决定的,但所有制只是为相互关系的形成提供了经济条件。任何社会的统治阶级总是要利用上层建筑的力量,来巩固和发展同所有制相适应的相互关系。在资本主义社会,正如马克思指出的:资产阶级“不能单纯依靠经济关系的力量,还要依靠国家政权的帮助才能确保自己榨取足够的剩余劳动的权利”。资本主义发展的初期,资产阶级就是依靠上层建筑的力量,通过规定血腥的法律,用暴力手段迫使劳动者接受被统治、被剥削的关系。在社会主义社会,新型的社会主义相互关系的巩固和发展,要战胜剥削阶级的反抗,清除旧社会遗留下来的痕迹,克服残留在人们头脑中的剥削阶级意识形态,就更加必须依靠社会主义上层建筑的力量,坚持对资产阶级的全面专政,深入进行社会主义教育。我国正是在生产资料所有制方面的社会主义革命取得基本胜利以后,继续深入开展政治战线、经济战线和思想战线上的社会主义革命,社会主义相互关系才得以逐步地巩固和发展起来。

由此可见,社会主义相互关系是在生产资料社会主义公有制的基础上产生的,但它的巩固和发展又影响着社会主义公有制的巩固和发展;社会主义公有制建立以后,还必须依靠社会主义上层建筑的巨大反作用,社会主义相互关系才能巩固和发展。因此,那种只看到所有制在生产关系中的决定作用,不承认相互关系对所有制的能动作用,不承认上层建筑的反作用的观点,是极为错误的。我们必须坚持无产阶级专政下的继续革命,不断地调整和完善社会主义相互关系,以利于社会主义公有制的巩固和发展,促进社会生产力的迅速发展。

\subsection{人们在社会主义生产中的相互关系仍然贯穿着阶级矛盾和阶级斗争}

在阶级社会里,“人和人之间的关系,归根到底是阶级和阶级之间的关系”。由于社会主义社会还存在着阶级、阶级矛盾和阶级斗争,因而人们在生产中的相互关系也必然表现为阶级关系,反映着阶级矛盾和阶级斗争。

毛主席指出:“在社会主义社会里,在完成生产资料所有制的社会主义改造以后,阶级矛盾仍然存在,阶级斗争并没有熄灭。”我们必须坚持马克思主义关于阶级和阶级斗争的观点,运用阶级分析的方法,来正确地认识人们在社会主义生产中相互关系的阶级性质。

人们在社会主义生产中的相互关系,从总体上看,存在两类性质不同的关系,一是劳动阶级同剥削阶级之间的关系,二是劳动人民内部的关系。毛主席教导我们:“在我们的面前有两类社会矛盾,这就是敌我之间的矛盾和人民内部的矛盾。这是性质完全不同的两类矛盾。”

首先,劳动阶级同剥削阶级之间的相互关系。这也就是工人阶级、集体农民及其所属的知识分子同地主官僚资产阶级、资产阶级及其所属的知识分子之间的相互关系。

在我国,经过新民主主义革命和社会主义革命,地主、官僚资产阶级已被推翻和打倒,民族资产阶级已丧失了生产资料。但是,不仅它们作为阶级还仍然存在着,而且新的资产阶级分子又在旧的土壤上不断地产生出来。在社会主义生产中,工人阶级、集体农民同地主官僚资产阶级、资产阶级之间,是统治和被统治、改造和被改造的对立的阶级关系。这些剥削阶级处于被统治的地位,被迫在生产中按不同的方式接受改造。工人阶级和集体农民处于统治的地位,对地主官僚资产阶级按敌我矛盾处理,对具有两面性的民族资产阶级按人民内部矛盾处理,迫使这些剥削阶级中的大多数人在长期的劳动中逐步地改造成为自食其力的劳动者。而剥削阶级的本性,决定他们是不会甘心接受被统治、被改造的地位的,必然要进行各种形式的反抗,并妄图恢复他们的统治地位,重新剥削和压迫劳动人民。因而,这种改造必然是一个长期的反复的斗争过程。

其次,劳动人民内部的相互关系。这主要包括工人阶级内部、农民阶级内部、工农两个劳动阶级之间的相互关系。

在社会主义社会,无产阶级和广大劳动人民是社会的主人,有着共同的革命目标,根本利益是一致的,在生产中建立起新型的同志式的互助合作关系。这种同志式的互助合作关系,是社会主义生产中人们相互关系的主要方面。它对于巩固社会主义公有制,加强无产阶级专政,促进社会生产力的发展起着重要作用。

但是,在劳动人民内部的相互关系中,还存在着旧社会遗留下来的工农之间、城乡之间、脑力劳动和体力劳动之间的差别,还严重地存在着资产阶级法权。这主要表现在:

由于两种社会主义公有制的并存,在工业中占主要地位的是社会主义全民所有制,在农业中占主要地位的是社会主义集体所有制,两者的公有化程度不同,技术装备程度也不同,因而工业和农业之间还存在着本质的差别,工人和农民还是两个不同的劳动阶级。工人的劳动生产率比农民要高,他们在收入上也存在着差别。

由于两种社会主义公有制的并存,国营经济和农村集体经济是产品的不同所有着,工农之间、城乡之间人们相互交换劳动还必须采取商品交换的方式,遵循等价交换的原则。

由于旧的社会分工的存在,脑力劳动和体力劳动之间的差别的存在,不仅领导和群众之间、技术管理人员和直接生产者之间,而且直接生产者之间,也都存在着等级的差别。此外,社会主义社会还存在着资产阶级和其他剥削阶级思想意识的影响和腐蚀。这就会使一些人,特别是知识分子,傲视工农群众、轻视体力劳动;使一些领导人员滋生等级观念和特权思想,不以平等态度对待群众,当官作老爷,把同志式的关系变成统治和服从的关系;还会使有的人热衷于“利润挂帅”等修正主义黑货,背离社会主义方向,把社会主义协作关系变成金钱和竞争的关系。

以上这些,正表明了在劳动人民内部的相互关系中还存在着旧社会的痕迹,存在着资产阶级法权。它表明了劳动人民内部的同志式的互助合作关系还是不完善的、不巩固的。

由于劳动人民内部的相互关系还不完善,因而也就必然存在着矛盾。一般说来,它是在根本利益一致基础上的人民内部矛盾。但是,社会主义社会存在着阶级和阶级斗争,劳动人民内部的相互关系必然要受到无产阶级和资产阶级这个主要矛盾的支配和影响。这不仅是因为劳动人民内部矛盾的主要根源就是旧社会即资本主义的残余或痕迹,而且资产阶级特别是党内的资产阶级,党内走资本主义道路的当权派,为了复辟资本主义,总是竭力利用劳动人民内部相互关系方面存在的旧社会的痕迹,利用资产阶级法权,破坏社会主义相互关系。因此,劳动人民内部的相互关系,也必然反映着无产阶级和资产阶级的矛盾和斗争。在一般的情况下,人民内部的矛盾不是对抗性的。但是如果处理不适当,或者失去警觉,麻痹大意,也可能发生对抗。我们必须采用民主的说服教育的方法,正确处理和解决人民内部矛盾。

由此可见,人们在社会主义生产中的相互关系仍然具有阶级性质,贯穿着阶级矛盾和阶级斗争。为了巩固和发展社会主义的相互关系,就必须深入开展两个阶级、两条道路、两条路线的斗争。现代修正主义者竭力鼓吹人们的关系是什么“同志、朋友和兄弟”的关系,出现了什么“无差别的境界”;林彪一类也鼓吹什么“四海之内皆兄弟了”,其实质是抹煞社会主义社会存在的阶级和阶级斗争,使人们放弃阶级斗争和路线斗争,为他们变社会主义相互关系为资本主义相互关系,全面复辟资本主义开辟道路。对他们的阴谋,必须彻底揭露,深入批判。

\subsection{在斗争中巩固和发展社会主义的相互关系}

国际国内无产阶级专政的历史经验告诉我们,在无产阶级专政的条件下,党内资产阶级一般并不公开地直接改变生产资料的社会主义公有制,而是利用他们所篡夺的权力,扩大资产阶级法权,发展资本主义势力,使资本主义的金钱关系、竞争关系泛滥起来,破坏社会主义的同志式的互助合作的关系,把社会主义相互关系改变为资本主义相互关系,从而瓦解社会主义公有制,达到颠覆无产阶级专政、复辟资本主义的目的。苏修叛徒集团就是利用这种手段,把无产阶级专政的社会主义国家“和平演变”为社会帝国主义国家。刘少奇、林彪一类也想照此办理,只是在毛主席革命路线的指引下,经过无产阶级文化大革命的斗争,才粉碎了他们复辟资本主义的阴谋,进一步巩固了社会生义制度和无产阶级专政。但是,这方面的斗争决不会由于我们取得了胜利而止息的。

毛主席根据国内外社会主义革命和社会主义建设的经验,反复教导我们,在所有制问题基本解决以后,要特别注意调整人们的相互关系。我们必须以阶级斗争为纲,坚持党的基本路线,坚持无产阶级专政下的继续革命,镇压和打击一小撮反对社会主义、进行资本主义复辟活动的阶级敌人;逐步缩小旧社会遗留下来的工农之间、城乡之间、脑力劳动和体力劳动之间差别,限制资产阶级法权;加强社会主义教育,批判修正主义,批判资产阶级,按照社会主义原则来调整人们的相互关系,使社会主义相互关系在斗争中逐步地巩固和发展起来。

为了做好调整工作,更好地巩固和发展社会主义的相互关系,

必须学好无产阶级专政的理论,搞清楚相互关系与所有制、上层建筑的辩证关系,重视调整和变革相互关系的必要性和重要意义。要提高对资产阶级特别是党内的走资本主义道路当权派利用尚存的那部分资产阶级法权,破坏社会主义相互关系,瓦解社会主义经济基础,阴谋复辟资本主义的罪恶活动的警惕性,自觉地把相互关系方面的革命不断深入下去。

必须加强各级领导班子的建设,把领导权牢牢地掌握在真正的马克思主义者和广大工农群众的手里。只有解决了领导权的问题,才能认真执行毛主席的革命路线和政策,带领广大群众正确处理两类不同性质的矛盾,加强对资产阶级的全面专政,抵制各种错误倾向,在人民内部造成一种团结战斗、平等互助、生动活泼的政治局面。

必须在上层建筑领域和经济基础领域中深入开展两个阶级、两条道路、两条路线的斗争。要坚持上层建筑领域中的革命,加强无产阶级专政,深入批判修正主义,批判资产阶级,用马克思主义去逐步战胜残留在人们头脑中的资产阶级思想意识。要在经济基础领域中,逐步缩小三大差别,限制资产阶级法权,加强党对经济工作的领导,严格实行对社会主义生产、分配、交换的计划管理和监督,同一切违背社会主义原则,破坏社会主义协作关系的行为进行坚决的斗争。

必须热情地支持社会主义新生事物。在社会主义革命深入发展过程中所出现的社会主义新生事物,如各级领导班子的老中青三结合、广大干部走五七道路、知识青年上山下乡、赤脚医生不断成长壮大和合作医疗更加巩固、工农兵上大学和教育的改革、工农兵理论队伍的壮大等等,对于缩小三大差别,限制资产阶级法权,从而调整相互关系,具有重大的意义。积极地扶持社会主义新生事物的成长,必将猛烈地冲击资本主义的传统和痕迹,促进共产主义因素的逐步壮大,推动社会主义相互关系的巩固和发展。

\section{社会主义协作和竞赛}

\subsection{发扬共产主义风格,开展社会主义协作}

各个企业、各个部门、各个地区之间的社会主义协作,是劳动人民内部的同志式的社会主义相互关系的一个重要方面。

人类的任何生产活动,都是在分工与协作的基础上进行的。马克思指出,不同劳动的同时存在就是分工,而“许多人在同一生产过程中,或在不同的但互相联系的生产过程中,有计划地一起协同劳动,这种劳动形式叫做协作”。随着社会生产的发展,社会分工越来越细,就越需要加强协作,这是不以人们意志为转移的社会生产发展的客观规律。但是,在不同的社会制度下,协作具有不同的性质和特点。

在资本主义制度下,协作是在生产资料的资本家所有制的基础上,由资本家组织进行的。资本家组织劳动者协同劳动的目的,是为了加强对工人的剥削,榨取更多的利润。因此,这种协作关系必然表现为无产阶级和资产阶级之间的剧烈的阶级对抗。同时,虽然个别企业内的协作是有计划的,但由于资本家所有制把各个企业分割开来,因而根本不可能在全社会范围内,在各个企业、各个部门、各个地区之间有计划地组织协作。它们之间建立起来的某种协作关系,在竞争和无政府状态规律的支配下,也是极不稳定的,经常遭到破坏。

在社会主义制度下,协作是在生产资科的社会主义公有制的基础上,在劳动人民根本利益一致的前提下,由无产阶级专政的国家有计划地组织进行的。生产资料公有制使劳动的性质发生了根本的变化,劳动者“不是替剥削者做工,而是为自己做工,为自己的阶级做工,为社会做工”。因而,社会主义协作是广大劳动群众为了多快好省地建设社会主义,相互交换自己劳动的一种形式。它体现着劳动人民内部同志式的社会主义相互关系。生产资料公有制不仅根本改变了协作的性质,而且为协作开辟了广阔的活动场所。社会主义协作不仅可以在一个企业内有计划地进行,而且能够突破一个企业的局限,在全社会范围内,在各个企业、各个部门、各个地区之间有计划、有组织地进行。

社会主义协作的内容和形式包括许多方面:共同制造产品或零部件的生产协作;结合起来完成一些工程设计和产品加工的技术协作;互相供应和调剂设备、工具、原材料的物资协作;为改变生产条件进行新建、改建而组织的劳动力协作,等等。生产建设的大会战,是在不断调整、完善社会主义相互关系基础上,广大群众在党的统一领导下开展社会主义协作的伟大创举。它突破了协作的一般形式,打破了条条块块的界限,把各方面力量组织起来,集中优势兵力,在短期内解决一些重点工程、尖端技术和急需产品,显示了社会主义协作的巨大威力。如大庆油田的开发、南京长江大桥的建设、成昆铁路的建成、大型远洋轮船的制造、空间和原子技术的发展等,都是开展社会主义协作所取得的丰硕成果。

广泛地发展社会主义协作,可以挖掘现有的一切潜力,发挥专业化的特长,可以产生新的生产力,从而完成一个单位不能完成的任务,加快生产建设速度,取得多快好省的效果。正如马克思指出的:“不仅是通过协作提高了个人生产力,而且是创造了一种生产力,这种生产力本身必然是集体力”。社会主义协作,还可以培养广大群众关心全局,热爱集体,艰苦奋斗,自力更生的革命精神。

社会主义协作是新型的社会主义的互助合作关系,它要求参加协作的单位实行无产阶级政治挂帅,发扬共产主义风格,胸怀全局,把困难留给自己,方便让给别人,保质保量按期地完成协作任务;同时,也要坚持社会主义原则,执行无产阶级的各项经济政策,遵守等价交换的原则。社会主义协作,同资本主义的尔虞我诈、互相倾轧的相互关系是根本对立的,同资产阶级本位主义也是根本对立的。“不顾大局,对别部、别地、别人漠不关心,就是这种本位主义者的特点。”有些单位由于资产阶级本位主义的侵蚀,在协作关系上,有的片面追求利润和产值,不按国家计划和协作合同按质、按量、按期完成生产任务;有的只要求别人为自己服务,自己却不为别人承担协作任务;有的只想当主角不想当配角,只愿干整机不愿干配套;有的借协作之机捞国家一把,卡别人的脖子;有的甚至在协作中混水摸鱼,搞资本主义的邪门歪道。这些都说明在开展社会主义协作的过程中存在着激烈的阶级斗争。我们只有抓住阶级斗争这个纲,坚持无产阶级政治挂帅,批判修正主义,批判资产阶级,加强计划领导,依靠广大群众,发扬共产主义风格,执行社会主义政策,才能为广泛发展社会主义协作扫清障碍。

\subsection{发扬团结互助精神,开展社会主义竞赛}

在社会主义制度下,劳动者不仅彼此协同劳动,而且开展着社会主义竞赛。

社会主义竞赛“是在千百万劳动群众最大积极性的基础上建设社会主义的共产主义方法”,是劳动人民内部同志式的相互关系的生动表现,是广大劳动群众在社会主义生产过程中进行自我教育、共同提高的重要形式。

社会主义竞赛是在社会主义生产关系的基础上产生并发展起来的社会主义新生事物,它同资本主义的竞争有着根本的区别。“竟争的原则是:一些人的失败和死亡,另一些人的胜利和统治。”“社会主义竞赛的原则是:先进着给予落后者以同志的帮助,从而达到普遍的提高。”

在资本主义制度下,资本家之间在冷酷的金钱交易中,为了摄取和瓜分最大限度的利润彼此进行着你死我活的争夺。这种竞争关系,必然会渗透到社会关系的各个方面,成为支配人们相互关系的基本原则。

社会主义制度的建立,消灭了人剥削人的现象,劳动者成了社会的主人,共同为革命而劳动,劳动成了光荣豪迈的事业。他们在协同劳动的过程中,为了共同的革命目标,开展互相促进、共同提高的社会主义竞赛。同时,在社会主义生产过程中,由于人们的觉幅程度不同、劳动态度不同、工作能力和技术水平也不同,还必然存在着先进与后进的差别,正确与错误的矛盾。正确处理这些经常产生并大量存在着的人民内部矛盾,既是不断完善劳动群众之间相互关系的需要,又是促进社会主义生产发展的条件。因而,也有必要开展社会主义竞赛。那种认为社会主义劳动竞赛可有可无的观点,是非常错误的。

列宁曾经指出:“现在当社会主义政府执政时,我们的任务就是要组织竞赛。”毛主席也教导我们:“必须实行劳动竞赛”。

我国解放后,在党的领导下,轰轰烈烈地开展了社会主义竞赛。无数事实,充分证明社会主义竞赛具有强大的生命力和巨大的优越性。通过竞赛,树立典型,表扬和学习先进人物的先进思想和先进事迹,可以更好地对广大群众进行生动具体的社会主义教育,发扬共产主义劳动态度,克服旧的思想影响,扶持新生事物的成长。通过竞赛,不断地暴露和解决群众中存在的先进与后进、正确与错误的矛盾,可以充分调动和依靠广大群众的力量,在比先进、学先进、赶先进、帮后进,你迫我赶的群众运动中,推动先进的更先进,后进的向先进转化,进一步发展劳动人民内部同志式的相互关系。通过竞赛,及时总结和推广先进经验,可以使个别人、个别单位在生产技术和管理上所达到的先进水平,迅速地为大多数人或单位所掌握,变成全社会的水平,有助于不断提高劳动生产率,加快社会主义建设的步伐。

可见,社会主义竞赛不仅是群众性的生产运动,而且是群众性的思想教育运动。它的推动力量只能来自于广大群众政治觉悟的不断提高和为实现共产主义理想的自觉斗争,决不是来自于“物质刺激”的引诱。所以,只要我们加强社会主义教育,坚持“抓革命,促生产”的方针,彻底批判刘少奇、林彪一类所推行的修正主义路线,批判资产阶级法权思想,注意调整社会主义的相互关系,并做好竞赛的具体组织工作,就能够进一步激发广大群众当家作主的责任感和为革命而忘我劳动的高尚精神,使社会主义竞赛深入持久地开展下去,发挥它在社会主义革命和社会主义建设中的重要作用。

\section{社会主义企业管理}

\subsection{正确处理企业内部相互关系,是社会主义企业管理的重要任务}

劳动人民内部的相互关系,大量地发生在企业内部。正确地处理企业内部相互关系,是社会主义企业管理的一项重要任务。

现代化的大生产是极为复杂的,必须进行有效的组织和领导,才能使它顺利地进行。企业管理的这种必要性,在任何社会形态下都是相同的。但是,企业管理从来都是由占有生产资科的阶级实行的。它是这个阶级的意志的体现,是为巩固和发展一定的生产关系服务的。

在资本主义制度下,企业管理是资本的职能,是资本家及其代理人的特权。它是为最大限度地榨取剩余价值和维护资本主义剥削关系服务的,是资产阶级对无产阶级实行专政和统治的手段。“资本家所关心的是怎样为掠夺而管理,怎样借管理来掠夺。”

在社会主义制度下,企业管理则是由劳动群众及其代表来实行的,是劳动群众的一项根本权利。它是为加速社会主义建设,发展社会主义生产关系和巩固无产阶级专政服务的。

社会主义企业中劳动人民之间的相互关系,主要有两个方面:领导和群众之间的关系,管理人员、技术人员和直接生产者之间的关系。他们之间虽然分工不同,但都是企业的主人,共同管理企业,为了一个共同的革命目标而工作。因而,他们在根本利益一致的基础上,建立了同志式的社会主义关系。但是,他们还受着旧的社会分工的束缚,领导人员、管理人员和技术人员一般属于脑力劳动者,直接生产者则是体力劳动者,他们之间存在着等级差别和生活待遇高低的差别。同时,由于资产阶级思想意识的侵蚀,一些领导人员、管理人员和技术人员就会违反社会主义原则,不以平等的态度对待群众,把同志式的社会主义关系变成统治和服从的关系;有的群众也会离开社会主义原则,来处理他们之间的相互关系。因而,在他们之间也还存在着矛盾。这些矛盾,一般属于人民内部矛盾。可是,如果不及时地正确处理,任其发展下去,企业内部劳动人民之间的社会主义相互关系就会变为资本主义相互关系,社会主义公有制就会瓦解,社会主义企业就会蜕化变质。

因此,社会主义企业管理虽然包括多方面的内容,诸如政治管理、生产管理、计划管理、技术管理、物资管理、财务管理和生活管理等,但其根本任务则是紧紫抓住阶级斗争这个纲,深入开展阶级斗争,努力使企业的领导权牢牢掌握在真正的马克思主义者和广大工农群众手里,巩固和发展人们之间的社会主义相互关系,促进生产高速度地发展,使企业成为巩固无产阶级专政的坚强阵地。那种把企业仅仅看成是生产单位,把企业管理的任务仅仅理解为发展生产的观点是极为错误的。

毛主席关于“管理也是社教”的指示,从社会主义时期两个阶级、两条道路、两条路线斗争的高度,概括了社会主义企业管理的实质,为搞好社会主义企业管理指明了方向。

\subsection{全面贯彻“鞍钢宪法”,不断完善企业内部相互关系}

在社会主义企业管理的问题上,一直存在着两条路线的斗争。毛主席历来教导我们,从事党的各项工作,都必须坚持政治挂帅,加强党的领导,实行群众路线。刘少奇一伙出于反革命目的,疯狂对抗毛主席的指示,全面推行修正主义的办企业路线。他们否定无产阶级政治挂帅,反对党的领导和群众运动,实行“生产第一”、“利润挂帅”和“物质刺激”,大搞“专家治厂”,扩大资产阶级法权,扩大等级差别,对广大群众实行资产阶级专政,破坏社会主义相互关系,阴谋把社会主义企业引上复辟资本主义的道路。

毛主席在同修正主义的斗争中,总结了社会主义革命和社会主义建设的历史经验,于一九六○年三月二十二日就中共鞍山市委给中央的一个报告,亲自批示了“鞍钢宪法”,提出了管理社会主义企业的基本原则。这就是:坚持政治挂帅,加强党的领导,大搞群众运动,实行“两参一改三结合”(即坚持干部参加劳动,群众参加管理,改革不合理的规章制度,实行工人、干部和技术人员的三结合),大搞技术革新和技术革命。“鞍钢宪法”是无产阶级办企业路线的集中体现,是社会主义企业管理的伟大纲领,是正确处理企业内部劳动人民之间相互关系的准则。

坚持政治挂帅,就是要坚持企业的社会主义方向。马克思主义历来认为,经济是基础,政治则是经济的集中表现。政治与经济、技术、业务的关系是对立统一的关系,但政治是统帅,是领先的,它决定着经济、技术、业务的发展方向。因此,在企业的一切工作中,必须以阶级斗争为纲,坚持党的基本路线,坚持无产阶级政治挂帅,深入开展阶级斗争和路线斗争,加强政治思想工作,批判修正主义,批判资产阶级,限制资产阶级法权。我们要坚决反对那种埋头业务,忽视政治,用物质刺激代替政治思想工作的错误倾向,批判反动的唯生产力论。只有这样,才能保证企业的社会主义方向,正确处理企业内部劳动人民之间的相互关系,调动广大群众的社会主义积极性,促进生产的迅速发展,巩固和发展社会主义公有制。

加强党的领导,是把企业的领导权牢牢地掌握在真正的马克思主义者和广大工农群众手里,使企业沿着社会主义方向发展的根本保证。要加强党对企业的一元化领导,必须遵照“要搞马克思主义,不要搞修正主义;要团结,不要分裂;要光明正大,不要搞阴谋诡计”的三项基本原则,加强党的思想建设和组织建设,坚持和执行党的正确路线和政策,批判和抵制错误的路线和政策,在马克思主义、列宁主义、毛泽东思想的基础上,统一认识,统一政策,统一计划,统一指挥,统一行动。我们必须深入批判刘少奇一伙顽固推行的“一长制”,牢固地树立党是领导一切的思想,同一切危害和削弱党的领导的错误倾向作斗争。

大搞群众运动,就是要依靠广大工人群众和贫下中农办企业。马克思主义认为,人民群众是历史的创造者,只有在企业的一切工作中,坚持群众路线,大搞群众运动,把各方面人员的积极性充分调动起来,拧成一股劲,才能使生产多快好省地向前发展。要大搞群众运动,必须批判把不同的车间、工种人为地分为优劣贵贱的资产阶级法权思想,并打破束缚群众手脚的过细过死的劳动分工;企业领导必须端正对群众运动的态度,坚决反对那种只依靠少数人冷冷清清办企业的错误倾向,肃清“专家治厂”的流毒;必须坚持从群众中来到群众中去的马克思主义的领导原则,真正把党的路线、政策、任务和计划变成群众的自觉行动;必须坚持集中领导和群众运动相结合的原则,切实加强对群众运动的领导。

实行“两参一改三结合”,是正确处理企业内部劳动人民之间相互关系的社会主义准则,是限制资产阶级法权的一项重要措施。

干部参加集体生产劳动,以普通劳动者的姿态出现在群众之中,可以有效地克服官僚主义和防止修正主义。同时,也有利于干部改造思想,转变作风,密切干群关系,实现正确的领导和指挥。这就从根本上避免了同群众处于尖锐的对立状态,为正确处理领导和群众、技术管理人员和直接生产者之间的相互关系创造了有利条件。群众参加管理,打破了那种只有资产阶级“专家”、“权威”才能管理企业的偏见,极大地激发了广大群众当家作主的责任感。工农群众参加管理,关心企业的各项活动,不仅可以培养他们从容管理工作的能力,而且可以充分发挥他们对干部的监督作用,帮助干部贯彻执行党的路线、方针和政策,坚持企业的社会主义方向。因而,坚持干部参加劳动、群众参加管理,是社会主义制度下一件带根本性的大事。

在干部参加劳动、群众参加管理的基础上,实行工农群众、干部和技术管理人员的三结合,这是企业管理上的新创造,具有强大的生命力。三结合使领导和群众、工农群众的实践经验和技术人员的理论知识正确地结合起来,它不仅可以调动各方面的积极性,促进生产、技术的迅速发展,而且有利于加强党对知识分子的团结、教育和改造,实现知识分子劳动化和工农群众知识化,逐步缩小脑力劳动和体力劳动的本质差别,促进社会主义相互关系的发展。

改革不合理的规章制度,是正确处理社会主义相互关系的又一个重要方面。管理社会主义企业,需要建立一套科学的规章制度。社会主义企业管理的规章制度,是为了更好地调动群众的社会主义积极性,促进生产的发展,而不是束缚群众的手脚,同资本主义企业对群众实行“管、卡、压、罚”的规章制度是根本对立的。因而,它必须是无产阶级政治挂帅的,既要符合生产需要,更要体现工农群众是企业的主人,正确处理企业中人与人的相互关系,有利于调动工农群众的社会主义积极性。总之,制度要有利于群众。我们必须根据革命和生产发展的需要,由领导和群众相结合,改革不合理的规章制度,建立和健全必要的合理的规章制度,以不断完善社会主义相互关系和促进生产力的发展。有了合理的制度,还必须在正确路线的指导下贯彻执行。

大搞技术革新和技术革命,就是要大力开展群众性的技术改革运动,实现生产技术的现代化。它是无产阶级和广大劳动群众高速度地发展社会主义生产,尽快地实现社会主义的工业、农业、科学技术和国防现代化的重要途径。

毛主席在“鞍钢宪法”的批示中所确立的管理社会主义企业的基本原则,正确处理了政治与经济、革命与生产、上层建筑与经济基础、生产关系与生产力之间的关系。以阶级斗争为纲,坚持党的基本路线,全面落实“鞍钢宪法”,就能进一步完善社会主义相互关系,促进社会主义建设事业的迅速发展。

\chapter{社会主义的农业和工业}

农业和工业是国民经济中两个最主要的物质生产部门,农业是国民经济的基础,工业是国民经济的主导。发展国民经济以农业为基础、工业为主导,把发展工业和发展农业结合起来,将有利于正确处理工人和农民两个劳动阶级的关系,巩固和发展工农联盟,促进国民经济有计划按比例、高速度地发展,加速实现农业、工业的现代化。

\section{发展国民经济要以农业为基础、工业为主导}

\subsection{农业是国民经济的基础}

农业是国民经济的基础,这是由农业生产的特点所决定的。首先,农业是人类生存和一切生产活动的先决条件。“食物的生产是直接生产者的生存和一切生产的首要的条件”。人们要进行生产或从事其他社会活动,首先就要解决吃饭、穿衣问题,而农业生产正是为人们提供衣食的来源。其次,农业也是工业、交通运输业、商业和科学文化事业能够产生和发展的基础。因为只有农业劳动生产率一定程度的提高,使人们无需用全部劳动时间来取得维持生命的必要生活资料,才能使一部分人从事农业以外的其他活动。正如马克思指出的:“超过劳动者个人需要的农业劳动生产率,是一切社会的基础”,“社会用来生产小麦和牲畜等等所需要的时间愈少,用来进行其他的生产——物质和精神的生产的时间就愈多。”

马克思考察了历史上各种生产方式所得出的这个一般结论,同样也适用于社会主义社会。在社会主义条件下,农业的基础作用,具体表现在以下几个方面:

农业是粮食等人类赖以生存的生活资料的来源。粮食,对于国计民生具有特殊重要的意义。列宁指出:“经济的真正基础是粮食储备。”无论是建设,还是打仗,总得先有饭吃。手里有粮,心中不慌,有了粮食,就什么都好办了。因此,农业能够提供多少商品粮,不仅直接关系到人民的生活,而且直接影响到工业和其他事业的发展。所以说,农业是国民经济的基础,而粮食则是基础的基础。

农业是工业原料的重要基地。工业所需要的原料,除由工业本身提供一部分外,还有相当大的一部分是由农业提供的。我国目前农业所提供的工业原料,约占全部工业原料的百分之四十,占轻工业原料的百分之七十左右。随着现代工业的发展,工业原料中由农业提供的那部分的比重,虽然会逐步降低,但农业作为工业原料的重要基地这一点是不会改变的。

农业是工业品的重要市场。市场是实现社会再生产的重要条件。社会主义工业主要依靠国内市场。我国亿万农业人口的广大农村,是工业品最大的销售市场。“农民——这是中国工业市场的主体。只有他们能够供给最丰富的粮食和原料,并吸收最大量的工业品。”农业生产越发展,农民购买力就越高,也就越能为工业提供广阔的市场。目前,我国轻工业产品约有三分之二销售于农村市场。农业不仅是轻工业的重要市场,而且也是重工业的重要市场。随着农业的技术改革逐步发展、农业的日益现代化,对农业机械和其他农业生产资料的需要将日益增多,农业作为重工业的重要市场的作用也就日益明显起来。

农业是工业和国民经济其他部门劳动力的重要来源。工业和国民经济其他部门的发展,除了依靠提高本部门的劳动生产率以外,还需要增加劳动力。这些增加的劳动力,一部分来自城市人口,而从长远来看,则主要来自农村,依靠农业部门来提供。毛主席指出:“农民——这是中国工人的前身。将来还要有几千万农民进入城市,进入工厂。如果中国需要建设强大的民族工业,建设很多的近代的大城市,就要有一个变农村人口为城市人口的长过程。”但是,农业能够为工业和国民经济其他部门提供多少劳动力,取决于农业劳动生产率的水平。当前,在我国农业劳动生产率水平较低的情况下,还不可能从农业中解放出大量的劳动力。随着农业生产的发展,农业劳动生产率的不断提高,农业就有可能逐步地抽调出大量的劳动力转移到工业和其他部门,促进整个国民经济更快地发展。

农业是国家建设资金积累的重要来源。资金积累是扩大再生产的源泉。社会主义国家经济建设所需要的资金,只能依靠本国内部的积累。毛主席指出:“为了完成国家工业化和农业技术改造所需要的大量资金,其中有一个相当大的部分是要从农业方面积累起来的。这除了直接的农业税以外,就是发展为农民所需要的大量生活资料的轻工业的生产,拿这些东西去同农民的商品粮食和轻工业原料相交换,既满足了农民和国家两方面的物资需要,又为国家积累了资金。”当前,在我国的财政收入中,由农业直接和间接提供的资金积累约占一半左右。这就不难看出,农业在为国家积累建设资金方面所起的重要作用。

农业是国民经济的基础,是一个不以人们的意志为转移的客观规律,它制约着整个国民经济的发展。这已为我国社会主义建设的实践所证明,哪一年农业生产发展较快,第二年国民经济的发展就较快;反之,哪一年农业生产发展较慢,第二年国民经济的发展也就较慢。例如,一九五二年农业丰收,农业总产值比上年增长百分之十五点三,因而一九五三年工业总产值就增长了百分之三十点二,国家财政收入增长百分之二十四,基本建设投资增长百分之八十四。一九五四年农业歉收,农业总产值只增长百分之三点三,一九五五年工业总产值就只增长了百分之五点六,国家财政收入只增长百分之三点七,基本建设投资只增长百分之三。一九五八年农业大丰收,整个国民经济就迅速发展。一九五九年至一九六一年农业连续三年遭受严重的自然灾害,整个国民经济各部门的发展都受到很大的影响。最近十多年来,农业连续获得丰收,因而就推动工业生产持续上升,整个国民经济蓬勃发展。可见,农业的发展水平,在很大程度上决定着工业和整个国民经济的发展规模和速度。

\subsection{工业是国民经济的主导}

农业是国民经济的基础,工业则是国民经济的主导。工业作为国民经济的主导,是由工业生产本身的特点所决定的。我们知道,工业不仅生产消费资料,而且生产生产资料,制造生产工具。而农业和其他生产部门的存在和发展,是决不能离开生产工具的。正如马克思指出:“狩猎、捕渔、耕种,没有相应的工具是不行的。”特别是当现代工业出现以后,工业对农业和国民经济其他部门的发展,就起着更加巨大的作用。一个国家的工业发展水平,直接决定着这个国家的生产技术水平和国民经济的发展水平。

在社会主义条件下,工业在国民经济中的主导作用集中地表现在,现代化的大工业是实现扩大再生产和对整个国民经济进行技术改造的物质基础。现代化大工业的发展,生产资料生产较快地增长,可以为扩大再生产提供所需要的追加生产资料,保证社会主义扩大再生产的顺利进行;可以以现代化的技术装备来武装农业和国民经济其他部门,促进整个国民经济的技术改造和发展;可以以现代化的武器来装备人民军队,实现国防的现代化。

工业包括轻工业和重工业。从上面的分析中,我们可以看到,工业的主导作用是由生产生产资料特别是制造生产工具的重工业来体现的。但是,这决不是说轻工业就不重要了。轻工业在国民经济中,也具有重要的地位和作用。我们知道,轻工业基本上是生产消费资料的工业。它生产出大量的日用工业品,满足人们吃、穿、用的需要。同时,轻工业和农业有着密切的联系,农业生产的粮食、经济作物和其他农副产品,一般地都需要经过轻工业的加工,才能被人们用于消费。因而,轻工业的发展,增加对农业原料的需要,扩大农副产品的利用范围,也就可以促进农业生产的发展,并使农业的基础作用得到更好地发挥。此外,轻工业一般具有投资少、建设较易、周转较快的特点。注意发展轻工业,就能够为国家提供较多的资金积累,用于农业、重工业和整个国民经济的发展。

由此可见,工业是国民经济的主导,同农业是国民经济的基础一样,是一个不以人们的意志为转移的客观规律。

“以农业为基础、工业为主导”,把国民经济搞上去

农业是基础,工业是主导,是个客观经济规律。它要求农业和工业两个国民经济中最重要的物质生产部门正确结合,互相促进。但是,在资本主义社会,由于生产资料资本主义私有制的存在,工业剥削农业、城市剥削乡村,工业和农业根本不可能做到正确结合。只有在社会主义社会,由于生产资料社会主义公有制的建立,消除了工业和农业之间、城市和乡村之间的对立,国民经济有计划按比例地发展,从而有可能通过国民经济计划的调节,使工业和农业正确地结合起来。

社会主义制度虽然为农业和工业的正确结合提供了可能,但要处理好它们之间的辩证关系,必须有一条正确的路线和方针。在怎样处理工业和农业关系的问题上,一直存在着两种根本不同的指导思想和两种不同的发展国民经济的方针。一种指导思想是,只重视重工业对农业和整个国民经济的促进作用。其方针就是少发展农业和轻工业,片面发展重工业。另一种指导思想是,既重视工业持别是重工业的主导作用,又重视农业的基础作用。其方针,就是毛主席制定的“以农业为基础、工业为主导”的发展国民经济总方针。这一伟大方针,不仅正确地反映了客观经济规律的要求,而且也丰富了马克思列宁主义关于社会主义农业和工业在国民经济中的地位、作用及其相互关系的原理。正确地贯彻执行这一方针,对于加快社会主义建设和巩固无产阶级专政具有极其重要的意义。

怎样贯彻执行“以农业为基础、工业为主导”的发展国民经济总方针呢?

首先,必须以农业为出发点,把农业放在首位,按照农、轻、重的顺序安排国民经济计划。这就是说,在安排国民经济计划时,必须考虑满足农业发展对生产资料和日用消费品的需要,根据农业能够提供多少商品粮食、原料、劳动力和资金等的可能性,来安排轻工业、重工业和国民经济其他部门的发展。当然,这并不是说只有完全满足农业需要之后才去安排其他方面,也不是说分配给农业的物资、资金必须占绝大部分,而是根据实际情况,从加速发展农业着手来开展社会主义建设事业。

当前,我国农业生产力水平还比较低,不能完全适应工业和国民经济发展的要求。因此,必须保证农业生产第一线有足够的劳动力,逐年增加对农业的投资,尽快地改变农业生产的条件,使农业生产有一个较大的发展。这样,才能使工业和整个国民经济的发展有可靠的基础。

其次,必须围绕农业的需要发展工业,坚决把工业部门的工作转移到以农业为基础的轨道上来。工业支援农业,积极为农业生产服务,这就要求社会主义工业化必须主要以农业现代化为目标,根据支援农业的需要,建立为农业服务的工业结构;要求重工业部门为加速发展农业,提供更多更好的、适合于各地农业特点和农业迫切需要的农业机械、化肥、农药和水利电力设备等生产资料;要求在工业内部,正确处理好主机和配件、制造和修理等关系,保证农机配件的供应,做好农机修理工作,以充分发挥农业机械设备的作用。同时,也要求轻工业部门积极为农村提供价廉物美的日用工业品,尽量地节约农业原料,开辟使用工业原料的途径,以减轻农业的负担。实际上,工业支援农业的过程,就是工业本身不断发展的过程。工业应该从促进农业生产的发展中,去取得自己发展的条件。

不仅工业要支援农业,各行各业都要支援农业,为大办农业作出积极的贡献。商业部门要千方百计地为农业生产和社员生活的各种需要服务。交通运输部门要保证支农物资和农产品的及时调运。文教卫生等部门也要为促进农业的发展,多做工作。只有农业搞好了,整个国民经济才能更快地发展。那种认为“支农吃亏”、“与已无关”的思想是错误的,必须彻底纠正。

贯彻“以农业为基础、工业为主导”的发展国民经济总方针,处理好工业和农业之间的辩证关系,实质上是一个正确处理工人和农民两大劳动阶级之间的关系,巩固和发展工农联盟的问题。毛主席指出:“我国有五亿多农业人口,农民的情况如何,对于我国经济的发展和政权的巩固,关系极大。”在社会主义制度下,工人和农民的根本利益是一致的。但是,由于还存在着社会主义公有制的两种形式,还存在着旧社会遗留下来的三大差别,工人和农民之间还存在一定的矛盾。实行“以农业为基础、工业为主导”,就是把发展工业同发展农业结合起来,把完成工业现代化和实现农业现代化结合起来,把加强工人阶级的领导作用同充分发挥广大农民的同盟军作用结合起来,从而使工农联盟在社会主义的现代化大生产的基础上得到进一步巩固和加强。

我国社会主义革命和社会主义建设的实践证明,凡是贯彻执行了毛主席制定的“以农业为基础、工业为主导”的发展国民经济总方针,农业和工业就得到迅速发展,工农联盟就得到进一步巩固。从全国来看是这样,从一个地区来看也是这样。

我们必须从贯彻执行党的基本路线,巩固无产阶级专政,防止资本主义复辟,建设社会主义和共产主义的高度,来认识正确处理工农关系的重要意义。必须彻底批判刘少奇一类所推行的“重工轻农”、“以工挤农”的反革命修正主义路线,更加自觉地贯彻执行“以农业为基础、工业为主导”的发展国民经济总方针。

\section{社会主义农业的发展}

\subsection{农业的根本出路在于机械化}

要充分发挥农业在国民经济中的基础作用,促进社会主义工业化的发展,把工农联盟建立在巩固的社会主义现代化大生产的基础上,就要求实现农业机械化。

我们党在农业问题上的根本路线是:在完成反封建的土地改革以后,第一步实现农业集体化,第二步在农业集体化的基础上实现农业机械化和电气化。这是党在农村的两条道路的斗争中,坚持社会主义、战胜资本主义的根本路线。

解放前,我国虽然是一个农业国,但由于受旧的生产关系的严重阻碍,农业生产技术长期停滞不前,处于十分落后的状态。解放后,在毛主席的革命路线指引下,党在农村中实行了土地改革、农业集体化等一系列社会改革,使农业生产有了很大的发展。但是,我国目前的农业生产技术水平还是不高的,基本上还是畜力耕种和手工劳动。因而,农业劳动生产率和农产品的商品率都比较低,粮食的单位面积产量不高,经济作物的比重较小,抵御自然灾害的能力不强。农业生产的这种状况,同工业和国民经济其他部门发展的要求是不相适应的。这是当前我国国民经济发展中存在的一个很大矛盾。

为了解决这个矛盾,毛主席及时地发出了“农业的根本出路在于机械化”的指示,向全党和全国人民提出了在巩固集体经济的基础上,加速农业技术改造,逐步实现农业机械化的伟大历史任务。

实现农业机械化,就是要求把社会主义农业建立在现代化的物质技术基础上。也就是说,农、林,牧、副、渔等一切能够使用机器的工作,统统使用机器,并大量地使用先进的农业生产资料。实现农业机械化,必将在思想、政治、经济、文化各个领域引起深刻的变化,是我国农村实现农业集体化之后的又一场伟大革命。

实现农业机械化,能够促进农业生产力的发展,更好地发挥农业在国民经济中的基础作用。因为,农业机械化可以大大地提高抗旱、排涝、灭虫的效率,增强人们同自然灾害作斗大地提高抗旱、排涝、灭虫的效率,增强人们同自然灾害作斗争的能力,改变“靠天吃饭”的被动局面;可以成倍地提高农业劳动生产率,把节省下来的劳动力,向农业生产的深度和广度进军,提高单位面积的产量,促进农、林、牧、副、渔的全面发展。这就可以为工业和国民经济其他部门的发展提供更多的粮食、原料、资金和劳动力,加快社会主义的工业、国防和科学技术现代化前进的步伐。

实现农业机械化,有利于农村人民公社集体所有制的巩固和发展。因为,农业机械化促进了农业生产力的发展,也就壮大了集体经济的力量,增加了社员的收入,充分地显示出人民公社“一大二公”的优越性。同时,随着农业机械化的发展,不仅生产队集体经济得到加强,而且公社、大队两级集体经济也进一步壮大起来。

实现农业机械化,有利于进一步巩固工农联盟和无产阶级专政。只有农业集体化,而没有农业机械化,工农联盟是不能完全巩固的。通过农业机械化把农业改造成现代化大生产的社会主义大农业,使工人和农民在生产上进一步互相支持、互相援助,就可以把工农联盟建立在更加巩固的基础上。农业机械化的实现,也将以生动的事实教育农民,广大贫下中农将会认识到社会主义现代化农业绝对优胜于小农经济和资本主义,应自觉地改变小生产的习惯,同私有制观念、旧的传统观念实行决裂。并且,由于农业生产广泛使用机器,生产技术水平大大提高,社、队普遍兴办工业,农民亦工亦农,从而使工农之间、城乡之间、脑力劳动和体力劳动之间的差别逐步缩小。

由此可见,实现农业机械化,是充分发挥农业在国民经济中的基础作用的客观要求,是促进农村人民公社集体所有制巩固和发展的必要条件,是巩固工农联盟和无产阶级专政的重大措施,具有十分重大的意义。

怎样实现农业机械化呢?为了保证农业机械化的胜利实现,必须以阶级斗争为纲,坚持无产阶级政治挂帅,用革命化统帅机械化;坚持从实际出发,因地制宜,实行机械化和半机械化同时并举;坚持自力更生、勤俭建国的革命精神,充分发挥中央和地方的两个积极性,认真贯彻农业机械制造以地方为主、农业机械产品以中小型为主、农业机械购买以集体为主的方针;坚持群众路线,建设一支又红又专的农业机械化队伍。

实现农业机械化,根本在路线,关键在领导。在要不要实现和怎样实现农业机械化的问题上,存在着两条根本对立的路线。刘少奇一类极力散布“人多地少不用机械化”、“精耕细作不能机械化”、人民公社“买不起,用不了,管不好”农业机械等反动谬论,大肆推销“国家出钱,农民种田”的修正主义黑货,妄图把农业机械化引上邪路,从根本上破坏这场伟大革命。经过无产阶级文化大革命,批判了刘少奇一类的修正主义路线,毛主席关于农业机械化的理论、路线、方针和政策深入人心,我国农业机械化事业得到了迅速发展。一九七○年和一九六五年相比,农机产品增加三百多种,耕作机械、排灌机械、农副产品加工机械、运输机械的产量有了成倍增长。一九七○年全国拖拉机拥有量和电力排灌设备都比一九六五年增长一倍左右。一九七○年的化肥产量比一九五七年增长十四倍左右。全国百分之九十以上的县建立了农机修造厂,一支群众性的农业科学技术队伍不断发展壮大。近四年来,为农业提供的排灌机械、化肥、拖拉机等,超过了过去的十五年,涌现了一批省、地、县农业机械化的先进典型。只要我们坚决执行毛主席的革命路线,加强党的领导,大搞群众运动,下定决心,鼓足干劲,一定能够实现农业机械化这个伟大的历史任务。

\subsection{以粮为纲,全面发展}

要发展农业生产,充分发挥农业在国民经济中的基础作用,还必须实行“以粮为纲,全面发展”的方针,正确处理农业内部的两个布局。

广义的农业包括农、林、牧、副、渔各业。狭义的农业(指种植业)中又包括粮食作物和经济作物,即粮、棉、油、麻、丝、茶、糖、菜、烟、果、药、杂等十二类。农业生产中存在着多种经营,这是农业再生产的特点所决定的。由于农业生产受着作物生长的自然规律的制约,生产期间和劳动期间不相一致,这样就使农业必然成为多种经营的生产部门。在社会主义社会,由于“社会主义不仅从旧社会解放了劳动者和生产资料,也解放了旧社会所无法利用的广大的自然界”,从而为多种经营开辟了广阔的道路。

正确处理农业内部的关系,必须认真贯彻“以粮为纲,全面发展”的方针。就整个农业来说,粮食生产是农业生产的中心环节。粮食是工业和整个国民经济发展的前提和基础,也是农业多种经营的前提和基础。没有粮食,就不可能进行其他农作物的生产,也不可能开展林、牧、副、渔业的活动。但是,粮食生产又不能孤立地发展,它要受林、牧、副、渔各业的影响和制约。就农业与林、牧业之间的关系来看,发展林业,能够保持水土,调节气候,是农业生产的重要保障;发展牧业,能够提供役畜和积累资金,特别是发展养猪业,能够提供大量的有机肥料,促进农业生产的发展。所以,毛主席指出:“农、林、牧三者互相依赖,缺一不可,要把三者放在同等地位。”就农业和副、渔业之间的关系来看,也是如此。积极地发展副业和渔业,不仅能够为工业提供更多的原料,为城乡人民提供更多的消费品,而且也能够为集体经济增加积累,促进农业生产的发展。至于在种植业内部,发展棉、油、麻、丝、茶、糖、莱、烟、果、药、杂等项生产,既能满足国民经济的多种需要,又能壮大集体经济的力量,促进粮食生产的发展。因此,在安排农业生产时,应当贯彻“以粮为纲,全面发展”的方针,对农、林、牧、副、渔各业和粮、棉、油、麻、丝、茶、糖、菜、烟、果、药、杂等作物的生产,进行统筹兼顾,适当安排,搞好这两个布局,使它们互相促进,全面发展。

实行“以粮为纲,全面发展”的方针,就能调整农业生产的内部结构,实现农业机械的综合利用;就能更好地发挥劳动力的潜力,充分利用丰富的自然资源;就能促进农业生产的发展,为工业和国民经济其他部门提供更多的粮食、副食品、工业原料以及出口物资,就能不断增加集体经济积累和社员收入,巩固和发展集体经济,加速实现农业机械化。

刘少奇、林彪一类出于破坏社会主义集体经济、阴谋复辟资本主义的罪恶目的,竭力干扰和破坏党的路线、方针和政策的贯彻落实。他们时而大搞“重钱轻粮”、“重副轻农”,反对“以粮为纲”;时而又诬蔑发展多种经营是“资本主义倾向”,破坏“全面发展”。为了使我国农业沿着社会主义道路迅速发展,正确贯彻执行“以粮为纲,全面发展”的方针,必须以阶级斗争为纲,进一步严格地分清路线是非和政策界限,处理好农业内部的两个布局。

为了保证农业生产的迅速发展,向生产的深度和广度进军,还必须认真贯彻农业“八字宪法”。毛主席制定的“土、肥、水、种、密、保、工、管”的“八字宪法”,是我国劳动人民创造的丰富经验的科学总结。在实现农业机械化的过程中,因地制宜地认真贯彻农业“八字宪法”,大搞以改土、治水为中心的农田基本建设,实行科学种田,有利于保证农业生产的稳产高产,促进农、林、牧、副、渔的全面发展。

\subsection{农业学大寨}

“农业学大寨”,是毛主席发出的伟大号召。

山西省昔阳县大寨公社大寨大队,是我国农业战线上的一面红旗。二十多年来,大寨大队的贫下中农,在党的领导下,以阶级斗争为纲,坚决贯彻执行毛主席的无产阶级革命路线,坚持无产阶级专政下的继续革命,坚定地走社会主义道路。他们抵制了修正主义路线的破坏和干扰,粉碎了资本主义势力的猖狂进攻,打击了阶级敌人的破坏活动。他们坚持对小生产者的教育改造,比较充分地调动了广大贫下中农和干部的社会主义积极性,依靠集体经济的力量,自力更生,艰苦奋斗,以敢教日月换新天的大无畏精神,改造旧山河,不断地向生产的深度和广度进军,实现了农业生产的稳产高产。因此,使农村的社会主义阵地日益巩固,对国家的贡献越来越多,集体经济的家业越来越大,社员收入也有了合理的增加,把一个穷山恶水、贫穷落后的旧山庄,建设成繁荣富庶的社会主义新农村。大寨的发展道路,生动地体现了我国社会主义农业的发展道路。

学习大寨,就要象大寨那样,坚持无产阶级政治挂帅、毛泽东思想领先的原则。大寨的贫下中农深刻地认识到,社会主义同资本主义之间是一场你死我活的斗争,“堵不住资本主义的路,就迈不开社会主义的步”。因此,他们牢记党的基本路线,一天也没有放松过两个阶级、两条道路的斗争,一天也没有停顿过在政治、思想、经济领域里的社会主义革命,一天也没有放松过无产阶级对资产阶级的全面专政,深入地批判修正主义,批判资本主义倾向,批判资产阶级法权思想,限制资产阶级法权。就是这样,不管资本主义、修正主义在什么地方露头,披着什么外衣,都能及时发现,有力打击,把巩固无产阶级专政的任务落实到基层。

学习大寨,就要象大寨那样,发扬自力更生、艰苦奋斗的革命精神。大寨的贫下中农大批资本主义,大干社会主义,大批促大干。他们用革命统帅生产,坚持“鼓足干劲,力争上游,多快好省地建设社会主义”的总路线,依靠集体的力量,坚持社会主义方向,发挥群众的干劲、智慧和创造性,不向国家伸手,自力更生战胜困难,使农业生产年年发展,农业技术年年进步,从根本上改变了大寨的面貌。解放前,大寨的粮食平均亩产不到一百斤,总产量最高年产是八万斤。到一九七四年,总产量就达到了七十七万斤,接近解放前总产量的十倍;总收入达到十九万四千八百多元,增长了十倍还多。

学习大寨,就要象大寨那样,树立爱国家、爱集体的共产主义风格。大寨党支部坚持用马列主义、毛泽东思想教育农民,不断进行反修防修的教育,破除剥削阶级的陈腐的传统观念,使群众逐步从小农经济习惯势力的束缚下解放出来,坚定走社会主义道路的决心。他们在处理国家与集体的关系中,坚持了兼顾国家利益、集体利益和个人利益的原则;在处理集体积累与社员分配的关系中,坚持了不断壮大社会主义集体经济的原则;在处理集体与集体之间的关系中,不断地提倡爱自己的集体、也爱兄弟队的集体的思想。他们不仅教育干部和社员热爱自己的集体,而且也教育干部和社员时时不忘全中国、全人类。

总之,学习大寨,最根本的就要象大寨那样,用马列主义、毛泽东思想教育农民,自觉地以阶级斗争为纲,坚持党的基本路线,从政治、经济、思想文化各个领域加强无产阶级对资产阶级的全面专政。

大寨之所以能够把巩固无产阶级专政的任务落实到基层,多快好省地发展社会主义农业,在于有一个坚决执行毛主席革命路线的、敢斗敢干的党支部,在于领导权牢牢地掌握在真正的马克思主义者和广大贫下中农的手里。大寨党文部非常注意经常开展批评和自我批评,不断加强自身的革命化建设,坚持学习马列和毛主席著作,坚持反对资产阶级思想的腐蚀,坚持干部参加集体生产劳动,坚持接受群众监督。正是这样,反对资本主义就能不断胜利,大干社会主义就能不断前进。

“农业学大寨”的群众运动,已经成为亿万农民的自觉行动,不仅大寨式的社队遍及全国,而且涌现出一大批学大寨的先进县,农村人民公社更加巩固,农业生产连续多年丰收。

当前,在全国范国内,正在掀起一个“农业拿大寨”,普及大寨县的伟大的革命群众运动。普及大寨县,是我们党领导广大农民深入社会主义革命,加快社会主义建设的伟大进军。建设大寨县,就是要把大寨精神推广到全县,使全县的各级领导、各个部门都能象大寨那样地工作,使大寨式社、队在全县占到绝大多数或大多数。只要我们坚决执行毛主席制定的路线、方针、政策和工作方法,相信群众,依靠群众,社会主义农业的现代化就一定能够实现,我们现在还很落后的乡村就一定能够建设成为一个繁荣昌盛的社会主义新农村。

\section{社会主义工业的发展}

\subsection{走自己工业发展的道路,实现社会主义工业化}

无产阶级在夺取政权后,一项重要的任务,就是要实行工业国有化,建立和发展社会主义的物质技术基础。对于原来工业比较落后的国家,在革命胜利后,则必须迅速发展现代工业,实现社会主义工业化,以充分发挥工业在国民经济中的主导作用。

旧中国遗留下来的工业基础是极端落后的。一九四九年全国解放时,工业产值只占工农业总产值的百分之三十点一,工业产值中现代工业只占百分之五十六点四,其中生产资科生产只占百分之二十六点六,机器制造业只占百分之二点七,钢的年产量只有十五万八千吨。在工业分布上也极不合理,百分之七十以上的工业都集中在少数几个沿海城市。面对这样一个薄弱的工业基础,无产阶级革命胜利以后,迅速发展工业,实现社会主义工业化,就显得更加迫切。

什么是社会主义工业化呢?我国实现社会主义工业化的根本要求是什么呢?毛主席指出:“从中国境内肃清了帝国主义、封建主义、官僚资本主义和国民党的统治(这是帝国主义、封建主义和官僚资本主义三者的集中表现),还没有解决建立独立的完整的工业体系问题,只有待经济上获得了广大的发展,由落后的农业国变成了先进的工业国,才算最后地解决了这个问题。”在我国实现社会主义工业化,就是要在全国建立一个以钢铁工业和机械工业为中心,部门结构齐全,大中小型企业相结合,地区分布比较合理的社会主义工业体系。

这个工业体系必须是独立的,即主要是依靠本国的人力、物力和财力,自力更生地生产各种机器设备、原材料和其他工业产品,满足社会主义扩大再生产和对国民经济进行技术改造的需要,满足人民生活不断增长的需要。

这个工业体系必须是比较完整的,即工业部门结构比较齐全,重工业和轻工业,基础工业和加工工业,国防工业和民用工业,特别是为农业服务的工业,都能有计划按比例地得到发展。

这个工业体系必须是现代化的,即用先进的技术设备武装起来,在生产工艺上采用先进的科学技术,劳动生产率能够赶上和超过世界的先进水平。

社会主义工业化的逐步实现,必然会提高社会主义全民所有制在整个国民经济中的比重,增强国营经济的领导力量,进一步现固工农联盟和无产阶级专政;能够充分发挥工业在国民经济中的主导作用,向农业和国民经济其他部门提供现代化的技术装备,使农业逐步实现机械化,促进整个国民经济的现代化;能够保证社会主义国家在经济上和政治上的独立自主,增强支援世界革命的物质力量,更好地履行无产阶级国际主义义务。

怎样实现社会主义工业化呢?实现工业化有两条根本不同的道路:一条是独立自主、自力更生的道路;一条是依赖或者掠夺外国的道路。资本主义国家的工业化,走的是后一条道路。社会主义国家要实现工业化,决不能走掠夺外国和出卖主权、出卖人民利益的道路,而只能是走独立自主、自力更生的道路。

毛主席根据马克思列宁主义的基本原理,结合我国革命和建设的实践,制定了一整套两条腿走路的方针,总结出了一条独立自主、自力更生,多快好省地实现社会主义工业化的道路。其主要内容就是:

在充分发展农业和轻工业的基础上,加快重工业的步伐,实行农、轻、重并举的方针。毛主席指出:“这里所讲的工业化道路的问题,主要是指重工业、轻工业和农业的发展关系问题。我国的经济建设是以重工业为中心,这一点必须肯定。但是同时必须充分注意发展农业和轻工业。”重工业要较快地发展,必须抓好重工业的建设,否则社会主义建设就会遭到挫折,社会主义工业化就不能实现,这是毫无疑义的。但是,重工业的较快发展,要受到国民经济各方面条件的制约,其中最重要的是受农业,其次是受轻工业的制约。所以,重工业的较快发展,不能脱离农业这个基础,抛开轻工业这个相应的条件,而孤立地去发展。如果,用形而上学的观点看待工农业的关系,只抓重工业,而忽视农业和轻工业的发展,最终必将反过来阻碍重工业的发展,给社会主义建设事业带来损害。因此,把农业放在首位与较快地发展重工业并不是矛盾的。只有坚持以农、轻、重为序安排国民经济计划,迅速发展农业、轻工业,才能使重工业发展的基础更加牢固,确保重工业的较快发展,实现社会主义工业化。

第二,在集中领导、全面规划、分工协作的条件下,实行中央工业和地方工业同时并举,大型企业和中小型企业同时并举,多搞中小型企业的方针。在中央的集中统一领导下,充分发挥地方的积极性,利用地方的资源和潜力,发展地方工业,有利于社会主义工业化的实现。毛主席指出,“地方应该想办法建立独立的工业体系。首先是协作区,然后是好多省,只要有条件,都应建立比较独立的但是情况不同的工业体系。”地方工业体系的建立不能脱离全国工业体系的要求,而地方工业体系的建立又能促进全国工业体系的形成和发展。因此,必须把建立全国工业体系同建立地方工业体系正确地结合起来。

“我们必须逐步地建设一批规模大的现代化的企业以为骨干,没有这个骨干就不能使我国在几十年内变为现代化的工业强国。但是多数企业不应当这样做,应当更多地建立中小型企业”。大企业一般具有设备先进、生产效率高、产品成本低等优点,能够解决国民经济中有决定意义的关键问题。要改变我国工业的落后面貌,就必须集中人力、物力和财力,积极地建设一批现代化的大型骨干企业。但是,发展大型企业是要受一些条件限制的。而举办中小型企业,不仅具有投资少、建设易、收效快等优点,而且可以充分利用当地闲置设备和资源,有利于大搞群众运动,调动各方面的力量来办工业。因此,大型企业和中小型企业并举,多搞中小型企业,是充分调动广大群众的力量,多快好省地进行工业建设的重要方针。

在独立自主、自力更生的基础上,实行洋法生产和土法生产并举的方针。实现社会主义工业化,在工业生产建设上要尽量采用现代化的技术,即洋法生产。但是,我国原有的技术基础比较薄弱,土法生产是工人群众在生产实践中,从具体情况出发,因陋就简,克服困难,发扬自力更生、艰苦奋斗精神的伟大创造。因此,土洋并举不是权宜之计,而是一个长期的方针。实行土洋并举的方针,并不排除学习外国的先进经验和先进技术,而是要把学习外国的东西同自己的创造结合起来,能洋就洋,不能洋就土,从土到洋,逐步发展,不断提高,赶上和超过世界先进水平。这是一个既符合技术发展的规律,又体现了党的依靠群众多快好省地进行工业建设的正确方针。

实践证明,坚持独立自主、自力更生,认真贯彻执行一整套两条腿走路的方针,是迅速实现我国社会主义工业化的唯一正确的道路。在毛主席的革命路线指引下,在无产阶级文化大革命和批林批孔运动的推动下,我国的工业建设取得了伟大胜利。一九七三年,工业产值已占工农业总产值的百分之七十一。过去没有的新的工业部门建立起来了,过去自己不会设计和制造的许多重要的工业产品,现在能独立地设计和制造了。我国社会主义工业化取得的巨大成就,已初步改变了我国“一穷二白”的落后面貌,为建立一个独立的比较完整的工业体系打下了牢固的基础,并进一步发挥了工业在国民经济中的主导作用,促进了整个国民经济的迅速发展。

刘少奇、林彪一类把我国工业的发展,寄托于帝国主义、社会帝国主义的“仁慈怜悯”,主张跟在他们的后面一步一步地爬行,其目的就是要把中国变为帝国主义、社会帝国主义的附庸。我们必须彻底批判他们推行的反革命修正主义路线,提高执行毛主席革命路线的自觉性,坚定不移地走自己工业发展的道路,为尽快地实现社会主义工业化而努力。

\subsection{发展工业要以钢为纲}

发展工业,还要正确处理工业的内部关系。我们知道,工业的内部关系,除了重工业和轻工业的关系外,在重工业内部,又有钢铁、机械、煤炭、化工、电力等许多部门,这些部门之间存在着密切的依赖关系。在这种错综复杂的关系中,只有抓住重点,才能带动一般。

毛主席根据马克思列宁主义关于社会再生产的基本原理,在总结国内外社会主义建设经验的基础上,提出了“以钢为纲”的方针。钢铁工业是整个工业的基础,是带动其他工业部门发展的中心。马克思列宁主义关于社会再生产的原理告诉我们,在生产资料的生产中,制造生产资料的生产资料生产,必须比制造消费资料的生产资料生产的增长要快,这是在技术不断进步的情况下,实现社会扩大再生产的必要条件。钢铁正是制造生产资料的生产资料,是制造机器的主要原材料。有了钢铁,才能制造机器,有了机器,才能用现代化的技术装备来武装农业、轻工业、国防工业和国民经济其他部门。同时,钢铁工业的发展,也要求煤炭、电力、机械、交通运输等部门与它相适应地发展,这就可以带动和促进其他工业部门的发展。因此,“以钢为纲”的方针,是工业生产内部客观比例关系的正确反映。发展工业“以钢为纲”,这就抓住了工业生产内部的主要矛盾,正确处理了钢铁工业和其他工业之间的辩证关系。以钢为纲,纲举目张,必将促进整个工业的全面发展。

发展钢铁工业,矿山必须先行。在钢铁工业内部,包括采矿、冶炼、轧钢三个主要生产环节。有了矿石,才能炼铁、炼钢;有了钢,才能轧制成材。因此,发展钢铁工业,必须首先抓紧“开发矿业”,不能搞“无米之炊”。

刘少奇一伙在钢铁工业中,推行“抓中间,带两头”的反动方针,强调冶炼而忽视采矿和轧钢,妄图使钢铁工业陷入“无米之炊”的困境,破坏社会主义工业的发展。林彪一类则炮制了一个“电子中心论”,鼓吹什么“中国工业要以电子为中心”,“带动、促进整个机械工业和其他工业”,疯狂反对“以钢为纲”的方针。电子技术对于实现工业、国防的现代化固然十分重要,但电子工业并不是基础工业,它不仅不能提供工业发展所需要的原材料和动力,相反地还需要其他工业部门为它提供原材料和技术装备。显然,林彪一类鼓吹“电子中心论”的目的,就是妄图从根本上破坏我国社会主义工业的发展。

\subsection{工业学大庆}

发展社会主义工业,实现社会主义工业化,充满着两个阶级、两条道路、两条路线的尖锐斗争。

大庆油田的广大工人群众和干部,在毛主席的革命路线指引下,粉碎了刘少奇、林彪一类反革命修正主义路线的破坏和干扰,战胜了帝国主义和社会帝国主义的经济技术封锁,坚持了“鼓足干劲,力争上游,多快好省地建设社会主义”的总路线,发扬了独立自主、自力更生、艰苦奋斗、勤俭建国的革命精神。十五年来,他们以阶级斗争为纲,坚持党的基本路线,坚持无产阶级政治挂帅,贯彻执行“鞍钢宪法”,克服了生产建设中的重重困难,不仅只用了三年的时间,就拿下了大庆油田,为我国石油基本自给作出了贡献;而且一直保持着高速度的发展,革命和生产呈现一派热气腾腾的大好形势。大庆油田,是在两条路线斗争中,按照毛主席的无产阶级革命路线,多快好省地建设社会主义工业的光辉典范。大庆的道路,是社会主义工业发展的必由之路。毛主席号召我们“工业学大庆”。

学习大庆,就要象大庆那样,认真学习马列主义、毛泽东思想,努力把思想和政治路线搞端正。大庆的石油会战,是靠“两论”(《实践论》、《矛盾论》)起家的。大庆的广大工人和干部始终不渝地坚持唯物论和辩证法,反对唯心论和形而上学。他们运用“两论”的基本思想,坚持党的基本路线,紧紧抓住两个阶级、两条道路、两条路线斗争这个主要矛盾,顶住了逆流,用无产阶级政治统帅经济,使企业沿着社会主义方向不断前进。正因为大庆油田做到了打击阶级敌人不间断,批判资产阶级、修正主义不停顿,抵制资本主义倾向不松懈,坚持在阶级斗争、路线斗争中打主动仗、进攻仗,才能在困难时期搞上去,胜利形势下不转向,文化大革命以来大发展。

学习大庆,就要象大庆那样,坚持独立自主、自力更生、艰苦奋斗、勤俭建国的革命精神。会战初期,大庆的广大工人和干部怀着无产阶级的雄心壮志,有条件要上,没有条件创造条件也要上,“头顶青天,脚踏草原”,人拉肩扛,住干打垒,发扬革命加拼命的精神,只用了三年的时间,就建设成初具规模的大庆油田。他们在成绩的面前,戒骄戒躁,继续革命,提出了“大干社会主义有理,大干社会主义有功,大干社会主义光荣,大干了还要大干”的豪迈口号。广大职工保持旺盛的干劲,越是艰苦,干劲越足,越是困难,越要胜利。他们以“敢笑天下第一流”的英雄气慨,同洋奴哲学、爬行主义决裂,把革命精神同科学态度结合起来,大搞技术革新和科学实验,不断攀登科学技术高峰。他们全面贯彻执行党的社会主义建设总路线,坚持多快好省的要求,高质量、高速度地开发建设油田。一九六六年以来,大庆的原油产量以平均每年递增百分之三十一的速度持续跃进。就原油产量来说,现在的一个大庆,等于文化大革命前的五个大庆。

学习大庆,就要象大庆那样,加强党的领导,建设一个革命化的领导班子,培养一支革命化的工人阶级队伍。大庆党委教育各级领导干部,坚持学习马列著作,坚持“三个面向”\footnote{面向群众,面向基层,面向生产。}、“五到现场”\footnote{生产指挥到现场,政治工作到现场,材料供应到现场,设计科研到现场,生活服务到现场。}的革命作风,凡是要求群众做到的,领导干部首先要做到。大庆党委注意抓基层,打基础,充分发挥党支部的战斗堡垒作用和广大党员的先锋模范作用。大庆党委紧密联系阶级斗争和路线斗争的实际,坚持对广大职工进行阶级教育、路线教育,培养一支具有阶级斗争和路线斗争觉悟、坚持艰苦奋斗的优良传统和“三老四严”\footnote{对待革命事业,要当老实人,说老实话,办老实事。对待革命工作,要有严格的要求,严密的组织,严肃的态度,严明的纪律。}“四个一样”\footnote{不论是一个小分队、一个班组、一个人,在干工作中,黑夜和白天一个样,坏天气和好天气一个样,领导不在场和领导在场一个样,没有人检查和有人检查一个样。}的革命作风的队伍。

总之,大庆经验的根本之点,就是以阶级斗争为纲,坚持党的基本路线,贯彻执行“鞍钢宪法”,狠抓两个阶级、两条道路、两条路线的斗争,用无产阶级政治统帅经济,发扬独立自主、自力更生、艰苦奋斗、勤俭建国的精神,促进生产力的迅速发展。特别可贵的是,大庆坚持五·七道路,实行工农结合、城乡结合,为限制资产阶级法权和逐步缩小三大差别提供了很好的经验。

在毛主席的“工业学大庆”的伟大号召下,大庆的道路日益深入人心,“工业学大庆”的群众运动蓬勃发展,大庆式的企业不断涌现。这必将加快社会主义工业的发展,促进我国社会主义工业化的实现。

\chapter{社会主义社会的商品制度}

在社会主义社会里,还存在着商品生产和货币交换。我们既要看到它存在的必然性,又要充分地认识它是私有经济的遗物,仍然体现着资产阶级法权,是产生资本主义的土壤。因而,在发挥它的历史作用的同时,对它的消极作用必须加以限制。只有这样,才能促进社会主义经济基础的巩固和发展,加强工农联盟,防止资本主义复辟。

\section{社会主义社会的商品和货币}

\subsection{社会主义社会仍然存在着商品生产和商品交换}

在分析社会主义社会的商品制度之前,需要先弄清楚什么是商品。商品是用来交换的劳动产品。作为商品,首先必须有用,是可以用来满足人们某种需要的东西。商品的这种有用性,叫做使用价值。然而,并不是任何有用的东西都是商品。作为商品还必须是劳动产品。其次,只有通过交换(指买卖的形式)而到达别人手里的有用的劳动产品,才能成为商品。那末,千差万别的使用价值怎么能够进行比较而互相交换呢?那是因为它们在生产中都花费了人类的劳动。这种凝结在商品中的人类一般劳动,就是价值。因此商品具有使用价值和价值两种属性,是使用价值和价值的统一体。

人们之间按照价值互相交换商品,实际上是互相交换各自的劳动。因而,商品与商品之间的关系,体现着人与人之间的生产关系。

这种以交换为目的的生产,就是商品生产。商品生产是一个历史范畴,是同生产发展的一定历史阶段相联系的。它产生的条件:社会分工和生产资料私有制。

由于出现了社会分工,每个生产者只能生产一种或几种产品,而人们的需要又是多种多样的,为了满足这种多方面的需要,人们需要互相交换产品。由于私有制的存在,决定了这种交换只能采取商品买卖的形式。在原始社会一个很长的时间里,并没有商品生产。只是到了原始社会末期,出现了社会分工和生产资料私有制,才产生了专门以交换为目的的商品生产。

在社会主义社会以前,有过两种形式的商品生产,即小商品生产和资本主义商品生产。它们都是建立在私有制基础上的,但又具有不同的性质。小商品生产是以生产资料的个体私有制和个人劳动为基础的商品生产。小商品生产者的生产条件和熟练程度各不相同,生产同一商品的个别劳动时间也不一样,但在市场上只能按相同的价格出卖。这种情况,必然引起两极分化。在封建社会末期,由于小商品生产者的两极分化,少数生产条件好的,积累了较多的财富,雇用工人变成了资本家,而多数人则贫困破产,丧失生产资料,沦为无产者,从而导致资本主义生产关系的产生。所以,“资产阶级是产生于商品生产的”。资本主义商品生产是以资本家占有生产资料和剥削雇佣劳动、为摄取剩余价值而进行的商品生产。在资本主义社会,商品生产发展到最高阶段,不仅一般劳动产品普遍采取了商品的形式,而且劳动力也成了商品。

 在社会主义社会,所有制变更了,公有制代替了私有制,但为什么还存在商品生产和商品交换?这主要是由于社会主义公有制还存在着两种形式。因为,两种社会主义所有制的同时并存,全部生产资料和产品并不是归整个社会所有,还存在着不同的所有者。社会主义全民所有制国营经济的生产资料和产品属于代表全体劳动人民的国家所有,而劳动群众集体所有制集体经济的生产资料和产品则归各个集体所有。然而,由于社会分工的存在,每个经济单位只能生产一种或几种产品,因此,国营经济和集体经济之间、以及各个集体经济之间都互相需要对方的产品。例如,国营经济需要农村集体经济生产的粮食、各种经济作物和其他农副产品;农村集体经济需要国营经济生产的农业机器、化肥、农药等生产资料和其他日用工业品。那末,在产品分别属于不同的所有者,互相又不能无偿调拨对方产品的条件下,它们之间的经济联系如何实现呢?“为了保证城市和乡村、工业和农业的经济结合,要在一定时期内保持商品生产(通过买卖的交换)这个为农民唯一可以接受的与城市进行经济联系的形式”。这就是说,在不同所有有的公有经济之间,只能实行商品制度,即各自把产品作为商品来生产,进行商品交换,以自己的产品去换取对方的产品,以达到满足各自需要的目的。如果不是实行商品生产和商品交换,而互相无偿调拨对方的产品,那实际上就是否定两种社会主义所有制的同时并存。

两种社会主义所有制之间实行商品交换,决定了全民所有制企业之间的交换关系,也需要通过商品货币形式来进行。我们知道,各个社会主义全民所有制的企业并不是生产资料和产品的不同的所有者,它们之间的交换关系只是全民所有制内部的一种产品调拨关系,所交换的产品实质上已经不是商品。但是,社会主义经济是一个统一的整体,两种社会主义所有制之间的商品关系必然要反映到全民所有制内部的交换关系上来,国营企业在经营管理上又具有相对的独立性,这就决定了国营企业之间的产品交换也要利用商品货币形式,进行计价付款,实行等价交换的原则。

社会主义社会实行商品制度,最终是与生产力发展水平有关,只要还存在两种社会主义所有制,只要全民所有制还拿不出极为丰富的产品来对全国人民实行按需分配,就只能实行商品生产和商品交换。

由此可见,在社会主义社会里,商品生产和商品交换的存在是不可避免的,在国家计划的统一领导下,进行商品生产和商品交换,有利于沟通国营经济和集体经济之间、工业和农业之间、城市和乡村之间的经济联系,有利于工农业生产的发展。

\subsection{对商品制度中存在的资产阶级法权必须加以限制}

社会主义的商品生产,同资本主义的商品生产是有区别的。首先,社会主义商品生产是在公有制基础上进行的,目的不是为了利润,而是为了满足社会和人民群众的需要。它体现了社会主义制度下工农之间,国家、集体和个人之间的相互关系。其次,在社会主义制度下,商品的范围受到了一定的限制,劳动力不再是商品,土地、河流、矿藏等重要资源已退出了商品领域,而不象资本主义商品生产那样无所不包。第三,社会主义的商品生产是在国家计划的规定和指导下进行的,消除了竞争和无政府状态。至于现阶段还存在的那一部分小商品生产,则可以通过无产阶级专政国家的领导和管理,社会主义经济的调节,而引导他们沿着社会主义方向发展。这些都是由于同商品生产相联系的经济条件发生了变化,而在社会主义公有制基础上形成的特点。

社会主义的商品生产和商品交换,虽然有上述特点,但它毕竟还是旧社会遗留下来的东西,“跟旧社会没有多少差别”,仍然是产生资本主义的土壤和条件。

社会主义社会的商品生产和商品交换所体现的生产关系虽然发生了变化,但商品的属性并没有什么变化,仍然具有使用价值和价值的两种属性,存在着使用价值和价值的矛盾。在资本主义社会里,资本家生产商品,不是为了商品的使用价值,而是为了攫取工人劳动所创造的剩余价值。他们之所以生产一定的使用价值,只是因为使用价值是价值的物质承担者。如果不提供一定的使用价值,商品没人要,就不能实现价值,也就不能撮取更多的剩余价值。唯利是图的资本家为了发财致富,往往以次充好,以假充真,什么人间丑事都干得出来的。而在社会主义社会里,生产的目的不是为了追求利润,而是为了满足整个社会和人民群众的需要,因而商品生产首先是为了生产使用价值。关心商品的价值,只是为了不断地减少劳动耗费,为社会提供更多的积累,从而更好地满足整个社会和人民群众的需要。但是,社会主义社会的商品仍然是使用价值和价值的统一体,这就为那种背离国家计划,追求产值和利润的资本主义倾向提供了条件。由于社会主义社会还存在阶级、阶级矛盾和阶级斗争,资产阶级利已主义思想还在我们中间腐烂发臭并且毒害我们,唯利是图的资本主义原则还会侵入到我们队伍中来,破坏社会主义的经济关系。这样,一些资本主义思想严重的人就会在他们所把持的企业中,搞资本主义经营,背离社会主义方向,只重视价值生产,片面追求产值和利润,而忽视使用价值的生产,将国家和人民群众需要的产品甩在一边。这种情况发展下去,企业的性质就会改变,社会主义企业将蜕变为资本主义企业。

任何不同的商品相交换,都要求在价值相等的基础上进行,社会主义社会的商品交换也必须遵循等价的原则。这个等价交换原则,从形式上看来是平等的,而实际上却掩盖着不同商品生产者由于生产条件不同而形成的在劳动耗费上的不平等,因而体现着资产阶级法权。在社会主义社会中,这种资产阶级法权的存在,就会引起经济条件上存在差别的集体经济单位之间收入高低的不同,出现实际上的不平等。如果对此不加限制,将会发生两极分化的现象,产生资本主义和资产阶级。

商品经济中的买卖关系,还会侵入社会主义的政治生活,以至党内生活,腐蚀干部和群众的思想,滋长钻营个人私利的资产阶级作风,把人们相互关系变成赤裸裸的利害关系。从而,就有可能产生资本主义和资产阶级,导致资本主义的复辟。

由此可见,社会主义社会的商品生产和商品交换,决不象苏修叛徒集团鼓吹的那样,是什么“不会出现剥削,不可能产生资本主义,不可能转变为资本主义的商品生产”,而是产生资本主义的土壤和条件。问题在于如何正确地对待它。因此,我们既要看到社会主义社会实行商品制度的必然性,要在国家经济计划的统一领导下发展商品生产和商品交换,以满足国家和人民群众的需要;同时又必须看到商品制度是旧社会的痕迹,跟旧社会没有多少差别,还存在着资产阶级法权,是产生资本主义的土壤,必须在无产阶级专政下加以限制,坚持社会主义计划经济,把商品的生产和交换纳入国家计划,由国家实行统一的管理和监督,并逐步创造最终消灭商品制度的条件。

\subsection{社会主义社会的货币仍然是一般等价物}

在社会主义社会中,由于存在着商品生产和商品交换,也就必然存在着货币和货币交换。列宁指出,在社会主义条件下,“能不能一下子把货币消灭呢?不能。还在社会主义革命以前,社会主义者就说过,货币是不能一下子就废除的”,“货币暂时还要保留下来”。

货币是一种固定地起着一般等价物作用的商品,它是在商品生产和商品交换的长期发展过程中而自发产生的。贷币具有价值尺度、流通手段、贮藏手段、支付手段和充当世界货币的职能。它是“社会财富的结晶”,有了货币就可以同一切商品相交换。

在不同的社会制度下,货币所反映的生产关系是不同的。几千年来,在剥削阶级统治的社会里,货币一直是剥削阶级用来掠夺和剥削劳动人民的工具。在资本主义制度下,劳动力成为商品,资本家利用货币购买劳动力,使货币转化为资本,残酷地剥削无产阶级和劳动人民。而在社会主义制度下,由于所有制变更了,货币的发行和管理大权掌握在无产阶级专政的国家手里,成为国家有计划地进行社会主义革命和社会主义建设的工具。

货币在社会主义经济中起着重要作用。首先,国家利用货币作为价值尺度,对国民经济实行计划管理和对国营企业实行核算,计划和监督社会产品的生产和分配。其次,国家利用货币作为商品流通的媒介,实现工农业产品的交换,巩固工农联盟。第三,国家利用货币作为支付手段,实现企业对国家上缴利润,实现国家对企业单位的投资拨款和向事业单位拨付经费,以及对职工发放工资进行消费品的分配。

我国在肃清了伪币、外币和禁止金银流通,制止了国民党反动派统治造成的长期恶性通货膨胀的基础上,建立起独立自主的、统一的、稳定的贷币制度。我国的人民币执行着货币的职能。我国在毛主席的革命路线指引下,工农业生产不断发展,财政收支平衡,以及货币流通的计划调节,使人民币保持了长期的稳定。我国人民币的稳定,对促进生产的发展,沟通城乡物资交流发挥着重要作用。

我们对于社会主义社会的货币和货币交换,不能只看到它存在的必要性和为社会主义服务的一面,还必须看到它是旧社会的痕迹,是产生资本主义和资产阶级的土壤的一面。社会主义社会的货币仍旧是一般等价物,谁握有货币,就等于占有了商品和财富。但是,不同的人握有货币的数量是有多有少的,从而所能购买的东西也是有多有少的。因此,在货币交换中存在着形式上平等而事实上不平等的资产阶级法权。由于社会主义社会还存在着阶级、阶级矛盾和阶级斗争,“金钱万能”的资产阶级思想还在散发着臭气,从而驱使一些人无限度地追求货币,追求发财致富。虽然社会主义制度不允许把货币转化为资本、变成剥削他人劳动的手段,但有些握有货币的人,仍有可能将货币转化为资本,变成剥削人的工具。事实上,有一些新老资产阶级分子和想上升为资产阶级的人,用各种合法的和大量非法的手段占有尽可能多的货币,拿来办黑工厂、黑包工队,变成工业资本,或者拿来买商品,搞长途贩运、投机倒把,变成商业资本,或者拿来放债,变成借贷资本,进行资本主义剥削。在历史上,货币和货币交换的发展,曾促使原始社会公有制的瓦解和私有制的产生。那末,今天在社会主义制度下如果对货币和货币交换中的资产阶级法权不加限制,任其自发发展,就会使资本主义剥削行为泛滥起来,导致资本主义复辟的严重后果。

因此,我们在自觉地利用货币来为社会主义和社会主义建设事业服务的同时,对货币交换中的资产阶级法权,必须在无产阶级专政下加以限制。

在货币交换领域中加强无产阶级专政,要充分发挥国家银行的无产阶级专政工具的作用。“大银行是我们实现社会主义所必需的‘国家机构’”,“没有大银行,社会主义是不能实现的。”社会主义的国家银行,是全国的信贷中心、结算中心和现金出纳中心,是货币发行银行。银行的一切业务活动,都属于货币交换范畴。因此,银行对于在货币交换领域中加强无产阶级专政,负有重要责任。这就要求银行以阶级斗争为纲,坚持党的基本路线,认真贯彻执行党和国家的金融政策、法令、制度,充分发挥它对经济活动的服务和监督的职能。银行要按国民经济发展的需要有计划地发行货币,稳定货币,对资本主义投机倒把的现象进行斗争;要加强现金管理,缩小现金流通的范围,扩大非现金结算,打击货币交换中出现的资本主义活动;要加强银行的信贷监督,督促社会主义企业改善经营管理,并同资本主义经营倾向作斗争,要开展信用合作,对非法的信用活动以及高利贷进行斗争。国家银行对有利于社会主义,有利于对资产阶级实行全面专政的新生事物,要满腔热情地给予支持;对那些违背政策,扩大资产阶级法权,利用货币交换搞资本主义活动的行为,则要坚持原则,敢于斗争,坚决抵制,充分发挥其作为无产阶级专政工具的作用。

\section{社会主义社会的价值规律}

\subsection{社会主义社会存在商品生产,价值规律也就必然存在并发生作用}

价值规律是商品生产的经济规律。凡是有商品生产存在的地方,就有价值规律的存在并发生作用。

价值规律的基本内容是:商品的价值由生产商品的社会必要劳动时间决定,商品的交换要按照商品的价值来进行。各种不同的商品要在价值相等的基础上进行交换,这就是我们通常所说的“等价交换”。但是,商品在交换时,它的价格(商品价值的货币表现)与价值往往是不一致的。这种情况并不是对价值规律的否定,而正是它发生作用的表现形式。正如恩格斯指出的:在以私有制为基础的商品经济中,“只有通过竞争的波动从而通过商品价格的波动,商品生产的价值规律才能得到贯彻,社会必要劳动时间决定商品价值这一点才能成为现实”。

在社会主义社会,由于还存在着商品生产,价值规律也就必然存在并继续起着作用。但是,价值规律在社会主义经济中的作用,同它在私有制商品经济中的作用相比,有很大的不同,它的作用范围和程度已经受到了很大的限制。我们知道,价值规律在以私有制为基础的商品生产的条件下,是社会生产和流通的调节者。它的自发调节作用是通过市场供求关系的变化、价格围绕着价值上下波动来实现的,它使商品生产者将生产资料和劳动力转移到商品的价格高于价值的部门,退出那些商品的价格低于价值的部门。因而,它是作为统治人们的一种异己力量表现出来的。价值规律的这种自发作用,经常引起小生产者的两极分化。而在社会主义制度下,由于生产资料公有制的建立,社会生产是有计划地进行的,价格由国家统一规定,价值规律已不再是社会生产和流通的调节者,调节社会生产和流通的是反映社会主义基本经济规律和国民经济有计划按比例发展规律要求的国民经济计划。它也不再是统治人们的一种异己力量,而是可以被人们认识和自觉利用的,作为完成国家计划的工具。

社会主义国家自觉利用价值规律,来促进社会主义建没的发展,主要表现在以下几个方面,

社会主义国家可以利用价值规律对社会主义生产的影响,促进工农业生产的发展,以利于国家计划的实现。“价值规律在我国社会主义生产中,并没有调节的意义,可是它总还影响生产,这在领导生产时是不能不考虑到的”。价值规律对两种社会主义经济的生产的影响程度是不完全相同的。社会主义国营企业的生产是由国家直接安排的,但它们是独立的经济核算单位,在经营管理上具有相对的独立性,原材料和产品价格的高低仍是企业组织生产时所关心的问题。因此,国家应考虑到价值规律的要求,合理地安排价格,有利于促进国营企业生产的发展,全面地完成国家计划。价值规律对社会主义集体经济的生产,则有较大的影响。社会主义集体经济的生产是在国家计划指导下进行的,但各个集体经济是自负盈亏的经济单位,它的积累水平以及成员的收入水平都直接取决于它本身的生产收入。因此,集体经济在安排生产时,要考虑各种产品的价格和生产成本,往往倾向于多生产一些成本低而收入高的产品。所以,国家在加强对集体经济的计划领导,深入进行社会主义教育的同时,合理地安排产品的收购价格以及各类产品之间的比价,有利于指导他们多生产哪些产品、少生产哪些产品,促进生产的发展,保证国家计划的完成。

社会主义国家可以利用价值规律对消费品流通的重要影响,来指导消费,作为国家计划调节的工具。我们知道,在社会主义制度下,进入流通领域的消费品的总量和构成,是由国家制定的统一计划决定的。但是,除了在一定时期内,一些同国计民生关系很大而一时又不很充裕的消费品,如粮、棉、油、布等,由国家实行计划供应外,对于每个消费者购买其他消费品的数量和品种是不能直接加以计划规定的,而是由人们根据自己的购买能力和需要来选购的。因而,在人们的购买力一定的情况下,某种消费品价格的高或低,就会减少或增加消费者对它的需求,相应地销售量也会随之缩减或增加。由此可见,价值规律对某些个人消费品的流通有着重要的影响。国家利用价值规律的这个作用,通过有计划地调整商品价格,可以促使某些消费品的供需平衡和国家商品流转计划的完成。

社会主义国家利用价值规律,促使社会主义企业搞好经济核算,贯彻勤俭节约的方针。在社会主义制度下,社会主义企业各类产品的价格是由国家统一制定的。这样,企业生产产品的劳动耗费的多或少,直接关系到它是盈利、保本还是亏损。因此,国家根据党在一定时期的政治经济任务,考虑到价值规律的作用,以生产某一产品的社会必要劳动时间为基础,规定这一产品的统一价格,可以促使社会主义企业搞好经济核算,贯彻勤俭节约的方针,为国家和集体提供更多的积累。

在我们自觉利用价值规律的同时,要充分地认识到价值规律虽然不再是社会生产的调节者,它的自发性已经受到了严格的限制,但它毕竟是商品生产的经济规律,充分地反映了商品制度中的资产阶级法权,因而只要它还存在,就不可避免地要发生消极作用,带来危害。

表现在价值规律的作用会引起一些集体经济单位之间富裕程度的不同,形成事实上的不平等。我们知道,商品的价值是由生产商品的社会必要劳动时间决定的,而各个自负盈亏的集体经济单位,由于生产条件各不相同,生产同一种农产品所耗费的劳动时间也是不一样的,有的个别劳动时间高于社会必要劳动时间,有的个别劳动时间则低于社会必要劳动时间。可是,在价值规律要求的等价交换的原则下,各个集体经济单位的同种产品是按照同一的价格出售的。这样,由于它们的个别劳动时间与社会必要劳动时间的不一致,条件好的集体经济单位较之条件差的集体经济单位收入要高,就会出现富裕程度的差别。

表现在价值规律的自发作用会冲击国家计划。我们知道,社会主义国家制定商品的价格时,要以价值为基础,但并不意味着价格与价值完全一致,实际上也存在着价格与价值相背离的情况。正是这样,要求等价交换的价值规律的存在和发生作用,有可能促使某些国营企业热衷于生产价格高、利润大的产品,而妨碍全面完成国家计划;也会促使有些集体经济单位去追求生产价格高、收入多的产品,如有些农村集体经济单位擅自更改国家规定的种植计划,大搞自由种植等。

表现在价值规律的自发作用给资本主义自发势力以可乘之机。在社会主义社会往往存在着价格与价值不一致的情况,特别是农村集市贸易的价格基本上是随着供求关系的变化而涨落的。这就可能为资本主义自发势力所利用,大搞长途贩运,投机倒把,牟取暴利,破坏国家计划。

对于价值规律的上述消极作用,如果不加限制,势必会使目前存在的富裕程度的不同变成贫富悬殊,导致两极分化;会使一些国营企业和集体经济单位不顾国家和人民的需要,片面追求产值、利润,破坏国民经济计划,改变公有制的性质;会使小生产的自发资本主义倾向泛滥起来。被推翻的地主资产阶级以及新生的资产阶级分子,也会利用商品制度和价值规律,破坏社会主义计划,腐蚀党员、干部、职工和贫下中农,使商品制度成为他们复辟资本主义的工具。因此,我们必须以阶级斗争为纲,坚持无产阶级专政,坚持政治挂帅,加强国家的计划领导和管理,限制价值规律的消极作用,防止资本主义自发势力的发展,防止阶级敌人利用它进行复辟活动。

在怎样认识和对待价值规律的问题上,一直存在着两条路线的激烈斗争。苏修叛徒集团为在苏联全面复辟资本主义,竭力扩大价值规律的作用,公开鼓吹价值规律是“社会主义社会生产的客观调节者”。他们利用价格、利润来刺激企业的生产,改变了企业的社会主义性质,整个国民经济也陷于混乱之中。在我国,刘少奇一类出于地主资产阶级的反革命本性,忽而鼓吹立即取消商品生产和商品交换,忽而叫嚷什么“千规律,万规律,价值规律第一条”、“要用价格刺激生产”,妄图以此取消国家的统一计划领导,由价值规律自发地调节生产和流通,从而瓦解社会主义经济,复辟资本主义。

在社会主义社会,价值规律的存在并发生作用,是不依人们的意志为转移的。我们的原则是“计划第一,价格第二”,在计划调节生产和流通的基础上,自觉利用价值规律的积极作用,限制价值规律的消极作用。

\subsection{社会主义社会的价格是国家有计划规定的计划价格}

在商品货币关系存在的条件下,国家需要正确运用价格这一工具,来为发展社会主义经济服务。

社会主义社会中的价格,主要是由国家有计划制定的。计划价格的制定,是国家自觉利用价值规律为社会主义建设服务的具体体现。

国家在制定价格时,必须服从党在一定时期的政治经济任务,首先从政策和计划的要求出发,兼顾国家、集体和个人的利益,同时也要考虑价值规律的要求,以价值为基础,贯彻等价交换的原则。有些商品的价格的制定,也要参考市场供求关系。因此,价格以价值为基础,并不意味着价格与价值完全一致,实际上是存在着价格与价值相背离的情况的。

社会主义国家在制定计划价格时,采取稳定物价的政策。稳定物价是社会主义计划经济的客观要求,因为只有保持物价的相对稳定,才有利于实行计划经济,打击资本主义势力,巩固和发展社会主义经济,促进工农业生产,安定和改善人民生活。

稳定物价是我国物价工作的指导方针。二十多年来,我们坚决执行了毛主席的革命路线,批判了刘少奇一类“用价格指导生产”、“全面涨价”、“大涨大落”的反动主张,全国物价一直保持基本稳定。

当然,保持物价稳定,不等于“冻结”物价。我们的原则是“基本不动,个别调整”,随着国民经济的发展,在保持物价总水平的前提下,对不合理的价格进行适当的调整。

国家在制定计划价格时,要正确处理各种产品之间、特别是工农业产品之间的价格比例关系。正确处理工农业产品价格的比例关系,对于缩小工农、城乡之间的差别,巩固工农联盟,具有重要意义。

在旧中国,由于帝国主义、封建主义和官僚资本主义残酷地掠夺农民,形成了工业品价格增长速度快于农产品价格增长速度的工农业品价格的“剪刀差”。解放后,根据毛主席的教导,粉碎了刘少奇一类“卡农民脖子”、“以高对高”,妄图剥削农民,破坏工农联盟的阴谋,逐步对工农业品价格作了合理调整,缩小了“剪刀差”。二十多年来,农产品价格逐步有所提高,而供应农村的工业品价格稳中有降,从而使农民能够在正常年景下,从增加生产和商品交换中,逐年增加集体积累和个人收入,进一步巩固了工农联盟,促进了农业生产的发展。

此外,对工业品采取了薄利多销的政策。工业品薄利多销,价格稳中有降,可以使国家有合理的积累,产品有广阔的销路。这就促进了生产和流通,改善了人民生活,有利于巩固工农联盟和物价的稳定。当然,薄利多销也要体现区别对待的原则。

社会主义社会的计划价格,是计划经济的一个重要方面。我们必须坚持中央统一领导和地方分级管理的原则,加强物价管理工作,限制价值规律的消极作用,坚决同资本主义倾向作斗争。对某些企业不执行计划价格,自行提价和乱订价格的资产阶级经营作风,要坚决制止。对农村集市贸易要从各方面加强领导和管理,使集市价格接近于国家计划价格。

\section{计划调拨和社会主义商业}

\subsection{交换由生产决定又反作用于生产}

生产中创造的社会产品,只有经过交换,才能进入生产消费和个人消费。因此,交换是联结生产与消费的中间环节,是保证社会再生产顺利进行的必要条件。这个从生产到消费的运动过程,就是流通过程。

在社会再生产过程中,生产与交换是密切联系又互相制约的对立统一的关系。马克思指出:“一定的生产决定一定的消费、分配、交换和这些不同要素相互间的一定关系。”也就是说,生产是基础,是起决定作用的,它不仅决定着交换的物质内容和规模,而且还决定着交换的性质和形式。但是,交换在社会再生产过程中决不是一个被动的消极因素,它“也反过来对生产运动起作用”,在一定程度上影响生产的发展速度、规模和结构,影响人民生活的状况。

在社会主义社会中,交换主要是由国家有计划有组织地进行的。国家有计划地组织交换,把工农业产品及时地集中起来,有计划、有步骤地把生产资料供给各个生产单位,从而使生产和再生产过程得以畅通无阻地进行,促进工农业生产的发展;并且把工农业生产的大量价廉物美、丰富多彩的消费品,进行合理的调配和及时供应,也就可以使广大人民群众的生活需要得到满足。同时,国家有计划地组织交换,坚持无产阶级政治挂帅,限制商品制度中的资产阶级法权,坚决抵制资产阶级不正之风,同资本主义倾向进行斗争,有利于巩固社会主义经济基础和无产阶级专政,防止资本主义复辟。反之,如果不去很好地组织交换,流通过程就会发生混乱,势必造成工农业生产部门一方面是产品积压,另一方面是不能及时得到规格对路的生产资料,人民群众对消费品的需要也不能得到保证。社会产品的产、供、销就会失去平衡,资本主义自发势力也就会乘机猖撅起来。从而社会主义建设事业和人民生活改善就会受到阻碍,工农联盟和无产阶级专政的巩固也会受到影响。

马克思主义关于生产与交换辩证关系的基本原理告诉我们,社会主义交换必须从生产出发,大力支援和促进工农业生产的不断发展;在此基础上,才能保障国家和人民群众不断增长的需要。

刘少奇一类蓄意颠倒生产和交换的位置,大肆鼓吹“流通决定生产”的谬论,胡说什么“企业生产要听你指挥,要什么就生产什么”、“农民也听你们指挥”。他们片面夸大交换的作用,要生产部门围着市场转,用价值规律、供求关系来调节和支配生产,其目的就是妄图改变社会主义交换的性质,破坏生产,窒息流通,进而瓦解社会主义经济基础和无产阶级专政,以实现其复辟资本主义的迷梦。我们必须批判刘少奇一类的修正主义路线,将交换放在一个恰当的位置上,有计划地合理地进行组织,从而充分地发挥社会主义交换的积极作用。

社会主义交换,要有适当的组织形式。在社会主义社会,存在着全民所有制与集体所有制之间;全民所有制企业之间;集体所有制经济之间;社员之间,非农业个体劳动者之间,以及他们同城镇居民和商业部门之间,国营企业同职工之间的交换关系。这五种交换关系所体现的经济关系是不同的,因而交换的形式也是不同的,分别通过国家计划调拨和社会主义商业来进行。

\subsection{全民所有制企业之间的交换主要采用国家计划调拨的形式}

全民所有制企业分布在各个部门、行业,从事不同的生产,需要相互取得各自所需的生产资料,它们之间也就必然要发生交换关系。这种交换关系,是在全民所有制内部进行的,并不发生所有权的转移,因而它体现着全民所有制内部的经济联系。由于社会主义国家对全民所有制企业的产品有权进行直接的调拨和分配,因而它们之间的交换是通过国家计划调拨的形式进行的。

全民所有制企业之间交换的产品,主要是生产资料。社会主义经济是计划经济,为了保证社会主义生产有计划按比例地发展,也就要求对重要生产资料的生产和分配更加直接地纳入国家计划的轨道。因此,客观上也有必要对全民所有制企业之间的交换,采取国家有计划地调拨和分配。

至于全民所有制企业之间的部分小额的零星的生产资料的交换,则须过商业部门的中间环节来进行,以利于满足各方面的需要。

社会主义国家对生产资料的计划调拨,是社会主义经济的一个重要特点,具有重要的意义。首先,社会主义国家根据党的方针政策和国民经济计划的要求,区别轻重缓急,有计划地组织生产资料的调拨和分配,从而为国民经济有计划按比例的发展提供了可靠的物资保证。其次,社会主义国家有计划地调拨和分配生产资料,搞好调剂,有利于不断调整生产资料的供需矛盾,使之在一定时期内保持相对平衡,从而保证各个部门、企业生产的顺利发展。第三,生产资料的计划调拨,减少中间环节,缩短流通过程,可以加速物资的周转,促进生产力的进一步发展。同时,社会主义国家通过对生产资料的计划调拨,把生产资料牢牢地掌握在自己的手里,有利于加强计划管理和打击资本主义自发势力,防止新老资产阶级分子利用生产资料的自由买卖来恢复资本主义经营,巩固社会主义经济基础和无产阶级专政。

社会主义国家对生产资料的计划调拨,是通过一定的调拨和分配方式来实现的。我国的物资供应体制实行“统一领导,分级管理”原则,在集中统一领导下,国家根据各类生产资料在国民经济中的重要性和用途,划分为“统一分配物资”、“部管物资”和“地方管理物资”,由各级部门分工经营。统一分配物资是一些对发展国民经济具有决定意义的物资,如钢、铜和重要的机电设备,由国家计划部门集中掌握、统一分配。部管物资是一些在国民经济中有重要作用的物资,如锡、镍以及配套性和专业性强的物资,由中央各主管部门平衡分配。地方管理物资则是上述两项以外的物资,由有关省、市、自治区平衡管理。

我国对物资的计划调拨,有定点供应、直达供货和物资部门供应两种基本形式。定点供应、直达供货是在国家的统一计划安排下,按照企业之间的产品供应合同,产品由生产单位直接向使用单位供货,不再经过任何中间周转环节的一种交换形式。这种形式适合于生产量和需要量比较大、产品的供需关系比较稳定的企业单位之间的交换。物资部门供应,是国家通过物资部门作为中间经营环节的一种对物资计划调拨的形式。它是生产单位的产品,按照产品供应合同,发运到国家物资部门,经过物资部门必要的加工和整理,再有计划地分散供应给使用单位。这种形式适用于某些生产单位比较集中,而使用单位比较分散的产品。

在国家对生产资料进行计划调拨的物资工作中,历来存在着两个阶级、两条道路的斗争。由于社会主义社会的商品制度中存在着资产阶级法权,一些受修正主义路线毒害和资产阶级思想侵蚀较深的人,就会以物揽权,以权谋私,破坏社会主义计划经济。我们要从路线上分清是非,限制商品制度中的资产阶级法权,批判资产阶级思想,切实搞好国家对生严资料的计划凋拨,保证社会主义建设事业多快好省地发展和无产阶级专政的巩固。

\subsection{社会主义商业是组织商品流通的一种经济形式}

在社会主义社会,除了全民所有制企业之间的交换采取国家计划调拨的形式以外,其他的几种交换都是通过社会主义商业进行的。这是因为,这几种交换是不同所有者之间的交换,经过交换发生产品所有权的转移,所以它们只能采取市场买卖的形式来实现。社会主义商业,正是组织商品流通的一种经济形式。

社会主义商业是国民经济的重要部门。它是沟通工业与农业、城市与乡村、社会主义两种所有制之间的经济联系的桥梁,联结生产与消费的纽带,对于发展工农业生产和满足人民生活需要,加强工农联盟和巩固无产阶级专政,都起着极为重要的作用。

社会主义商业要发挥自己的作用,必须以阶级斗争为纲,坚持党的基本路线,正确处理农商关系、工商关系以及商业和个人消费者之间的关系。

首先,在农商关系上,社会主义商业必须根据“兼顾国家利益、集体利益和个人利益”的原则,正确安排农副产品的购留比例,既要使国家得到必要数量的农副产品,又要不影响集体生产和农民生活的妥善安排。同时,在农副产品的收购价格上,也应当坚持等价交换或近乎等价交换的原则,既要有利于调动农民的社会主义积极性,又要保证国家的合理积累和市场物价的稳定。

在农副产品的收购中,要积极推行结合合同制。毛主席早在我国农业合作化期间就曾指出:“供销合作社和农业生产合作社订立结合合同一事,应当普遍推行。”订立结合合同,可以使集体经济的生产计划和国家对农副产品的收购计划直接衔接起来,有效地把集体经济一些分散零星的农副产品和社员家庭副业纳入社会主义计划经济的轨道,这既能保证国家经济建设和人民生活的需要,又在一定程度上堵塞了“自由贸易”的通道。

社会主义商业还要积极做好工业品下乡的工作,及时地供应农村所需要的生产资料和生活资料,做到品种规格对路、质量良好、价格适当,保证农业生产和农民生活的正常需要。

农村商业工作决不是单纯的经济工作,在同农民联系的过程中,要按照党的基本路线,通过本身的职能工作,积极主动地为改造小生产的残余,战胜旧的习惯势力,不断向农民灌输社会主义思想,批判资本主义倾向,引导农民坚持走社会主义道路。只有这样,才能更好地促进农村集体经济的巩固和发展,加快农业生产前进的步伐,为社会主义建设和广大劳动人民提供更多的农副产品,从而进一步加强工农联盟,巩固无产阶级专政。

在工商关系上,社会主义商业要密切工商协作,积极协助工业部门搞好产品规划,帮助解决原材科的供应,促进工业生产的发展。对于那些有销路而成本偏高的新产品,特别是支农产品,要实行一定时期内工业保本、商业无利或适当补贴的办法,以扶持生产的发展。社会主义商业还要主动向工业部门反映工农兵的意见,协助工业部门努力提高产品质量和增加新品种。

社会主义商业在正确处理工商关系时,要以阶级斗争为纲,坚持党的基本路线,不仅要克服自己对工业部门一手拿“刀子”,一手拿“鞭子”,多了砍、少了赶的资本主义倾向;也要对工业部门的“利润挂帅”,利大大干、利小小干、无利不干的资本主义经营进行坚决的斗争。

在与个人消费者的关系上,社会主义商业必须坚持为工农兵服务的方向,一切要从方便工农兵出发,积极增加经营品种和服务项目,合理设置网点,制定合理的方便群众的手续制度和购销形式,认真改进服务态度,提高服务质量。在目前莫些商品暂时还不够丰富的情况下,国家采取计划分配的办法,这是在商品分配方面限制资产阶级法权的一种措施。资产阶级的“走后门”的不正之风,扩大了资产阶级法权,不仅违背党和国家现行的商品供应政策,而且破坏社会主义计划经济,甚至为贪污盗窃、投机倒把大开方便之门。因此,社会主义商业要坚决抵制资产阶级不正之风,严格执行党和国家制定的商品供应政策,坚决为大多数人服务,以限制商品制度中的资产阶级法权。

\subsection{社会主义商业组织商品流通,必须通过一定的流通渠道。}

全民所有制的国营商业,掌握了绝大部分的商品以及全部商业批发环节,是社会主义市场的主体和领导力量。

供销合作社是社会主义商业的重要组成部分,它从上到下设有自己的领导管理系统。各级供销社在党的路线、方针、政策和国家计划指导下,领导农村商业,负责收购农副产品,供应农村需要的农业生产资料和日用工业品。它在社会主义革命和社会主义建设中都起着重要作用。

集体所有制的城镇合作商店,具有店小、分散、经营灵活的特点,对于增加商业网点,方便群众生活,有着一定的作用。但是,必须加强对它的领导和管理,以防止和克服可能出现的资本主义自发倾向。

在现阶段社员还保留着自留地和家庭副业的情况下,国家领导下的农村集市贸易,是国营商业和合作社商业的补充。它具有两重性:一方面是社会主义商业的补充,是农民之间互通有无、调剂余缺的场所;另一方面又是无计划市场,具有一定的自发性,如果任其自流发展,就会冲击社会主义计划市场,滋长资本主义自发势力。因此,对农村集市贸易必须切实加强领导和管理,按照政策规定,限制交易对象,限制上市品种,限制价值规律的破坏作用,打击投机倒把活动,取缔黑市交易,制止弃农经商。

市场始终是阶级斗争的重要场所。列宁指出:“贸易自由就是说倒退到资本主义去”。苏修叛徒集团在商业中广建自由市场,破坏社会主义计划经济,培植资本主义势力,瓦解社会主义经济基础,以实现其全面复辟资本主义的目的。在我国,刘少奇一类也竭力鼓吹大搞“自由市场”,其目的就是妄图扩大商品制度中的资产阶级法权,促使资本主义自发势力的泛滥,以打开缺口,实现资本主义复辟。“为了不让资本家和资产阶级的政权复辟,就要禁止投机买卖,就要使某些人不能用损人利己的手段来发财致富”。我们对商品制度中的资产阶级法权必须加以限制,反对商品经济的自由化,充分发挥国营商业的领导作用,搞好商品的计划供应和正常流通,加强对农村集市贸易的领导和管理,正确地发展商品生产和商品交换,加快社会主义建设步伐,加强工农联盟,巩固无产阶级专政。

\chapter{社会主义制度下的节约和经济核算}

社会主义建设所需要的资金积累,要靠广大人民群众自力更生、艰苦奋斗来解决。建立和加强经济核算制,厉行节约,提高劳动生产率,对于多快好省地建设社会主义具有重要意义。正确地认识这个问题,有助于我们进一步树立勤俭建国、厉行节约的思想和艰苦朴素的作风,加强企业管理,更好地贯彻节约原则,在各项工作中不断地提高劳动生产率,以促进社会主义建设事业的迅速发展。

\section{社会主义制度下的节约}

\subsection{节约是社会主义经济的基本原则}

所谓节约,是指人力、物力和财力的节省。一切节约,归根到底是劳动时间的节约。因为,人力的节省是活劳动的节约,物力的节省是物化劳动的节约,而财力的节省则是活劳动和物化劳动节约的货币表现。厉行节约,是多快好省地发展社会主义经济的客观要求,是由社会主义经济的性质所决定的。

我们知道,社会主义基本经济规律要求迅速地发展社会主义经济,以满足社会和人民群众日益增长的需要。而社会主义经济的高速度发展,需要有大量的建设资金。那末,所需要的建设资金从哪里取得呢?在资本主义制度下,资产阶级是通过剥削本国劳动人民、掠夺殖民地或向外国借取奴役性的贷款,来进行资本积累的。在社会主义制度下,社会主义经济的性质决定它解决资金积累的途径根本不同于资本主义。社会主义积累不能靠剥削和掠夺,也不能靠举借外债,只能靠广大人民群众自力更生、艰苦奋斗来解决。在增加生产的同时,厉行节约,是社会主义国家自力更生地增加积累,加快社会主义建设的一个重要途径。

在生产建设中贯彻节约原则,合理地分配和使用劳动力,充分调动人们在生产中的劳动积极性,不断地挖掘劳动潜力,提高劳动生产率,不仅可以节约劳动力的使用,而且能使现有的劳动力创造更多的财富;合理地分配和使用生产资队特别是节约使用粮、钢、煤、电等基本原材料和动力,使现有的资源发挥更大的效用,不仅可以生产更多的产品,而且可以增加社会主义积累,扩大再生产的规模。至于非物质生产部门,如国家机关、部队、学校、人民群众团体等单位,不断地反对铺张浪费,节约开支,就能减少国家财政支出,使国家有更多的资金用于经济建设。所以,厉行节约,用同样的人力、物力和财力办更多的事,可以促进社会主义革命和社会主义建设事业的发展。

节约不仅有重大的经济意见,而且有着重大的政治意义。贯彻节约原则,大力提倡艰苦奋斗、勤俭节约的优良作风,教育广大干部和群众坚持自力更生地搞生产、搞建设,这就有利于抵制追求享受、贪图安逸的资产阶级思想的侵蚀,巩固无产阶级专政,防止资本主义复辟。同时,贯彻节约原则,多快好省地建设社会主义,可以有更大的物质力量支援世界革命,履行无产阶级的国际主义义务。

我国是一个发展中的社会主义国家,厉行节约具有特别重要的意义。毛主席指出:“我们要进行大规模的建设,但是我国还是一个很穷的国家,这是一个矛盾。全面地持久地厉行节约,就是解决这个矛盾的一个方法。”建国以来,我国在毛主席的革命路线指引下,独立自主、自力更生地开展了大规模的经济建设,经济面貌发生了巨大的变化。但是,“要使我国富强起来,需要几十年艰苦奋斗的时间,其中包括执行厉行节约、反对浪费这样一个勤俭建国的方针”。为尽快地把我国建设成为实现农业、工业、国防和科学技术现代化的强大的无产阶级专政的社会主义国家而努力。

社会主义制度不仅要求厉行节约,而且也为节约开辟了广阔的道路。

在资本主义制度下,资本家为了取得更多的剩余价值,总是力图“节省”生产费用的。他们为了达到这一目的,采用了压低工人工资、恶化劳动条件和提高劳动强度等办法。因而,资本家资本支出的节约,是与广大工人群众的利益相对立的,必然要遭到反抗和抵制。同时,“资本主义生产方式迫使单个企业实行节约,但是它的无政府状态的竞争制度却造成社会生产资料和劳动力的最大的啸傲消费”。资本主义社会经常存在着庞大的失业队伍,特别是周期性地爆发生产过剩经济危机,是对社会劳动的巨大浪费。

社会主义制度则为全面持久地厉行节约,提供了可能性。在社会主义制度下,由于生产资料公有制的建立,节约的目的是为了使劳动人民为社会提供更多的积累,多快好省地建设社会主义,更好地满足整个社会和人民群众的需要,因而厉行节约同劳动人民的根本列益是一致的,有着牢固的群众基础。同时,社会主义的计划经济,国家在整个社会范围内有计划地分配人力、物力和财力,也就使它们能够得到合理的节约使用。

\subsection{厉行节约,反对浪费}

“节约是社会主义经济的基本原则之一。”在社会主义建设中,是勤俭节约,还是铺张浪费,存在着两个阶级、两条道路、两条路线的斗争。毛主席早在抗日战争时期就指出:“节省每一个铜板为着战争和革命事业,为着我们的经济建设”。解放后又进一步指出:“勤俭办工厂,勤俭办商店,勤俭办一切国营事业和合作事业,勤俭办一切其他事业,什么事情都应当执行勤俭的原则。”刘少奇、林彪一类为了复辟资本主义,疯狂地对抗毛主席的指示,推行了一条铺张浪费的修正主义路线,任意挥霍浪费国家财产,破坏了社会主义经济基础,腐蚀了群众的革命意志。在毛主席革命路线指引下,广大群众批判了刘少奇、林彪一类的修正主义路线,开展了增产节约、反对浪费的群众运动,发扬了艰苦奋斗、勤俭节约的优良作风,推动了我国社会主义建设事业的迅速发展。

勤俭节约是无产阶级和劳动人民的品德,铺张浪费则是资产阶级和一切剥削阶级的恶习。无产阶级和劳动人民对自己劳动的物质成果,从来是十分珍惜和爱护的,以挥霍浪费为耻;而资产阶级和一切剥削阶级则是靠剥削和掠夺起家的,必然轻视劳动人民用血汗换来的物质成果,以挥霍浪费为荣。因此,要很好地贯彻节约原则,肃清修正主义路线的流毒,必须开展对资产阶级思想的批判,同铺张浪费等不良倾向进行斗争。

贯彻节约原则,要求人们根据对立统一的规律,正确处理增产和节约、质量和节约之间的关系,多快好省地建设社会主义。我们知道,增加生产和厉行节约是不可分割的统一体,两者不可偏废。增加生产,可以为社会提供更多的物质财富,满足社会主义扩大再生产的需要;厉行节约,则可以节省人力、物力和财力,少花钱多办事,把它们用到最需要的地方去,从而促进社会主义建设事业的发展。如果只讲增产而不顾节约,就不可能有真正的增产;反之,忽视了增产,节约就失去了积极的目的。因而,要把节约和增产正确地结合起来,在增产中厉行节约,用节约的办法促进增产。

提高质量和厉行节约,也是密切相联系的。提高产品的质量,可以延长产品的使用期限和改善产品的使用性能,从而节约了人力和物力的消耗;而厉行节约,合理地选用原材料,用质高价低的原材料替代质差价高或质高价高的原材料,也就既实行了节约又提高了产品的质量。因此,要在提高质量的过程中厉行节约,在节约的基础上不断提高质量。

我们必须坚持全面节约的观点,清除那种把节约与增产、质量对立起来的错误认识,在实践中正确处理多快好省的辩证关系,加快社会主义经济的发展。

\section{社会主义制度下劳动生产率的不断增长}

\subsection{社会主义制度为劳动生产率的不断增长开辟了广阔的道路}

劳动生产率,是指人们在生产中的劳动效率,是由劳动者所生产的产品数量与所消耗的劳动时间的对比来表示的。在单位时间内,生产的产品少,劳动生产率就低;反之,劳动生产率就高。提高劳动生产率,以较少的劳动耗费生产更多的产品,也就意味着劳动时间的节约。

在社会主义制度下,不断提高劳动生产率具有重要意义。

不断提高劳动生产率,是迅速发展社会主义经济的重要途径。我们知道,发展生产有两条途经,一是增加劳动量,包括增加生产资料和劳动者的人数,延长劳动时间和提高劳动强度;一是提高劳动生产率。在社会主义社会,发展生产,需要增加生产资料和劳动者的人数。但是,它只能增加社会生产的总量,却不能提高按人口平均计算的生产量,并且最终也还要受到农业发展水平的限制。至于用延长劳动时间和加强劳动强度的办法来增加生产,则是社会主义制度所不能允许的。因而,发展社会主义生产,还必须增加每个劳动者单位时间内的产量,也就是提高劳动生产率。只有不断提高劳动生产率,才能保证社会主义生产持续地高速度发展。同时,劳动生产率的提高,可以减少产品生产的劳动消耗,降低单位产品成本中的工资支出,以及减少分摊到单位产品成本上的固定费用,从而促进产品成本的降低,为社会提供更多的积累,加快社会主义建设的速度。

不断提高劳动生产率,是改善人民生活的重要条件。生产决定分配,只有社会物质财富的增加,人民的生活水平才能得到提高。不断提高劳动生产率,为逐步改善人民生活提供了必要的物质条件。

不断提高劳动生产率,是巩固无产阶级专政和实现共产主义的必要条件。社会主义经济是无产阶级专政的物质基础。劳动生产率的不断提高,可以壮大社会主义经济,从而保证无产阶级专政建立在日益强大的物质基础之上。同时,无产阶级要实现共产主义,必须以阶级斗争为纲,坚持无产阶级专政,把社会主义革命进行到底,促进社会生产力的发展,提供实现共产主义所必需的物质条件。正是在这个意义上,没有劳动生产率的不断提高,实现共产主义也是不可能的。列宁指出:“提高劳动生产率是一个根本的任务,因为不这样就不可能最终地过渡到共产主义。”

劳动生产率的增长是社会发展的一般经济规律。从人类社会发展历史来看,一个新社会制度之所以能代替旧的社会制度,其重要原因之一,在于具有更高的劳动生产率。列宁指出:“劳动生产率,归根到底是保证新社会制度胜利的最重要最主要的东西。”但是,在不同的社会制度下,由于经济条件不同,劳动生产率增长的速度也是不同的。

在资本主义社会的初期,由于资本主义生产关系代替了落后的封建主义生产关系,机器生产代替了手工劳动,劳动生产率得到了比较迅速的提高。但是,在资本主义的剥削和压榨下,劳动者并不关心生产的发展。资本主义生产的目的是追逐剩余价值,新技术、新机器的采用受到了极大的限制。同时,由于资本主义社会所固有的生产的社会性和资本主义私人占有制之间的矛盾,必然导致周期性的经济危机。因而,在资本主义社会里,劳动生产率的提高是缓慢的和不稳定的。随着资本主义的发展,资本主义生产关系日益成为生产力发展的障碍,劳动生产率的增长不断地遭到破坏。正如马克思指出的:“对资本来说,劳动生产力提高的规律不是无条件适用的。”

在社会主义社会,生产资料公有制的建立,为劳动生产率的不断增长创造了社会经济条件。第一,社会主义公有制的建立,消灭了资本主义的剥削,工人阶级和劳动人民是企业的主人,他们在劳动中具有充分的社会主义积极性和创造性,主动关心生产的发展和劳动生产率的提高。第二,社会主义制度消除了资本主义所固有的矛盾,消除了竞争和生产无政府状态,国民经济是有计划按比例发展的,使人力、物力和财力能够得到合理的有效的使用,充分地利用劳动资源和节约劳动时间,促进劳动生产率的不断增长。第三,社会主义制度消除了资本主义采用新技术、新机器的局限性,可以充分地推广和运用先进技术和先进经验,使各部门生产技术不断改进,劳动生产率不断提高。特别重要的是,我们的党和国家以马克思主义、列宁主义、毛泽东思想为指导,制定正确的路线、方针和政策,使社会主义经济制度的优越性得到充分地发挥,从而促进劳动生产率迅速地不断增长。

由此可见,与资本主义制度相比,社会主义制度具有无比优越性,为劳动生产率的不断增长开辟了广阔的道路。

\subsection{任何社会主义经济事业要尽可能提高劳动生产率}

提高劳动生产率,是无产阶级和劳动人民的一项重要任务。毛主席指出:“任何社会主义的经济事业,必须注意尽可能充分地利用人力和设备,尽可能改善劳动组织、改善经营管理和提高劳动生产率,节约一切可能节约的人力和物力,实行劳动竞赛和经济核算,借以逐年降低成本,增加个人收入和增加积累。”

怎样提高劳动生产率呢?我们知道,决定劳动生产率的因素是多方面的。马克思指出:“劳动生产力率由多种情况决定的,其中包括:工人的平均熟练程度,科学的发展水平和它在工艺上应用的程度,生产过程的社会结合,生产资料的规模和效能,以及自然条件。”这就是说,决定劳动生产率即马克思说的劳动生产力的水平的,一是社会经济因素,二是物质技术因素,三是自然因素。自然条件,如地质的状况、资源的分布,以及土地的肥沃程度,对劳动生产率有一定的影响,特别是对农业和采掘业的劳动生产率有较大的影响。但是,随着社会经济条件和物质技术条件的发展,它的影响必将逐步减少。物质技术条件是影响劳动生产率的重要因素,其作用是日益增大的。至于社会经济条件,如劳动的性质、劳动组织、劳动者的政治思想觉悟等,则对劳动生产率的影响尤为直接和重要,它不仅直接决定着劳动生产率的提高,而且还影响或决定着提高劳动生产率的其它因素。

通过对决定劳动生产率的因素的分析,我们可以把社会主义制度下提高劳动生产率的途径,概括为以下几个主要方面。

坚持无产阶级政治挂帅,不断提高劳动者的社会主义觉悟和路线斗争觉悟,树立和发扬共产主义劳动态度。在社会主义制度下,提高劳动生产率的途径是多方面的,但坚持政治挂帅,不断提高劳动者的社会主义觉悟,则是决定性的条件。我们知道,劳动者在生产中是起主导和决定作用的,在一定物质技术条件下,劳动生产率能不能很快地提高,取决于人们的劳动态度,“只有千百万群众的劳动高潮和劳动热情才能保证劳动生产率的不断增长”。而劳动态度又是由思想觉悟程度决定的,思想和政治路线觉悟高,劳动态度就认真,行动就自觉,就能不断提高劳动生产率。毛主席在赞扬我国一九五七年生气勃勃开展的整风运动时指出:“凡是这样做了的地方,人民的社会主义觉悟迅速增长,错误思想迅速澄清,工作中的缺点迅速克服,人民内部的团结迅速加强,劳动纪律和劳动生产率迅速提高。”因此,要不断提高劳动生产率,最根本的就是必须以阶级斗争为纲,坚持党的基本路线,坚持政治挂帅,经常地进行思想和政治路线方面的教育,深入批判修正主义,批判资产阶级思想,使广大劳动者清除旧思想和旧传统的影响,不断提高社会主义觉悟,逐步树立起共产主义劳动态度,充分发挥劳动者的积极性和创造性。

开展技术革新和技术革命,不断提高生产技术水平。科学技术和生产工具的发展水平,是人类控制自然的尺度。采用新的科学技术和新的生产工具,可以生产更多的产品,并减少活劳动和物化劳动的耗费,从而提高劳动生产率,发展社会生产。因此,要提高劳动生产率,必须充分重视技术革新和技术革命。毛主席指出:“中国只有在社会经济制度方面彻底地完成社会主义改造,又在技术方面,在一切能够使用机器操作的部门和地方,统统使用机器操作,才能使社会经济面貌全部改观。”目前,我国的生产技术水平虽然有了飞跃的发展,但要赶上和超过世界先进水平,还需要进行艰苦的努力。开展技术革新,要勇于变革,敢于创新。只要人们的思想从无所作为、因循守旧的传统习惯的束缚中解放出来,发扬自力更生、艰苦奋斗的精神,克服盲目祟洋、懦夫懒汉、轻视群众的思想,实行工人、干部、技术人员和生产、科研、使用单位两个三结合,大搞群众运动,就可以打开技术革新的新天地,有所发现,有所发明,有所创造。当然,开展技术革新,一定要尊重客观规律,讲究实效,注意将革命精神与科学态度很好地结合起来。同时,外国的一切好的经验、好的技术,也要吸取过来,为我所用。

进行必要的生产技术和业务知识教育,不断提高劳动者的文化技术水平和劳动熟练程度。劳动者具有一定的文化技术水平,是有效地掌握和使用生产设备的必要条件。在其他条件相同的情况下,劳动者的文化技术水平和劳动熟练程度越高,劳动生产率也就越高。毛主席指出:“我们的同志必须用极大的努力去学习生产的技术和管理生产的方法”。我们必须为革命认真钻研技术,对技术精益求精,不断提高文化技术水平和熟练程度。任何鄙薄技术工作以为不足道的思想,都是错误的。

加强企业管理,改进生产组织和劳动组织,搞好社会主义的分工与协作。人们的生产和劳动,总是在一定的组织形式下进行的。而合理的生产组织和劳动组织,正确地组织企业内部的分工与协作,使生产连续地有节奏地进行,减少非直接生产人员,节约使用劳动力,是提高劳动生产率的重要方面。同时,从整个社会范围来看,改进生产组织和劳动组织,在国家集中统一领导下,搞好企业之间、部门之间、地区之间的分工与协作,可以有效地利用人力和物质资源,也有利于劳动生产率的提高。

在毛主席的无产阶级革命路线指引下,我国的社会主义革命和社会主义建设事业取得了伟大的胜利,劳动生产率的增长也非常迅速。当前,我国的劳动生产率水平,由于原有的基础十分落后,同一些发达的国家比较起来,还是比较低的。我们为了在本世纪内全面实现农业、工业、国防和科学技术的现代化,使我国国民经济走在世界的前列,必须坚持毛主席的无产阶级革命路线,自力更生,奋发图强,“抓革命,促生产”,迅速提高劳动生产率,多快好省地发展社会主义生产,以加强社会主义经济基础,巩固无产阶级专政。


\section{社会主义的经济核算制}

\subsection{经济核算制是管理社会主义企业的一项重要制度}

社会主义社会要全面贯彻节约的原则,要求在整个国民经济和各个企业中,进行经济核算。

经济核算就是对生产经营过程中的劳动耗费和劳动成果,进行记录、计算、分析和对比的活动。进行经济核算,对经济活动送行分析,可以有效地促进社会劳动的节约,是社会生产发展的客观要求。生产越是社会化,经济核算就越是重要。“簿记对资本主义生产,比对手工业和农民的分散生产更为必要,对公有生产,比对资本主义生产更为必要。”经济核算虽然对于资本主义生产和社会主义生产都是必要的,但由于生产资料所有制性质的不同,经济核算的内容、形式和社会后果也就截然不同,反映着不同的生产关系。

在资本主义社会,资本家进行经济核算的目的,是力图以最少的资本支出榨取尽可能多的剩余价值。因而,资本家越加强经济核算,对雇佣工人的剥则就越残酷。同时,在生产资料的资本主义私有制下,这种经济核算只能在一个企业的范围内进行,而整个社会生产则是盲目的、无计划的,不可能有整个国民经济范围的核算。在社会主义社会,经济核算则反映着社会主义的生产关系。它是为了充分挖掘人力、物力和财力的潜力,降低活劳动和物化劳动的消耗,生产更多更好的产品,充分地满足国家和人民不断增长的需要。同时,由于社会主义经济是以生产资料公有制为基础的计划经济,因而经济核算不仅能在企业内部运用,而且可以在整个国民经济范围内进行。实行社会主义经济核算,可以更好地厉行节约,多快好省地发展社会主义生产,从而有利于巩固和发展社会主义经济基础,巩固无产阶级专政。

在社会主义制度下,国家为了保证对国营企业的集中统一领导,又有利于发挥国营企业的社会主义积极性和经营责任心,对国营企业实行经济核算制的管理。社会主义经济核算制,是贯彻节约原则,管理社会主义企业的一项重要制度。

社会主义国家对国营企业实行经济核算制的管理,包括以下几个方面的内容:

国家对国营企业实行计划管理,给企业规定产量;品种;质量;原材料、燃料、动力消耗;劳动生产率;成本;利润等主要的经济指标,要求企业全面完成。

国家根据国营企业生产经营的需要,拨给企业一定的资金(流动资金和固定资金),企业以此按照国家规定的任务独立进行经营活动。

国家要求国营企业根据国家规定的价格出售产品,用销售产品的货币收入来抵偿生产、业务上的支出,并按照国家规定上缴利润,为国家增加积累。

国家使每个国营企业成为法律上相对独立的经济单位,企业的收入和支出应有一定的制度和手续,按年季月检查生产计划完成程度,进行成本计算,企业有权按照国家规定的计划从银行取得贷款,国家通过贷款对企业的生产经营活动进行监督。

国营企业之间,以及国营企业和集体经济组织之间的经济联系,必须实行等价交换的原则,建立严格的经济合同制度。

从上述对社会主义经济核算制的概述中,可以看到它实质上反映着社会主义国家同国营企业之间的国家集中统一领导和企业相对独立经营的关系,以及企业之间的社会主义互相协作、互相支援的关系。实行经济核算制,一方面可以避免对企业统得过死,束缚企业发展生产的社会主义积极性;另一方面也防止了企业各自为政,搞自由化的资本主义倾向。

在社会主义社会的一个相当长的时期内,还存在着劳动群众集体所有制。每一个集体经济组织是一个核算单位,在国家计划指导下组织生产,按照国家规定的价格出售商品,独立经营,自负盈亏,一方面为集体增加收入和增加积累,一方面通过缴纳税金的形式为国家增加积累。现阶段,我国农村人民公社集体所有制经济一般实行“三级所有,队为基础”的制度,公社、大队和生产队都是独立的自负盈亏的核算单位,各自进行经济核算。因此,公社、大队和生产队之间的财务往来、物资和劳动力的调剂,都要坚持“自愿互利,等价交换”的原则,这是现阶段农村人民公社生产关系的要求。

在农村人民公社集体经济中,公社、大队对各自所办的企业,也实行经济核算制的管理。公社、大队对所办企业实行统一领导、统负盈亏,拨给企业一定的资金,由它们负责运用,完成生产任务和积累任务。

每个社会主义企业是整个国民经济的一个组成部分,企业的经济核算和整个国民经济的核算是不可分割的统一体。我们知道,国民经济的核算,是无产阶级专政国家从整个国民经济有计划按比例、高速度发展的需要出发,对劳动耗费和劳动成果的核算。在社会主义制度下,国民经济的核算和企业的经济核算,总的看来是一致的。但是,由于它们两者的角度不同,有时也会发生矛盾。例如,从企业范围来看是合理的、经济的事情,但从整个国民经济来看却是不合理的、浪费的事情;或者从企业范围来看是不合理的、浪费的事情,但从整个国民经济来看却是合理的、经济的事情。正是这样,企业的经济核算,必须服从国民经济的核篇,这才有利于国家有计划地发展社会主义经济。我们要反对那种只讲企业经济核算,而不顾国家整体利益的现象。当然,同时也要反对企业借口从整个国民经济范围考虑经济效果,而不注意企业经济核算,从而影响企业发展的现象。

\subsection{坚持政治挂帅搞好经济核算}

在社会主义制度下,要不要实行经济核算,以及怎样进行经济核算,在这个问题上存在着两条路线的斗争。

任何社会主义经济事业,必须尽可能改善经营管理,实行经济核算,降低成本,增加积累。社会主义企业的赢利同资本主义企业的利润,有着本质的区别。资本主义企业的利润是剩余价值的转化形式,被资本家无偿地占有的。而社会主义企业的赢利是劳动群众为社会创造的价值,以上缴利润和税金的形式集中到社会主义国家手中,用来扩大社会主义生产和改善人民生活。社会主义的建设资金要靠国民经济的内部积累,而社会主义企业提供的积累是国家收入的重要来源。在我国的预算收入中,来自国营企业的上缴利润和税金就占百分之九十以上。因此,如果社会主义企业没有赢利,不能很好地为国家增加积累,甚至造成不应有的亏损,就会影响国家的收入,不利于社会主义生产的发展和人民生活的改善,拖住社会主义建设的后腿。

社会主义经济要重视从整个国民经济全局利益出发的赢利。斯大林指出:“如果不从个别企业或个别生产部门的角度,不从一年的时间来考察赢利,而是从整个国民经济的角度,从比方十年到十五年的时间来考察赢利(这是唯一正确的处理问题的方法),那末,个别企业或个别生产部门暂时的不牢固的赢利,就决不能与牢固的经久的高级赢利形式相比拟,这种高级赢利形式是国民经济有计划发展这一规律的作用及国民经济的计划化所提供给我们的”。因此,个别企业的赢利,必须服从整个国民经济的赢利。为了整个国民经济的利施可以允许有些企业,如由于工业的合理布局而发展起来的内地工业企业和地方工业企业、生产新产品和支农产品的企业,在一定时期内不赢利,甚至亏损。当然,这些暂时的计划亏损企业,也应切实加强经济核算,厉行节约,降低成本,尽快地扭转亏损和增加积累。

社会主义企业要为国家积累更多的建设资金,必须加强经济核算,搞好经营管理,增加生产,厉行节约。林彪一类却竭力否定社会主义企业进行经济核算的必要性,妄图使社会主义经济陷入混乱状态,破坏社会主义积累,以达到其瓦解社会主义经济基础、颠覆无产阶级专政的反革命目的。

另外,也应看到,社会主义社会还实行商品制度,需要利用同价值相联系的成本、价格、利润等经济范畴,运用货币形式进行经济核算。尽管这些经济范畴可以被利用来为社会主义建设服务,但它们毕竞是商品经济的范畴,是旧社会遗留下来的东西。由于社会主义社会存在阶级和阶级斗争,因而唯利是图的资本主义原则就会侵入我们的社会主义企业,出现“利润挂帅”的情况。“利润挂帅”,是把企业获取利润的多少作为评价企业生产经营工作好坏的唯一指标或最重要的指标,实行有利就干、无利不干的资本主义经营原则。这样做,必然会破坏社会主义经济关系,把企业引向资本主义歧途。它与社会主义企业为了发展社会主义生产,更好地满足社会的需要,在全面完成国家规定的各项经济指标的情况下取得的合理利润是根本不同的。

苏修叛徒集团在篡夺了苏联党和国家的领导权以后,为了破坏社会主义经济基础,竭力利用商品货币关系,推行“完全经济核算制”。他们的“完全经济核算制”,同推行物质刺激相结合,把追逐利润作为企业一切活动的出发点;并给予企业一系列“经营机动自由”,从企业有权“占有、使用和支配”其财产,有权编制和自行决定生产财务计划,有权对某些产品规定很高的“一次性临时议价”的价格,有权提取一部分利润归自己支配,直到有权招收和解雇工人,规定职工工资和支配奖金等。显然,这种“完全经济核算制”,完全背叛了社会主义经济核算制的原则,大搞利润挂帅,从根本上改变了企业的所有制性质,是苏修官僚垄断资产阶级复辟资本主义、剥削苏联劳动人民的工具。在我国,刘少奇一类也以实行经济核算为名,推行“利润挂帅”的修正主义路线,妄图改变企业的社会主义性质,复辟资本主义。

因此,实行社会主义经济核算制,必须坚持无产阶级政治挂帅,批判修正主义,批判资产阶级,限制资产阶级法权,划清社会主义积累和“利润挂帅”的界限,正确处理国家和企业之间、企业和企业之间,以及企业内部的关系。

在社会主义制度下,企业必须在坚决服从国家统一计划的前提下,发挥独立经营的积极性,而绝不能因为实行经济核算,片面地强调独立性,把企业同国家的关系对立起来。社会主义企业应该严格遵守党的各项经济政策,认真执行国家计划,全面完成国家规定的各项经济指标,通过经济核算,不断改善经营管理,努力增产节约,降低成本,积极为国家提供积累。这样的积累越多,对国家的贡献就越大。国家要求企业全面完成各项经济指标,就是加强计划经济,限制商品生产中的消极因素的一个方法。在各项经济指标中,产品的产量、品种和质量是体现产品的使用价值的,必须把它们放在首要地位来考察,这是社会主义生产目的的要求。如果企业不顾整体利益而只顾本位利益,违背国家计划,减少生产产值低、利润小的产品,扩大和超产产值高、利润大的产品,大搞“利润挂帅”,就势必破坏国民经济计划的平衡,程度不同地改变企业的性质,给社会主义建设带来严重的危害。

社会主义企业之间是互相协作、相互支援的关系,在根本利益上是一致的。但是,实行经济核算制的企业又都是相对独立经营的单位,各自计算经济效果的,因而,在经济联系中,企业之间常常会发生经济利益上的矛盾。这就要求各个企业,从全局出发,在国家计划指导下和政策允许的范围内,把困难留给自己,把方便让给别人,互相协作,互通有无。对于一些新老资产阶级分子和资本主义思想严重的人,用资本主义的商品买卖关系,来代替社会主义的协作关系,从中谋取个人或小集团利益的行为,必须坚决予以揭发和批判。

社会主义企业实行经济核算,必须依靠广大工人群众,开展群众核算。工人群众既是生产者,又是企业的主人,他们干什么,算什么,最能直接地反映多快好省的要求,最能坚持生产的社会主义方向。那种排斥群众,依靠少数人搞“专家理财”,是极端错误的。在开展群众核算的基础上,还需要发挥专业核算的作用,把专业核算和群众核算结合起来。只有这样,才能从组织上保证经济核算沿着正确路线开展起来。

\chapter{社会主义社会的分配}

在社会主义制度下,兼顾国家、集体、个人三者的利益,正确地分配社会产品和国民收入,对于社会主义经济的迅速发展和人民生活的逐步改善,具有重大的作用。个人消费品的分配是生产关系的一个重要方面,它由生产资料所有制和人们在生产中的相互关系所决定,又反作用于所有制和相互关系。我们既要认识到在社会主义阶段个人消费品的分配实行“各尽所能,按劳分配”原则是不可避免的,更要认识到对按劳分配中的资产阶级法权必须加以限制的必要性。必须坚持无产阶级政治挂帅,限制分配方面的资产阶级法权,以促进社会主义经济基础和无产阶级专政的进一步巩固。

\section{社会主义社会国民收入的分配}

\subsection{人民收入是社会总产品的一部分}

社会产品生产出来以后,必须经过交换和分配,才能进入消费,满足社会的需要。因此,社会产品的分配,是再生产过程的必要环节。

一个国家在一定时期内(通常指一年)由物质资料生产部门的劳动者生产出来的产品,就是社会总产品。物质资料的生产部门,包括工业、农业、建筑业、货物运输业、服务于生产的邮电业,以及作为生产过程在流通领域内继续的那部分商业(如产品保管、包装等)。非物质资料生产部门,则不创造社会总产品。

由于社会主义社会还存在着商品货币关系,因而社会总产品也还具有两种表现形式。社会总产品在实物形式上,分为生产资料和消费资科两大类,在价值形式上,社会总产品就是社会总产值,包括三个部分:(一)在生产过程中消耗了的生产资料转移到新产品中的价值,(二)劳动者所创造的归自己的那部分产品的价值,(三)劳动者所创造的归社会的那部分产品的价值。

在社会总产品中,扣除用来补偿已经消耗掉的生产资料以后剩下来的那部分社会产品,就是国民收入。从价值形式来看,它是物质资料生产部门的劳动者在一定时期内新创造的全部价值,即社会总产值中的(二)、(三)两部分。从实物形式来看,一部分是扣除补偿已经消耗掉的生产资料以后所剩下来的生产资料;另一部分是该时期内所生产出来的全部消劳资料。

由于国民收入是物质资料生产部门的劳动者所创造的社会总产品的一部分,因而只有生产的发展,才是国民收入增长的源泉。生产发展越快,国民收入的增长也就越快。

在社会主义制度下,由于生产资料公有制的建立,消除了资本主义所固有的矛盾,在马克思主义路线指导下,可以充分地、合理地利用劳动资源,能够不断地提高劳动生产率和节约使用生产资料,使社会主义生产高速度地发展。因而,社会主义社会的国民收入能够以比资本主义社会高得多的速度增长。

\subsection{国民收入要按照劳动人民的根本利益进行分配}

国民收入如何进行分配,是由生产资料所有制形式决定的,即生产资料归谁占有,国民收入就按谁的利益进行分配。在资本主义制度下,生产资料为资本家私人占有,国民收入的分配权掌握在资产阶级的手里,按照资产阶级的利益进行分配。分配的结果,社会财富越来越集中到少数剥削者手中,劳动者则日益贫困化。而在社会主义制度下,生产资料是公有的,劳动人民处于主人翁的地位,国民收入则按照劳动人民的根本利益进行分配,用来满足社会主义国家和劳动人民不断增长的需要。

马克思在《哥达纲领批判》中,精辟地阐述了社会主义社会的社会总产品和国民收入分配的原理。马克思指出,首先要从社会总产品中扣除“用来补偿消费掉的生产资料的部分”。所剩下的是国民收入,从它里面应该扣除“用来扩大生产的追加部分”,“用来应付不幸事故、自然灾害等的后备基金或保险基金”。国民收入作了上述扣除以后,在把剩下用来作为消费资料的部分进行个人分配之前,还得从里面扣除:“和生产没有关系的一般管理费用”,“用来满足共同需要的部分,如学校、保健设施等”,“为丧失劳动能力的人等等设立的基金”等。这些扣除,是为劳动者的整体利益和长远利益服务的,它归根到底还是为了劳动者的利益。正如马克思指出:“从一个处于私人地位的生产者身上扣除的一切,又会直接或间接地用来为处于社会成员地位的这个生产者谋福利”。

社会主义社会国民收入的分配,要经过初次分配和再分配两个过程。

社会主义社会的国民收入,首先是在物质资料生产部门中进行初次分配。由于还存在着两种社会主义公有制,国民收入的初次分配是在两种社会主义经济中分别进行的。

社会主义国营企业所创造的国民收入(即企业的净产值),归国家所有,由国家作统一的初次分配。经过初次分配,分解为两个部分:一部分以工资的形式发给职工,作为职工的个人收入,归他们个人支配;另一部分以税金和利润的形式上缴国家,作为国家集中的纯收入,由国家在整个社会范围内有计划地分配和使用。

集体经济所创造的国民收入,经过初次分配,分解为以下三个部分:

  \begin{enumerate}

\item 以农业税的形式上缴给国家,作为图家集中的纯收入;
\item 归集体经济使用的公积金(包括储备粮)和公益金,用于扩大再生产和举办社员的集体福利事业;
\item 社员的劳动报酬,作为社员的个人收入。

    \end{enumerate}
    
国民收入经过上述的初次分配,形成了社会主义国家、集体经济组织、物质资料生产部门的劳动者的原始收入。

国民收入经过初次分配以后,还必须在整个社会范围内进行再分配。这是因为:首先,在社会主义社会里,陈了工业、农业等物质资料生产部门以外,还存在着各种非物质资料生产部门,如文教卫生部门、商业和一部分服务性行业、国家行政管理机关、军队等。这些非物质资料生产部门对于满足社会的共同需要和巩固无产阶级专政,都是十分必要的。这些部门的工作者的劳动并不创造国民收入,但却是社会主义社会所不可缺少的。因此,维持和发展这些部门的各项费用,以及这些部门的工作者的劳动工资,就必须通过国民收入的再分配来提供。其次,在社会主义社会里,举办社会救济和集体福利事业所需要的社会保证基金,除了集体经济自筹部分以外,也必须通过国民收入的再分配来建立。第三,为了社会主义国民经济有计划按比例地发展和生产力的合理配置,国家需要把从国民收入初次分配中集中起来的建设资金,通过国民收入的再分配,有计划地分配到各个部门和地区。

社会主义社会国民收入的再分配,主要通过以下两个途径进行的:

国家预算。国家预算是国民收入再分配的主要途径。在国民收入的初次分配中,各物质资料生产部门上缴给国家的税金和利润,是预算收入的主要来源。国家根据各个时期政治经济任务的需要,通过预算支出,把集中起来的收入,用于工农业生产建设的投资、文教卫生部门和各种福利设施的费用、国家行政管理和国防的开支,以及对外授助的拔款等。

服务性行业的活动。工人、农民和其他工作者用他们的个人收入,来支付服务性行业提供各种服务所收取的费用。服务性行业将所取得的收入的一部分,作为工资支出,成为它们的工作者的个人收入;另一部分则以税金和利润的形式上缴给国家。此外,国民收入的再分配,在一定程度上还通过价格的变动来进行。例如,降低工业品的销售价格和提高农副产品的收购价格,也就增加了农民的收入。

社会主义社会的国民收入,经过分配和再分配,最后分解为两个部分:一部分归国家和集体经济支配,用来扩大再生产和满足社会的其他共同需要;另一部分归劳动者个人所有,用来满足劳动者的个人需要。最后在使用过程中,又归纳为积累基金、消费基金和援外基金。可列表如下:

    \begin{description}
\item [补偿基金]用于补偿物质生产资料部门已消耗掉的生产资料
\item 国民收入

    \begin{description}
\item 积累基金

    \begin{description}
    
\item [扩大再生产基金]用于工业、农业、运输业等生产性基本建设和增加企业的流动基金
\item [非生产性基金]用于文教卫生部门、国家行政和国防部门的基本建设,以及工农业生产部门的非生产性建设
\item [社会后备基金]国家和企业为了备战备荒,应付意外事变和自然灾害的物质储备


    
    \end{description}
\item 消费基金

    \begin{description}
   
\item 社会消费基金

    \begin{description}
    
\item [国家管理基金]用于国家行政管理和国防开支方面的支出
\item [文教卫生基金]用于科学、教育、保健、艺术事业方面的支出
\item [社会保证基金]国家和企业用于劳动保险、公费医疗和社会救济方面的支出
    
    \end{description}       
\item [个人消费基金]用于支付职工工资和社员劳动报酬
    \end{description}    

\item [援外基金]用于对外援助的支出

    \end{description}

    \end{description}

\subsection{在分配问题上必须兼顾国家利益、集体利益和个人利益}

在社会主义社会,积累和消费之间,存在着又统一又矛盾的关系。

我们知道,社会主义社会的积累和消费与资本主义的积累和消费有着根本的区别。在资本主义社会,积累是资本的积累,是进一步榨取无产阶级和广大劳动人民而摄取最大限度利润的手段。这种积累的结果,“在一极是财富的积累,同时在另一极,即在把自己的产品作为资本来生产的阶级方面,是贫困、劳动折磨、受奴役、无知、粗野和道德堕落的积属”。显然,在资本的积累和劳动人民的消费之间存在着对抗性的矛盾。而在社会主义社会,积累是国家和集体经济用来扩大再生产,发展社会主义经济的。社会主义积累的增加,将使社会生产的规模不断扩大,这就为巩固无产阶级专政和劳动者的物质文化生活的逐步提高,创造了必要的物质条件。因而它是为无产阶级和劳动人民的利益服务的。所以,社会主义社会的积累和消费是统一的,也就是说在根本利益上是一致的。

但是,社会主义社会的积累和消费之间也还存在着矛盾。这是因为,在一个时期内,国民收入是一个既定的量。在国民收入量既定的情况下,积累多一些,虽然可以加快社会主义扩大再生产的速度,但消费就会少一些,劳动人民当前生活的提高就要受到限制;反之,消费多一些,虽然可以较多地满足劳动人民当前的生活需要,但积累就只能少一些,社会主义扩大再生产的速度就要放慢,最终也会影响劳动人民生活的进一步提高。社会主义社会积累和消费之间的这种矛盾,是国家利益、集体利益和个人利益之间的矛盾,是长远利益和当前利益之间的矛盾,是根本利益一致基础上的非对抗性矛盾。

社会主义社会积累和消费的矛盾是非对抗性的矛盾,但不能就此认为它们是无足轻重的。正确处理积累和消费的矛盾,是社会主义建设中一个极其重要的问题,是正确处理人民内部矛盾的一个重要方面。

毛主席指出:“在分配问题上,我们必须兼顾国家利益、集体利益和个人利益。”这是我们正确处理积累和消费矛盾必须遵循的根本原则。首先,必须兼顾国家利益和个人利益,在劳动人民生活适当改善的基础上,尽可能地多增加一些积果。积累是扩大再生产的源泉,为了建设社会主义的全局利益和长远利益,应该发扬独立自主、自力更生、艰苦奋斗、勤俭建国的精神,尽可能地多积累一些。如果片面强调个人利益、当前利益,用不积累或少积累的办法去提高消费的比重,势必妨碍社会主义建设的发展和无产阶级专政的巩固,势必损害劳动人民的根本利益和长远利益。但是,积累的增加必须放在劳动人民生活适当改善的基础上,也就是说必须照顾到劳动人民的个人利益、当前利益。如果片面强调积累,而不注意改善劳动人民的生活,这不仅不符合社会主义生产的目的,也会挫伤劳动人民的生产积极性,不利于社会主义建设。因此,我们在安排积累和消费时,既要满足社会主义扩大再生产的需要,尽可能地多增加一些积累,又要注意适当改善劳动人民的生活,把国家利益和个人利益、长远利益和局部利益正确地结合起来,从而可以调动一切积极因素,有力地促进社会主义建设事业的发展,进一步巩固无产阶级专政。

其次,必须兼顾国家利益和集体利益,正确处现国家积累和农村集体经济积累的关系。一方面农村集体经济应本着先国家后集体的原则,积极完成对国家的纳税义务和国家的购粮任务;另一方面国家也要采取稳定负担、藏富于民的政策,使农村集体经济从增产中不断增加积累。

此外,在处理农村集体经济内部积累和社员收入的关系时,要兼顾集体利益和个人利益,在生产发展的基础上,既要逐步增加集体积累,又要尽可能使社员能够在正常年景下,从增加生产中逐年增加个人收入。集体经济在确定积累和消费的比例时,应按照“以丰补歉”的原则,在丰收时多积累一些,而在歉收时则可以少积累一些。同时,由于各个集体经济的生产条件和自然条件不同,在收入水平上存在着差别,收入水平较高的社队就应该适当地多积累一些。

正确处理积累和消费的关系,必须贯彻兼顾国家利益、集体利益和个人利益的根本原则,但与此同时,也应考虑国民收入的物质构成,即生产资料和消费资料的比例。我们知道,一定数量的资金是需要购买相应数量的物资的。积累基金主要用于扩大再生产,要有相应的生产资料;消费基金也要有相应的消费资料。不论是积累基金还是消费基金,都要求有相应的物资作保证,从而才能得到真正的实现。因而积累和消费的比例,是受着生产资料和消费资料的比例所制约的。一般地说,积累基金的增长应当同生产资料的增长相适应,消费基金的增长应当同消费资料的增长相适应。努力发展生产,生产出更多的生产资料和消费资料,就有利于我们更好地处理积累和消费的关系。

在积累和消费的问题上,历来存在着马克思主义同机会主义、修正主义两条路线的斗争。老机会主义分子拉萨尔和杜林之流,主张在社会主义和共产主义社会中,国民收入要分光,不留积累,实行所谓“不折不扣的劳动所得”,或“全部劳动所得”,妄图用资产阶级的福利主义来欺骗人民群众,把无产阶级革命引向改良主义的歧途。马克思和恩格斯曾对他们进行了无情的揭露和批判。恩格斯指出,在杜林的“共同社会”里,“积累完全被遗忘了”,这个做法就是“直接要求社员去进行私人的积累,因而就导致它自身的崩溃”,“除了重新产生金融巨头以外,再没有其他目的”。刘少奇、林彪一类站在地主资产阶级的立场上,叫嚷什么社会主义就是要“多分一点”、“多拿一点”,恶毒诬蔑我们是“‘国富’民穷”,攻击社会主义积累是什么“变相剥削”。他们妄图用分光、吃光、用光的办法,把国民收入分光吃净,由少数人的“私人的积累”代替社会的公共积累,破坏社会主义经济建设,瓦解社会主义经济,以达到复辟资本主义的目的。我们必须分清路线是非,兼顾国家、集体和个人三者的利益,正确处理积累和消费的关系,发扬独立自主、自力更生、艰苦奋斗、勤俭建国的革命精神,尽可能地多积累一些,以加快社会主义建设的步伐。

\section{社会主义社会个人消费品的分配}

\subsection{“不劳动者不得食”、“各尽所能,按劳分配”是社会主义}

社会个人消费品分配的基本原则

在社会主义社会,个人消费品分配的基本原则是“不劳动者不得食”、“各尽所能,按劳分配”。

“不劳动者不得食”,就是任何一个有劳动能力的人,不参加劳动就不能向社会领取消费品。

“各尽所能,按劳分配”,就是每个有劳动能力的人都应该尽自己的能力为社会劳动,社会则按照劳动者所提供的劳动的数量和质量分配个人消费品。“各尽所能,按劳分配”是一个完整的原则,不能把两者割裂开来,各尽所能是按劳分配的前提条件,首先要各尽所能,然后才能按劳分配。

在社会主义社会,为什么个人消费品的分配实行“不劳动者不得食”、“各尽所能,按劳分配”的原则呢?

任何一个社会的消费品的分配关系,都是由一定的生产资科的占有关系所决定的,而不是以人们的主观意志为转移。马克思指出:“消费资料的任何一种分配,都不过是生产条件本身分配的结果。而生产条件的分配,则表现生产方式本真的性质。”在资本主义社会,生产资料掌握在资本家手里,工人除了自己的劳动力以外没有任何生产资料。资本家凭借着占有生产资料,将工人创造的大量剩余价值摄为己有,过着“不劳而获”的寄生生活;工人在最好的情况下也只能得到相当于自己劳动力价值的工资,来维持劳动力的再生产。而在社会主义社会,生产资料的社会主义公有制代替了生产资料私有制,劳动人民做了生产资料的主人,劳动力不再是商品,劳动产品也归劳动人民所有,按照有利于劳动人民的原则进行分配。同时,在生产资料公有制的条件下,每个人“除了自己的劳动,谁都不能提供其他任何东西,另一方面,除了个人的消费资料,没有任何东西可以成为个人的财产”。因此,在社会主义社会,凡是有劳动能力的人都必须参加劳动,只有向社会提供劳动,才能从社会取得个人消费品,不劳动者不得食。列宁指出:“‘不劳动者不得食’——这是工人代表苏维埃掌握政权后能够实现而且一定要实现的最重要、最主要的根本原则。”

但是,这还没有说明为什么在社会主义阶段必须实行“按劳分配”而不能实行“按需分配”。

要说明这个问题,就必须对社会主义社会作进一步的考察。我们可以看到,社会主义社会在经济、道德和精神方面都还带着它脱胎出来的那个旧社会的痕迹。

 \begin{enumerate}

\item 在社会主义社会,生产力虽然有了很大的发展,但它的发展水平还不很高,社会产品还没有达到极大丰富的程度。因此,还不具备实行按需分配的物质条件。
\item 在社会主义社会,还存在着旧的社会分工,存在着工农之间、城乡之间、脑力劳动和体力劳动之间的差别。这些差别是客观存在的,也不是很快就能消除的。
\item 在社会主义社会,人们思想觉悟虽然有了很大的提高,不计报酬、忘我劳动的先进人物不断涌现,但对大多数人来说,劳动还没有成为生活的第一需要,还被当作谋生的手段。同时,在社会主义社会,始终存在着阶级和阶级斗争,剥削阶级竭力散布“好逸恶劳”、“不劳而获”等腐朽思想来毒害劳动人民,共产主义劳动态度还不可能在短期内普遍地树立起来。因此,在这种情况下,要实行按需分配,完全以劳动者的需要作为个人消费品分配的尺度,是不可能的。而只能实行按劳分配,用劳动作为个人消费品分配的尺度。

 \end{enumerate}

由此可见,在社会主义社会,由于生产资料公有制的建立,由于社会主义社会中还存在着旧社会的痕迹,决定了个人消费品的分配要实行“不劳动者不得食”、“各尽所能,按劳分配”的原则。“权利永远不能超出社会的经济结构以及由经济结构所制约的社会的文化发履。”

\subsection{“各尽所能,按劳分配”既是社会主义原则,又存在着资产阶级法权}

“各尽所能,按劳分配”是社会主义的分配原则。它是在无产阶级专政和生产资料公有制的基础上实行的,是对几千年来建立在生产资料私有制基础上的人剥削人的分配制度的否定。因而,它是分配制度上的一次深刻革命,是一种很大的进步。我国宪法明确规定:“国家实行‘不劳动者不得食’、‘各尽所能、按劳分配’的社会主义原则”,这是在以毛主席为首的党中央的领导下,我国人民经过长期英勇奋斗、流血牺牲,建立了无产阶级专政的社会主义国家,并基本上完成了所有制方面的社会主义革命,所取得的巨大成果。

但是,“各尽所能,按劳分配”这个原则,“就产品‘按劳动’分配这一点说,‘资产阶级法权’仍然占着统治地位”。怎样理解在按劳分配中还存在着资产阶级法权呢?我们知道,在按劳分配原则下,每个劳动者向社会提供一定的劳动量,在作了各项必要的社会扣除以后,再从社会领取与他提供的劳动量相等的一份消费品。劳动者以一种形式给予社会的劳动量,又以另一种形式领了回来。这实际上是一种形式的一定量的劳动和另一种形式的同量劳动相交换。因而,按劳分配所“通行的就是调节商品交换(就它是等价的交换而言)的同一原则”。在这里,每个劳动者的报酬都以劳动这同一尺度来计量,等量劳动领取等量报酬,这是平等的权利。按劳分配所体现的这种平等权利,却是“默认不同等的个人天赋,因而也就默认不同等的工作能力是天然特权”。由于各个劳动者的体力和智力是有差别的,他们在同一时期内能为社会提供的劳动量是不等的;各个劳动者的家庭负担情况也是各不相同的,因而把同一的尺度应用到实际情况各不相同的人身上,就必然出现事实上的不平等,出现富裕程度或生活水平上的差别。例如,一个身体较强和技术比较熟练的人,较之一个身体较弱和技术不熟练的人,能在同一时间内向社会提供较多的劳动,从而获得较多的收入。一个赡养人口少的人比一个赡养人口多的人,即使收入一样多,分摊到每个人口身上的实际收入却要多些。至于那些收入多而赡养人口少的人,比那些收入少而赡养人口多的人,在生活富裕程度上的差别就更为明显了。由此可见按劳分配的平等权利,是形式上平等而事实上是不平等的,按照原则仍然是资产阶级法权。正是在这个意义上,按劳分配“跟旧社会没有多少差别”。

通过上面的分析,我们对“各尽所能,按劳分配”的原则可以有一个全面的认识:它既是社会主义的原则,又还存在着资产阶级法权。

在社会主义历史阶段,“各尽所能,按劳分配”的原则,具有一定的历史作用。

实行这个原则,要求人们都各尽所能地参加社会劳动,不劳动者不得食,有利于加强劳动纪律,监督那些轻视、鄙视劳动的寄生虫和懒汉,反对剥削阶级的“好逸恶劳”、“不劳而获”等腐朽思想。

实行这个原则,在分配方面适当地照顾到劳动者向社会提供的劳动的差别,有利于正确处理劳动人民内部在分配方面的矛盾,更好地发挥劳动者的社会主义积极性。因而,正确地实行“各尽所能,按劳分配”的原则,可以促进社会主义经济事业的发展。

但与此同时,我们必须充分地认识到,按劳分配中存在的资产阶级法权,是滋生资本主义的土壤和条件。这是因为,一方面,按劳分配中通行的是与商品等价交换同一的原则,存在着收入和生活上的事实上不平等,这就会使一些人用商品买卖的关系来对待劳动和报酬的关系,产生“给多少钱、干多少活”的雇佣观点;有些人就会比级别、争地位、闹待遇,滋长追求名利的资产阶级思想。阶级敌人也正是竭力利用按劳分配中的资产阶级法权,腐蚀干部和群众的思想,诱使一些人去追名逐利,来破坏社会主义和反对无产阶级专政。另一方面,党内走资本主义道路的当权派打着“按劳分配”的旗号,大搞“物质刺激”、“奖金挂帅”,扩大、强化分配方面的资产阶级法权,这就必然会引起贫富悬殊、两极分化的出现,少数人暴发为新的资产阶级分子,而大多数人日益贫困,重新沦为雇佣奴隶,从而导致人们相互关系的改变和公有制经济的瓦解。

对于按劳分配如何认识,采取什么态度,这是一个关系到能否巩固无产阶级专政,防止资本主义复辟的重大问题。在这个问题上,存在着马克思主义同修正主义两条路线的激烈斗争。修正主义者把按劳分配绝对化,竭力扩大、强化资产阶级法权及其所带来的那一部分不平等,用物质刺激代替按劳分配,以此达到他们复辟资本主义的罪恶目的。苏修叛徒集团在苏联全面复辟资本主义,同他们扩大分配方面的资产阶级法权,从而腐蚀人们的思想灵魂,培植一小撮特权阶层,是密切相关的。刘少奇、林彪一类大搞“物质刺激”、“奖金挂帅”,也正是妄图利用资产阶级法权在我国复辟资本主义。

马克思主义认为,在社会主义阶段不可避免地还要实行按劳分配,但它毕竟不是共产主义的东西,而是同旧社会的痕迹相联系的,是一种“弊病”。因而,不能把按劳分配绝对化、凝固化,要在实行按劳分配的同时,对其中存在的资产阶级法权必须加以限制,并逐步创造条件将来最后消灭它。毛主席在一九五八年就明确地提出了要限制资产阶级法权的问题,在一九七四年底关于理论问题的重要指示中再一次指出:“这只能在无产阶级专政下加以限制”。

在社会主义历史阶段,如果看不到按劳分配中存在的资产阶级法权的消极作用,不加限制,甚至予以扩大和强化,就要犯极大的错误。我们必须在毛主席的革命路线指引下,以阶级斗争为纲,坚持党的基本路线,坚持无产阶级政治挂帅,正确地实行“各尽所能,按劳分配”的原则。

\subsection{工资制和工分制是个人消费品分配的两种形式}

在社会主义社会,采用什么具体形式来正确地实行“各尽所能,按劳分配”的原则,进行个人消费品的分配,是一个非常重要的问题。

现阶段,社会主义全民所有制经济分配个人消费品的具体形式,主要是工资制。

工资的性质,是由生产关系的性质所决定的。社会主义制度下的工资同资本主义制度下的工资,有着本质的区别。在资本主义制度下,劳动力是商品,工人所得到的极其微薄的工资,是工人出卖的劳动力的价值或价格。同时,由于大量失业工人的存在,资本家总是极力把在业工人的工资压低到劳动力价值之下。资本家付给工人很少的一点工资,维持劳动力的生产和再生产,是服从他们继续剥削工人的需要的。因此资本主义制度下的工资,体现着资本家和工人之间的雇佣被雇佣、剥削被剥削的关系。而在社会主义制度下,工人是国家和企业的主人,劳动力不再是商品,工资也不再是劳动力的价值或价格,而是国家根据“各尽所能,按劳分配”的原则,借助于货币对劳动者分配个人消费品的一种形式。因此,社会主义制度下的工资是体现着国家与职工在根本利益一致基础上的分配关系。

社会主义制度下的工资,主要有计时工资和计件工资两种形式。计时工资是以劳动时间作为计算劳动报酬的单位,即在一定时间内,根据所评定的工资等级,发给固定工资。计件工资是以劳动产品作为计算劳动报酬的单位,即根据劳动者完成的合格产品件数,按照一定的单价付给工资。

采用哪种具体工资形式,要看客观条件,要看是否有利于促进生产的发展和劳动生产率的提高,是否有利于加强职工内部的团结和促进职工的思想革命化,特别是要从巩固无产阶级专政、限制按劳分配中的资产阶级法权的高度来对待这个问题。由于计件工资把劳动与报酬联系得更加紧密、更加直接,因而它所体现的资产阶级法权也就更为突出。实行计件工资这种形式比较容易使劳动者产生片面追求产量、忽视提高产品质量和节约物资消耗的倾向,滋长斤斤计较的个人主义思想,从而不利于工人内部的团结和技术革新运动的深入开展。在我国,随着社会主义革命的逐步深入和社会主义建设的迅速发展,采用的工资具体形式有所改变。在一九五八年大跃进中,绝大多数企业取消了计件工资,改行计时工资;在社会主义教育运动中,一九六四年把名目繁多的奖金改为综合奖;一九六六年又取消综合奖,改为附加工资。这种变化,正是随着客观条件的变化,职工政治思想觉悟的提高,对分配方面存在的资产阶级法权进行限制的体现。

当前,我国国营企业工资的主要形式是计时工资。计时工资的工资制度也不平等,因为计时工资作为实现按劳分配的具体形式,它仍然以承认不同劳动者在劳动上的差别为前提,这就必然存在着等级差别,存在着形式上平等而事实上不平等的资产阶级法权。因此,必须加以限制,注意防止差别高低悬殊,并创造条件,逐步地加以缩小以至消灭。

社会主义集体所有制经济的个人消费品分配,也要实行“各尽所能,按劳分配”的原则,但同社会主义全民所有制经济相比,还具有不同的特点。

我们知道,社会主义全民所有制经济的生产资料和产品归国家所有,工资标准是由国家统一规定的,职工的工资收入不受本企业生产经营状况的影响。但是,在我国农村人民公社的集体所有制经济中,基本生产资料和产品归各个集体所有,社员的劳动报酬是同自己所在集体的生产和收入水平紧密联系的。在现阶段一般实行“三级所有,队为基础”的情况下,社员所在生产队的收入水平较高,他们的劳动报酬水平也就较高;反之,所在生产队的收入水平较低,劳动报酬水平也就较低。这种情况,说明了农村集体经济在分配方面的资产阶级法权,较之全民所有制更为多一些。这正表明了集体所有制是一种公有化程度比较低的社会主义公有制。当然,现阶段我们还必须承认由各个集体经济收入水平不同而引起的社员在劳动报酬上的这种差别,不能搞“一平二调”,但也应通过使收入水平较高的生产队多增加一些集体积累,以及积极帮助后进的生产队发展生产、增加收入的办法,来逐步缩小各个生产队社员之间在劳动报酬上的这种差别。

当前,我国农村人民公社集体所有制经济的个人消费品分配,主要采用工分制,用货币和实物来支付。工分是衡量社员在生产队里劳动消耗的尺度,同时又是计算社员劳动报酬的尺度。一个社员从生产队里取得收入的多少,不仅取决于他所做的工分的多少,而且还取决于工分值的高低。工分值是根据这个生产队全年收益并扣除一定积累以后确定的,因而它不是固定不变的,而是随着生产和收入的变化而变化的。农村人民公社集体经济采用工分制的分配形式,是由生产资料集体所有制的性质和农业生产的特点所决定的,它有利于调动社员群众的社会主义积极性。

正确地实行工分制,必须搞好劳动计酬。目前,在劳动计酬的方法上,一般采用评工记分和定额记分相结合的方法。不论实行那种具体办法,都必须坚持无产阶级政治挂帅,不断提高社员为革命种田的觉悟,限制分配方面的资产阶级法权,同资本主义倾向作不懈的斗争。但是,也要坚持实行“各尽所能,按劳分配”的原则,适当地体现不同社员在劳动报酬上的合理的差别。

\section{坚持无产阶级政治挂帅,正确实行“各尽所能,按劳分配”的原则}

\subsection{加强政治思想教育,培养共产主义劳动态度}

在社会主义社会,始终存在着两个阶级、两条道路、两条路线的激烈斗争,存在着资本主义复辟的危险性。而按劳分配的原则,并没有摆脱资产阶级法权的狭隘眼界,是滋生资本主义的土壤。它同“物质刺激”虽然有原则的区别,但却没有不可逾越的鸿沟。关键在于路线,在于用什么思想去指导实行按劳分配。如果脱离无产阶级政治挂帅,放松思想和政治路线方面的教育,就可能使一些劳动者迷失方向,斤斤计较个人利益,滋长资产阶级思想;一些企业的领导人也就会单纯地把多劳多得作为调动群众生产积极性的诱饵,而滑到“物质刺激”的邪路上去。党内走资本主义道路的当权派正是用“物质刺激”偷换按劳分配,鼓吹和推行“物质刺激”这种修正主义路线,阴谋复辟资本主义。因此,实行“各尽所能,按劳分配”的原则,必须以阶级斗争为纲,坚持无产阶级政治挂帅,加强政治思想教育,批判资产阶级法权思想,培养和发扬群众的共产主义劳动态度。

什么是共产主义劳动呢?“共产主义劳动,从比较狭窄和比较严格的意义上说,是一种为社会进行的无报酬的劳动,这种劳动不是为了覆行一定的义务、不是为了享有取得某种产员的权利、不是按照事先规定的法定定额进行的劳动,而是自愿的劳动,是无定额的劳动,是不指望报酬、没有报酬条件的劳动”。

加强政治思想教育,培养共产主义劳动态度,就是要使广大劳动群众充分认识自己是国家和企业的主人,使他们立足本职,胸怀全局。把日常的生产劳动同加速社会主义革命和社会主义建设联系起来,同支援世界革命、履行国际主义义务联系起来,同实现共产主义的伟大目标联系起来,为中国革命和世界革命尽自己的能力,贡献自己的最大力量。

加强政治思想教育,培养共产主义劳动态度,就是要使广大劳动群众同传统的所有制和私有观念实行最彻底的决裂,打破那种“冷酷地斤斤计较,不愿比别人多做半小时工作,不愿比别人少得一点报酬的狭隘眼界”。正确地认识国家、集体和个人三者之间的关系,自觉地使个人利益服从全局利益、眼前利益服从长远利益。发扬共产主义精神,提倡讲贡献、讲干劲、讲团结、讲进步,正确地对待个人生活待遇。

加强政治思想教育,培养共产主义劳动态度,就是要坚持党的基本路线,彻底批判“物质刺激”,肃清反革命修正主义路线的流毒。苏修叛徒集团用“物质刺激”偷换按劳分配,大肆鼓吹“必须广泛地采用物质刺激”,胡说什么“更大的劳动数量,更好的劳动质量,即更熟练、更紧张、更重要或更负责的工作,要用更高的报酬来刺激”,“各项经济刺激措施”是提高劳动生产率的“最重要杠杆”。刘少奇、林彪一类也疯狂推行反革命修正主义路线,反对无产阶级政治挂帅,大搞“物质刺激”、“奖金挂帅”,胡说什么“不加点钱,生产积极性不高。”,“干劲是从物质刺激中来的”,甚至搬出了孔老二“小人怀惠”、“小人喻于利”的谬论作为根据,竭力鼓吹“利己多欲乃规律”,对无产阶级和广大劳动人民极尽其诬蔑之能事。从苏修叛徒集团到刘少奇、林彪之流,他们所鼓吹的“物质刺激”,就是资产阶级的“金钱万能”论。“做事就是为了拿钱,——这是资本主义世界的道德。”“物质刺激”所调动的“积极性”,只能是资产阶级个人主义为名为利的积极性,发展资本主义的积极性。调动了这种“积极性”,势必腐蚀劳动群众的思想灵魂,滋长反革命的经济主义,瓦解社会主义经济基础,导致资本主义的复辟。我们发展社会主义生产,绝不能靠“物质刺激”,而是靠战无不胜的马克思主义、列宁主义、毛泽东思想。正如毛主席指出的:“代表先进阶级的正确思想,一旦被群众掌握,就会变成改造社会、改造世界的物质力量”。用马克思主义、列宁主义、毛泽东思想武装人们的头脑,就能最充分地调动亿万群众的社会主义积极性,促进社会主义建设高速度地向前发展。

坚持无产阶级政治挂帅,加强政治思想教育,培养共产主义劳动态度,这是对按劳分配体现的资产阶级法权加以限制的根本措施。只有这样,才能做好当前的社会主义的按劳分配工作,也才能为将来实现共产主义的按需分配创造条件。因而,那种把进行共产主义思想教育同实行“各尽所能,按劳分配”原则对立起来的认识,是错误的。当然,也不能用共产主义思想教育去代替“各尽所能,按劳分配”的社会主义原则。我们在进行共产主义思想教育的同时,必须严肃认真地执行党的实行社会主义分配原则的现行经济政策。

\subsection{要反对绝对平均主义,更要防止高低悬殊}

实行“各尽所能,按劳分配”的原则,承认不同劳动者在劳动上的差别,就会在劳动报酬上保持一定的差别。这在社会主义历史阶段是不可避免的。而绝对平均主义却要求个人消费品的分配必须平均、划一,不问劳动的数量和质量,不问对国家贡献的大小,不问工作需要,一律付给平均的报酬,只有这样才算平等。这种绝对平均主义是小资产阶级的虚伪的平等观在分配问题上的表现,是手工业和小农经济的产物,同社会主义和共产主义毫无共同之处。毛主席指出:“绝对平均主义不但在资本主义没有消灭的时期,只是农民小资产者的一种幻想;就是在社会主义时期,物质的分配也要按照‘各尽所能按劳取酬’的原则和工作的需要,决无所谓绝对的平均”。因此,实行“各尽所能,按劳分配”的原则,要反对绝对平均主义。

但是,我们也要反对在劳动报酬上扩大差别,防止高低悬殊。这是因为,在社会主义阶段,虽然在劳动报酬上保持一定的差别是不可避免的,但由于阶级斗争和三大差别的存在,如果对这种差别不是加以限制,并逐步地缩小它,而是去扩大它,造成分配上高低悬殊,就势必引起两极分化,资本主义和资产阶级就会很快地产生和发展起来。

马克思主义者历来反对劳动报酬上的高低悬殊,主张合理地逐步缩小差别。马克思在总结巴黎公社这世界上第一个无产阶级政权的经验时,充分地肯定了巴黎公社采取的“从公社委员起,自上至下一切公职人员,都只应领取相当于工人工资的薪金”的原则。列宁非常重视巴黎公社的经验,强调在工资政策中必须坚持巴黎公社的分配原则,并对在苏维埃政权建立初期,为了利用旧社会遗留下来的资产阶级知识分子,而不得不实行的高薪制作了深刻的批判,指出:“这个办法是一种妥协,是离开巴黎公社和任何无产阶级政权的原则的”,“高额薪金的腐化作用既要影响到苏维埃政权……也要影响到工人群众,这是无可争辩的”。我们一切工作干部,不论职位高低,都是人民的勤务员,绝不要实行对少数人的高薪制度,应当合理地逐步缩小而不应当扩大工作人员同人民群众之间的个人收入的差距,防止一切工作人员利用职权享受任何特权。

苏修叛徒集团完全背叛了社会主义的分配原则。他们打着“按劳分配”的旗号,竭力从分配方面扩大差别,通过高工资、高奖金、高稿酬以及名目繁多的各种个人津贴,培植了一个特权阶层和一批精神贵族,大肆剥削广大工人农民的劳动成果,加剧了两极分化,加快了在苏联全面复辟资本主义的进程。这是一个严重的历史教训。

因此,正确地实行“各尽所能,按劳分配”的原则,就必须对按劳分配所体现的资产阶级法权加以限制,劳动报酬上的差别决不能过大,并要逐步地缩小差别,坚决防止高低悬殊。

我们不仅要防止国家工作人员和人民群众之间、工人内部之间在分配上的高低悬殊,逐步缩小他们在个人收入上的差距,而且对于工人和农民之间在收入上的差距,也应逐步缩小而不应扩大。我们知道,当前工人的劳动生产率比农民高得多,而农民的生活费用比城市工人又省得多,因而在历史上长期形成的工农之间、城乡之间的差别没有消除以前,工人和农民在收入上存在某种适当的差距是不可避免的。但是,这种差距决不能扩大,而是应当随着工农业生产的发展,进行适当的调整,逐步地缩小工农在收入上的差距,以巩固工农联盟。

在社会主义社会,消费品除了按照“各尽所能,按劳分配”的原则分配给劳动者以外,还有一部分是通过国家和集体举办的集体福利事业供给劳动人民享用的。其中如国家对职工实行的免费医疗等,已具有共产主义按需分配的因素。积极地扶植和发展这种共产主义的因素,是限制分配方面资产阶级法权的重要措施。我们要随着生产的发展,逐步地扩大社会集体福利事业,扩大它在个人消费品分配中的比重,以增加共产主义按需分配的因素。

社会主义社会个人消费品的分配方式,不是一成不变的。恩格斯在一八九零年评论当时德国党内关于未来社会中的产品分配的辩论时,指出:“分配方式本质上毕竟要取决于可分配的产员的数量,而这个数量当然随着生产和社会组织的进步而改变,从而分配方式也应当改变。但是,在所有参加辩论的人看来,‘社会主义社会’并不是不断改变、不断进步的东西,而是稳定的、一成不变的东西,所以它应当也有个一成不变的分配方式。但是,合理的辩论只能是:

 \begin{enumerate}

\item 设法发现将来由以开始的分配方式,
\item 尽力找出进一步的发展将循以进行的总方向。”这就是说,社会主义社会个人消费品的分配方式将由它的“各尽所能,按劳分配”,逐步地向共产主义的“各尽所能,按需分配”过渡。这个过渡,需要经历整个社会主义社会的历史阶段,是一个对分配方面的资产阶级法权进行批判,限制、逐步缩小它的作用范围,并在最后加以消灭的过程,也是一个极大地发展社会生产力的过程。随着社会生产力的发展、三大差别的逐步缩小以及人们共产主义觉悟的不断提高,按需分配因素的逐步增加和按劳分配因素的逐渐减少,社会必将在自己的旗帜上写上:“各尽所能,按需分配”。

 \end{enumerate}

\chapter{社会主义国家的对外经济关系}

对外经济关系,是社会主义国家对外关系的一个重要方面。世界各国人民的革命斗争,从来都是相互支持、相互援助的。已经取得革命胜利的社会主义国家,在无产阶级革命外交路线的指导下,发展对外经济关系,对于支援世界革命和加速国内社会主义建设具有重大意义。我们要进一步树立起支援世界革命是无产阶级应尽的国际主义义务的思想,提高执行毛主席革命外交路线的自觉性,为人类的进步事业作出较大的贡献。

\section{社会主义国家的对外经济关系必须以无产阶级革命外交路线为指导}

\subsection{对外经济关系是社会主义国家对外关系的一个重要方面}

马克思主义认为:“社会主义不能在所有国家内同时获得胜利。它将首先在一个或者几个国家中获得胜利,而其余的国家在一段时期内将仍然是资产阶级的或者资产阶级以前时期的国家。”我们的时代,是帝国主义和无产阶级革命的时代。已经取得革命胜利的社会主义国家,除了要和社会制度相同的社会主义国家发生经济联系外,还要和其他不同社会制度的国家发生一定的经济联系。

社会主义国家的对外经济关系,是社会主义国家对外关系的一个重要方面。经济应该服从政治,对外经济活动必须服从无产阶级革命外交路线。

“谁是我们的敌人?谁是我们的朋友?这个问题是革命的首要问题。”社会主义国家的对外关系是极为复杂的,必须首先明确国际范围内的阶级阵线,分清在国际上谁是我们的敌人,谁是我们的朋友,才能制定出符合无产阶级利益的革命外交路线。

当前国际形势的特点是天下大乱。在“天下大乱”的形势下,世界上各种政治力量经过长期的较量和斗争,发生了大分化和大改组。一系列亚非拉国家纷纷取得独立,在国际事务中起着越来越大的作用。由于苏联蜕变为社会帝国主义,在战后一个时期内曾经存在的社会主义阵营,现已不复存在。由于资本主义发展不平衡规律的作用,西方帝国主义集团,也已四分五裂。从国际关系的变化看,当今世界实际上存在着互相联系又互相矛盾着的三个世界。苏联、美国是第一世界。亚非拉和其他地区的发展中国家是第三世界。处于第一世界和第三世界之间的发达国家是第二世界。

第一世界代表当前世界上最反动的政治力量,它们奉行霸权主义,侵略、控制、颠覆和掠夺别的国家。第二世界是政治上的中间派,它们受第一世界的压迫、欺负,但有的国家还在压迫和欺负第三世界。第三世界是一股新兴的政治力量,它们受压迫、受欺负最深,所以反抗两霸最强烈。“三个世界”的划分,表明了苏美两霸是包括苏美两国人民在内的全世界人民的共同敌人,第二世界是可以争取、利用的力量,第三世界是反帝、反殖、反霸的主力军,从而使我们明确了国际上的阶级阵线,可以制定出符合无产阶级利益的对外政策,更好地开展对外活动。

中国是一个发展中的社会主义国家,属于第三世界。在国际事务中,我们要坚持无产阶级国际主义。中国永远不做超级大国。我们要同社会主义国家、同一切被压追人民和被压迫民族加强团结,互相支援;在互相尊重主权和领土完整、互不侵犯、互不干涉内政、平等互利、和平共处五项原则的基础上,争取和社会制度不同的国家和平共处,反对帝国主义、社会帝国主义的侵略政策和战争政策,反对超级大国的霸权主义。这些基本原则也是我国对外经济关系的出发点。


\subsection{处理对外经济关系要分别情况、区别对待}

社会主义国家在处理对外经济关系时,要在无产阶级革命外交路线的指导下,对各个国家要根据社会制度的不同情况,做到区别对待。

社会主义国家之间的关系,是人类历史上一种完全新型的国际关系。毛主席指出:“自从有历史以来,任何国家间的关系,都不可能象社会主义国家间这样休戚与共,这样互相尊重和互相信任,这样互相援助和互相鼓舞。这是因为社会主义国家是完全新型的国家,是推翻了剥削阶级而由劳动人民掌握权力的国家。在这些国家间的相互关系中,实现着国际主义和爱国主义相统一的原则。共同的利益和共同的理想把我们紧紧地联结在一起。”社会主义国家之间,坚持无产阶级国际主义的原则,在独立自主、完全平等的基础上,积极发展经济上的相互支持、相互援助,加快各国的社会主义建设,增强反对帝国主义和社会帝国主义的力量,为实现共产主义的伟大理想而共同奋斗。

社会主义国家和第三世界中的民族主义国家,尽管社会制度不同,但都有遭受帝国主义侵略的共同经历,都有在政治上、经济上反对两个超级大国的霸权主义,维护民族独立和发展民族经济的共同愿望,都有在坚持五项原则基础上建立和发展经济合作的共同要求。“不管被压迫民族中间参加革命的阶级、党派或个人,是何种的阶级、党派或个人,又不管他们意识着这一点与否,他们主观上了解了这一点与否,只要他们反对帝国主义,他们的革命,就成了无产阶级社会主义世界革命的一部分,他们就成了无产阶级社会主义世界革命的同盟军。”社会主义国家同他们在反帝、反殖、反霸的共同斗争中,在和平共处五项原则的基础上,帮助他们独立自主地发展民族经济,维护他们的民族独立和国家主权,就会使世界革命的力量更加壮大,促进人类的共同进步。

社会主义国家在处理同第二世界发达国家之间的经济关系时,既要看到这些发达国家在不同程度上受着这个或那个超级大国的控制、威胁或欺负,有想摆脱苏美两霸的干涉、压迫的一面;同时还要看到他们又压迫、剥削别人,对两霸还有依赖的一面。在当前反帝、反殖、反霸的潮流中,我们要争取联合第二世界,又联合又斗争,联合他们的积极面,斗争他们的消极面,以斗争求联合。因此,社会主义国家同第二世界发达国家,虽然在社会制度方面是根本对立的,但在和平共处五项原则的基础上发展经济往来,有利于促进反对超级大国的国际统一战线的广泛发展。

苏美两个超级大国是当代最大的国际压迫者和剥削者,是新的世界战争的策源地。他们以大欺小、以强凌弱、以富压贫,对别国在政治上进行控制、颠覆、干涉和侵略,在经济上进行剥削和掠夺,那个搞“声东击西”、玩弄“缓和”骗局最起劲的苏联社会帝国主义是今天最危险的战争策源地。社会主义国家对第一世界两个超级大国,只有进行针锋相对的斗争,坚持独立自主、自力更生,才能迫使两个超级大国在一定时期内,不得不在国家之间建立某种程度的和平共处关系,做些生意,组织一些人民之间的联系和交流,而决不能对他们抱有任何不切实际的想法。


\section{社会主义国家的对外经济援助}

\subsection{支援世界革命是无产阶级应尽的国际主义义务}

对外经济援助,是社会主义国家对外经济关系的一个重要组成部分。

马克思主义认为,无产阶级的革命事业从来就是国际性的事业。“资本是一种国际力量,要想彻底战胜它,就需要国际范围内的工人共同行动起来。”在世界范围内,各国资产阶级虽然为摄取剩余价值而互相竞争和冲突,但他们又总是互相勾结在一起,共同压迫各国的无产阶级。因此,无产阶级的革命对象,不能仅是本国的资产阶级,而且也包括国际的资产阶级。各国的无产阶级为彻底战胜资本主义,实现共产主义的祟高理想,也就必须互相帮助,互相支援,共同战斗。早在一百多年前,马克思和恩格斯在《共产党宣言》中指出:“在各国无产者的斗争中,共产党人强调和坚持整个无产阶级的不分民族的共同利益”,无产阶级国际联合的努力“是无产阶级获得解放的首要条件之一”,“全世界无产者,联合起来”。毛主席也指出:“在帝国主义存在的时代,任何国家的真正的人民革命,如果没有国际革命力量在各种不同方式上的援助,要取得自己的胜利是不可能的。胜利了,要巩固,也是不可能的。”无产阶级国际主义的思想,是无产阶级彻底革命、把世界革命进行到底的思想。无产阶级只有解放全人类,自己才能最后地得到解放。

世界各国人民的革命斗争,从来都是相互支持、相互援助的。一个国家的无产阶级能够夺取政权,建立无产阶级专政,胜利地进行社会主义革命和社会主义建设,是同国际无产阶级和各国人民的同情和支持分不开的。无产阶级取得革命胜利的社会主义国家,既不可抹杀国际无产阶级和各国人民对自己的援助,也不可片面夸大自己对别人的支持。必须坚决反对资产阶级的沙文主义和民族利己主义,坚持无产阶级的国际主义。必须把支持和援助全世界无产阶级和被压迫人民、被压迫民族的斗争,当作社会主义国家应尽的国际主义义务,而决不能看成是一种额外的负担。

坚持不坚持无产阶级国际主义原理,履行不履行对世界各国人民的正义斗争的支持和援助的义务,历来是马克思主义同修正主义斗争的一个焦点。一切修正主义者无不反对无产阶级国际主义,而奉行社会沙文主义和民族利己主义。苏修叛徒集团彻底背叛了无产阶级国际主义,疯狂地推行侵略扩张政策,拼命地同美帝国主义争夺世界霸权,已成为世界人民最危险的敌人。我国的刘少奇一类曾抛出了一条“三和一少”的反革命修正主义路线,叫嚷对帝国主义、社会帝国主义、各国反动派要“和”,对各国人民的正义斗争的授助要“少”,大肆散布民族利已主义,妄图阻挠我国对世界革命人民正义斗争的支持,向帝国主义、社会帝国主义屈膝投降。

毛主席领导全党和全国人民同这伙无产阶级叛徒及其修正主义路线进行了坚决的斗争,捍卫了无产阶级国际主义原则。毛主席指出:“已经获得革命胜利的人民,应该援助正在争取解放的人民的斗争,这是我们的国际主义的义务。”我国人民在毛主席的英明领导下,高举“全世界无产者,联合起来”和“全世界无产者和被压迫民族联合起来”的战斗旗炽,一贯反对帝国主义和社会帝国主义的侵略政策和战争政策,坚决支持和援助各国人民的正义斗争,在力所能及的条件下,忠实地履行了自己应尽的国际主义义务。

\subsection{社会主义国家的对外经济援助是真正的援助}

社会主义国家对外提供经济援助,是履行自己应尽的国际主义义务。它的基本出发点是:支援社会主义兄弟国家进行社会主义建设,以增强社会主义国家的经济力量;支援第三世界发展中国家独立自主、自力更生地发展民族经济,以巩固政治独立;支援世界无产阶级和被压迫人民、被压迫民族的革命斗争,以取得政治独立,从而促进世界革命的发展。这种经济援助的最终目的,是要帮助受援国主要依靠本国人民的力量和智慧,掌握本国的经济命脉,充分利用本国的资源,逐步地发展本国的民族经济,而不是造成受援国对自己的依赖。

因此,社会主义国家的对外经济授助,应该有下述的特点:

社会主义国家对外提供经济援助,要严格尊重受援国的主权,不附带任何政治、军事条件,不要求任何特权或牟取暴利,不把单方面的意见强加于人。

社会主义国家对外提供经济援助,要根据受援国的具体情况,积极协助它们发展本国需要的多种经营的农业,协助它们发展一些投资少、建设时间短、收效快的轻工业和建立从原料到成品、包括重工业在内的完整的工业部门,逐步改变主要农产品和生活必需品依赖进口的状况;积极协助受援国培养本国的技术人员和管理干部,积累生产建设经验,为发展民族经济创造条件。

社会主义国家对外提供经济援助,贷款是无息或低息的,必要时可以延期还本付息,甚至减免债务负担,从不逼债,更不以逼债作为施加政治压力的手段;技术转让要实用、有效、廉价和方便;派驻受援国的专家和人员向受援国人民认真传授技术,尊重受援国的法令和民族习惯,而不要求特殊待遇,更不得进行非法活动。

由此可见,在无产阶级国际主义原则指导下进行的社会主义国家对外经济援助,是人类历史上从未有过的真正的援助。

社会主义国家的这种真正国际主义的对外经济援助,与帝国主义、社会帝国主义的假借“援助”之名,行控制、掠夺别国之实的所谓“援助”,有着本质的区别。

美帝国主义的所谓经济“援助”,是对受“援”国进行政治控制和推行其反革命全球战略的重要手段。美帝通过经济“援助”,控制受“授”国的内政外交,迫使受“援”国参加美帝把持的政治军事联盟,为美帝提供军事基地。如果受“援”国稍有反抗,就停止“援助”,施加种种压力,甚至进行颠覆活动,扶植新的傀儡政权。美帝国主义的所谓“经济援助”,也是对受“援”国进行经济掠夺的重要工具。美帝通过经济“援助”,要求受“援”国必须为美国垄断资本大开方便之门,给予美国投资以优惠的待遇;必须把开采出来的原科廉价卖给美国,有的还要把原料开采权让给美国垄断组织,由它们直接进行搜括;必须购买美国商品和剩余物资,而美国商品的价格又要高于国际市场的价格;必须接受美国对经济发展的控制,经济计划要由美国审核批准,“援”款的使用要由美国决定,接受美国的监督。同时,美国的三分之二以上的贷款的利息超过年利五厘,最高的达八厘,造成受“援”国偿还本息的沉重负但。可见,美帝就是利用所谓“援助”,控制受“援”国的经济命脉,掠夺受“援”国的自然资源,把受“援”国变成自己的商品销售市场、原料供应基地和资本输出的场所,对受“援”国进行残酷的剥削和掠夺,使其成为美帝国主义的经济附庸。

苏联社会帝国主义的对外经济关系,是打着社会主义的旗号,干着帝国主义的勾当。他们竭力鼓吹“国际分工”的谬论,胡说什么“国际分工”可以“提高社会生产效果,促进所有社会主义国家的经济和劳动人民福利的高速度增长”,是符合“无产阶级国际主义”的。恩格斯曾对英国资产阶级主张的“国际分工”作过深刻的揭露,指出这种谬论的实质在于“英国应当成为‘世界工厂’;其他一切国家对于英国应当同爱尔兰一样,成为英国工业品的销售市场,同时又供给它原科和粮食”,也就是要使其他国家成为英国的经济附庸。苏修的所谓“国际分工”,同资本主义的“国际分工”完全是一路货色,是地地道道的帝国主义奴役和掠夺的谬论。苏修利用“经互会”这个工具,在自己那个“大家庭”中,以所谓“经济合作”和“国际分工”的名义,推行什么“生产专业化”和“经济一体化”,竭力限制“经互会”其他成员国发展燃料和原料工业,从而迫使它们以高价进口苏修的燃料和原料;强迫成员国改组自己的工业生产结构,按照苏修规定购品种、规格、型号生产产品,成为苏修的附属加工厂或附属车间;剥夺某些成员国独立自主地发展本国工业的权利,强迫它们重点发展农业,使它们成为苏修的果菜园和畜牧场。由于苏修的残酷剥削和压榨,使所谓“大家庭”其他成员国的国民经济畸形发展,在经济上越来越成为苏修的附庸。

近年来,苏修更把“国际分工论”向第三世界推广,说什么发展中国家只有同苏修“合作”,才能“建立独立的民族经济”,公然要求发展中国家“逐步地分阶段地参加国际社会主义分工”。这是妄图把发展中国家按照“经互会”国家的模样,纳入苏修的势力范围,给这些国家重新带上殖民主义的枷锁。

苏修打着“援助”、“支持”的旗号,竭力地进行对外扩张和掠夺。苏修以“援助”、“贷款”为名,对“经互会”其他成员国高价推销陈旧设备,大搞资本输出,控制受“援”国的经济命脉,在经济上进行掠夺。

苏修这个自称是第三世界的“天然盟友”,通过经济“援助”,残酷地剥削和掠夺第三世界国家。苏修在向第三世界国家提供贷款时,规定贷款只能用于购买苏联出口的贷物,并要用输出原料来偿还债务。这就使受“援”的第三世界国家成了苏修的商品倾销市场和原料来源地。苏修通过经济“援助”,向第三世界国家大量倾销陈旧落后的机器设备,并掠夺了大量的战略原料和农产品,大发横财。苏修还以“援助”的形式,大做军火买卖,成了世界上最大的军火商。苏修还经常乘人之危,进行逼债,借以对受“援”国施加压力,侵犯受“援”国的主权,干涉受“援”国的内政,等等。可见,苏修的对外“援助”,目的是控制第三世界国家的经济和政治,竭力推行新殖民主义,同美帝国主义争夺势力范围,谋求霸权地位。

中国人民遵照毛主席关于各个方面要节约一点,多帮助人家一点的教导,积极承担了援外的国际主义义务。我国对外提供经济援助有双边和多边两种形式,其中双边援助是主要的。多边援助是我国参加联合国后的新形式,是为配合国际政治斗争,配合双边援助的辅助形式。双边援助有四种:一是成套项目援助,即帮助受援国建设工厂、矿山、医院、学校等工程项目,这是一种主要的援助形式;二是一般物资援助;三是现汇援助,即通过支付国际货币给予的援助;四是单项物资援助,这主要是军事物资的援助。军事援助是无偿的,我国永远不做军火商。

我国对外经济援助的原则是:

 \begin{enumerate}

\item 根据平等互利的原则对外提供援助,不把这种援助看作是单方面的赐予,而认为援助是相互的。
\item 在对外提供援助的时候,严格尊重受援国的主权,绝不附带任何条件,绝不要求任何特权。
\item 以无息或者低息贷款的方式提供经济援助,在需要的时候延长还款期限,以尽量减少受援国的负担。
\item 对外提供援助的目的,不是造成受援国对中国的依赖,而是帮助受援国走上自力更生、经济上独立发展的道路。
\item 帮助受援国建设的项目,力求投资少,收效快,使受援国政府能够增加收入,积累资金。
\item 提供自己所能生产的、质量最好的设备和物资,并且根据国际市场的价格议价。如果所提供的设备和物资不合乎商定的规格和质量,保证退换。
\item 对外提供任何一种技术援助的时候,保证做到使受援国的人员充分掌握这种技术。
\item 派到受援国帮助进行建设的专家,同受援国自己的专家享受同样的物质待遇,不容许有任何特殊要求和享受。

 \end{enumerate}

我国政府的八项原则,是毛主席的无产阶级革命外交路线在援外工作中的具体体现,也是我国援外工作实践的概括和总结。它充分体现了无产阶级国际主义精神,是我国对外援助工作的准则。

由于我国经济还比较落后,我们提供的物质援助是很有限的,对各国人民的支持主要还是政治上和道义上的。我们要遵照毛主席关于履行国际主义义务和“不称霸”的教导,随着我国社会主义革命和社会主义建设事业的发展,逐步改变这种力不从心的状况,对人类的进步事业做出较大的贡献。

\section{社会主义国家的对外贸易}

\subsection{社会主义国家的对外贸易是新型的对外贸易}

对外贸易属于商品流通的范畴,是一个国家同其他国家之间的商品交换。它是一种为本国统治阶级利益服务的有目的的活动。一个国家的对外贸易的性质,是由这个国家的社会制度所决定的。

在资本主义制度下,对外贸易是建立在资本主义私有制的基础上,为资产阶级所把持,是资本家残酷剥削其他国家人民,摄取高额利润的一种重要手段。资本主义发展到帝国主义阶段,对外贸易则成为垄断资本集团对外进行经济侵略,推行掠夺和扩张政策的一个重要手段。第二次世界大战后,美帝国主义依仗它的经济和军事实力,通过对外贸易扩大商品输出,倾销“剩余”物资,占领资本主义世界市场;同时,又以高额关税和其他办法,限制其他国家的商品进入美国市场。美帝还对社会主义国家采取“经济封锁”、“禁运”等政策,妄图扼杀新生的社会主义国家。显然,美帝的对外贸易是为其侵略和扩张政策服务的,是争夺世界霸权的一个重要手段。

苏联社会帝国主义的对外贸易,为官僚垄断资产阶级所把持。它是苏修对外进行经济侵略和政治控制,同美帝争夺世界霸权的重要手段。这种对外贸易,实质上就是帝国主义的对外贸易。苏修在同第三世界国家的贸易中,凭借其提供经济“援助”的债权国的地位,采用种种敲诈勒索的伎俩,掠夺第三世界国家的财富。例如,苏修以贷款名义向第三世界出口的机器设备,质量差而价格高,有的要比国际市场价格高百分之十五到二十五,而从第三世界进口的产品,价格却被压得比国际市场低百分之十到十五,有的甚至低百分之三十。

社会主义国家的对外贸易,同帝国主义、社会帝国主义的对外贸易完全不同,是一种新型的对外贸易。它是建立在社会主义公有制的基础上,为无产阶级专政的国家所掌握,用来促进国民经济的发展,加速社会主义革命和社会主义建设,支援世界革命的重要手段。

完全新型的社会主义国家对外贸易,具有明显的特点:

社会主义国家的对外贸易,是由无产阶级专政国家统制的。“人民共和国的国民经济的恢复和发展,没有对外贸易的统制政策是不可能的。……对内的节制资本和对外的统制贸易,是这个国家在经济斗争中的两个基本政策。谁要是忽视或轻视了这一点,谁就将要犯绝大的错误。”把一切对外贸易活动置于国家集中领导和统一管理下,实行国家统制,是无产阶级专政的国家有效地抵制帝国主义的经济侵略,顺利地建设社会主义,捍卫国家独立和巩固无产阶级专政的一项重要政策。如果不实行对外贸易的国家统制,而搞什么自由贸易,就不能保证进出口贸易符合社会主义建设和对外斗争的需要,国内市场就会受到冲击,社会主义建设就要遭到破坏,甚至有可能被国内外阶级敌人利用来进行颠覆活动。我国建国以后,废除了帝国主义在我国的特权,收回了被帝国主义长期霸占的海关管理权,结束了外商对我国对外贸易的操纵;同时,没收了官僚资本的对外贸易企业,对私营进出口商逐步地进行了社会主义改造,从而把对外贸易牢牢地掌握在无产阶级专政的国家手中,使它成为我国同世界许多国家和地区进行正常贸易往来,加快社会主义建设步伐和更好地配合外交斗争的一个有力工具。

社会主义国家的对外贸易,是在平等互利、互通有无的原则下进行的。作为社会主义国家对外经济关系的一个重要方面的对外贸易,必领遵循无产阶级革命外交路线,贯彻执行平等互利、互通有无的原则,要求相互尊重对方的主权和愿望,符合双方的需要和可能,价格应当公平合理。社会主义国家不在贸易中附加不平等条件,去损害对方国家的主权和利益;不强迫对方国家接受他们不需要的商品,或出售他们不愿出售的商品。社会主义国家必须遵循无产阶级国际主义原则,为帮助社会主义兄弟国家和发展中国家解决因难,应尽力供应他们急需而自己也需要的重要物资,或购买他们销售有困难而自己又不太需要的货物,社会主义国家在同资本主义国家的贸易往来中,要同他们的贸易歧视政策进行斗争,争取进出口商品的公平合理的价格。

社会主义国家的对外贸易,是在国家管理下有计划地统一地进行的。社会主义国民经济是有计划按比例发展的,对外贸易也就必须根据国家进口的需要和出口的可能,有计划地组织内外物资交流。只有这样,才能充分发挥对外贸易的作用,促进整个国民经济的高速度发展。同时,为了贯彻执行无产阶级革命外交路线,紧密配合外交斗争,有计划地组织物资的出口和进口,对外贸易就必领在中央的集中领导下,坚持对外活动的统一性。因此,集中领导,统一对外,是社会主义国家对外贸易的一条基本原则。社会主义国家的各级对外贸易部门,要在统一认识,统一政策,统一计划,统一指挥,统一行动的基础上,积极开展各项业务活动,做好对外贸易工作。

\subsection{在独立自主、自力更生的基础上发展对外贸易}

社会主义国家的经济建设,是依靠本国的人力、物力和财力,立足于国内市场,独立自主、自力更生地进行的。它同资本主义国家依赖于对外贸易,要从国外输入大量粮食、原材科和向国外倾销商品的情况是截然不同的。

社会主义国家坚持独立自主、自力更生的方针,但并不否定对外贸易对社会主义经济发展的积极作用。这是因为,世界上任何一个国家都不能生产自己所需要的一切。社会主义国家通过对外贸易,可以互通有无,促进生产的发展。一方面,进口某些必要的物资,可以调剂由于生产条件的限制和自然条件的影响所造成的某些产品或资源的暂时短缺,可以调节由于偶然因素所造成的计划安排中出现的暂时缺口;引进一些先进技术作为借鉴,可以减少摸索过程,促进国民经济的技术改造。另一方面,社会主义国家组织一部分物资出口,不仅可以解决偿付进口所需要的外汇,也有助于促进国内工农业生产的发展,增加资金积累。因而,在独立自主、自力更生的基础上发展对外贸易,有利于加速社会主义建设。

社会主义国家的对外贸易,对支援世界革命和配合外交斗争也起着积极的作用。社会主义国家通过对外贸易,可以帮助社会主义兄弟国家的经济建设迅速发展;可以支援发展中国家发展独立自主的民族经济;可以加强同其他国家人民和政府之间的了解和友谊,扩大国际反帝统一战线,孤立和打击苏美两霸。

社会主义国家在独立自主、自力更生的基础上发展对外贸易,要正确认识和处理对外贸易和国内贸易的关系、对外贸易和发展生产的关系。

对外贸易和国内贸易是同属于流通领域的两个经济部门,对外贸易是进行国外的商品交换,国内贸易则是担负国内的商品交换。社会主义国家的对外贸易和国内贸易的根本目的是一致的,都是为了满足整个社会和人民的需要服务的。它们之间是互相协作、互相配合的关系,但也存在着一些矛盾,例如在资源分配上就不免常有矛盾。在处理内外贸易的关系上,要“统筹兼顾,适当安排”,首先要从国家建设和人民生活的需要出发,立足于国内市场;同时,国外市场也很重要,不能忽视。那种不问国内情况如何,不加区别地提出“内销服从外销”,势必要破坏国内的社会主义市场,破坏工农联盟,破坏整个国民经济计划,因而是错误的。反之,那种不问对外政治经济斗争的需要,不顾国家计划,不问什么物资,片面地强凋“外贸服从内贸”,不仅不能及时地组织出口和某些必需物资进口,而且也要影响对世界革命的支援,破坏社会主义国家的国际声誉,因而也是错误的。在国家统一计划下的外贸出口任务必须积极完成。

对外贸易和发展生产的关系,也就是流通和生产的关系。对外贸易的发展,可以影响生产,但决定对外贸易发展的却是生产。社会主义生产的发展,是对外贸易发展的物质基础。社会主义国家在一般情况下,一定时期内的进口规模是取决于出口规模的。社会主义国家只有大力发展工农业生产,具有充足的出口货源,才能进口更多的必需物资,从而扩大对外贸易。所以,外贸部门必须深入生产第一线,关心生产,促进生产,才能为做好外贸工作创造条件。

坚持不坚持“独立自主、自力更生”的方针,历来是外贸工作中两条路线斗争的焦点。苏修叛徒集团恶毒攻击“独立自主、自力更生”这一马克思主义的方针,说什么是“闭关自守”、“自给自足经济”,是“单干”。刘少奇一类叫嚷什么要充当帝国主义的“红色买办”,要同帝国主义国家“多做生意”,鼓吹“外汇挂帅”,主张“一切为了外汇”,竭力贩卖买办洋奴哲学,妄图改变我国社会主义对外贸易的政治方向,把我国变成帝国主义、社会帝国主义的经济附庸。

“我们必须尽可能地首先同社会主义国家和人民民主国家做生意,同时也要同资本主义国家做生意。”建国以来,我国在毛主席的革命路线指引下,同刘少奇一类进行了坚决的斗争,在独立自主、自力更生的基础上,有区别地同“三个世界”的各国发展了贸易往来关系,促进了工农业生产发展,扩大了对外影响,支援了世界革命。

当前,国际形势一片大好。我们为了更好地履行自己应尽的无产阶级国际主义义务,支援世界革命,加速国内的社会主义建设,一定要坚决执行毛主席的革命外交路线,积极发展对外经济关系。

\chapter{共产主义必然在全世界胜利}

社会主义制度代替资本主义制度,是人类社会制度的一次伟大革命。但是,无产阶级革命的最终目的,是要在社会主义社会的基础上建立共产主义社会。从社会主义前进到共产主义,仍然是一次极其深刻的社会革命。我们必须坚持无产阶级专政下的继续革命,为在全世界实现共产主义而努力奋斗。

\section{共产主义社会的两个阶段}

\subsection{实现共产主义,是无产阶级的最高理想}

马克思主义政治经济学科学地分析了社会基本矛盾运动的客观规律,特别是分析了资本主义社会发生、发展和灭亡的规律,得出了资本主义必然灭亡和共产主义必然胜利的科学结论;揭示了无产阶级的历史使命是推翻资产阶级和一切剥削阶级,用无产阶级专政代替资产阶级专政,用社会主义战胜资本主义,彻底消灭一切阶级和阶级差别,最终实现共产主义。

什么是共产主义呢?毛主席指出:“共产主义是无产阶级的整个思想体系,同时又是一种新的社会制度。这种思想体系和社会制度,是区别于任何别的思想体系和任何别的社会制度的,是自有人类历史以来,最完全最进步最革命最合理的。”作为一种思想体系,共产主义就是以马克思列宁主义为理论基础的无产阶级思想体系。作为一种社会制度,共产主义就是彻底消灭了阶级和阶级差别的社会,是全体人民具有高度的共产主义思想觉悟和道德品质的社会,是生产力高度发展从而具有极其丰富的社会产品的社会,是实行“各尽所能,按需分配”原则的社会,是国家消亡了的社会。但是,在共产主义社会也还仍然存在着矛盾和斗争,存在着先进与落后、新与旧之间的矛盾和斗争,只是矛盾的性质已发生根本的变化罢了。

在全世界实现共产主义,这是无产阶级和广大劳动人民的最高理想,是无产阶级政党的最终目的。毛主席指出:“我们的将来纲领或最高纲领,是要将中国推进到社会主义社会和共产主义社会去的,这是确定的和毫无疑义的。我们的党的名称和我们的马克思主义的宇宙观,明确地指明了这个将来的、无限光明的、无限美炒的最高理想。”半个多世纪以来,我国人民在中国共产党的领导下,在毛主席的革命路线指引下,经过前赴后继、英勇不屈的斗争,终于取得了新民主主义革命的伟大胜利。建国以来,又乘胜前进,进一步取得了社会主义革命和社会主义建设的伟大胜利,取得了无产阶级文化大革命的伟大胜利。现在,我国人民正在中国共产党的领导下,在毛主席的革命路线指引下,以阶级斗争为纲,坚持党的基本路统,坚持无产阶级专政下的继续革命,大力开展让会主义革命和社会主义建设,不断粉碎帝国主义、社会帝国主义和一切反动派的破坏阴谋,为建设社会主义并最终实现共产主义的伟大理想而努力奋斗。

\subsection{社会主义社会和共产主义社会是同一社会形态的两个发展阶段}

无产阶级在推翻了资本主义制度以后,并不能立即进入共产主义社会。马克思指出:“在资本主义社会和共产主义社会之间,有一个从前者变为后者的革命转变时期。同这个时期相适应的也有一个政治上的过渡时期,这个时期的国家只能是无产阶级的革命专政。”列宁也曾经指出:“人类从资本主义只能直接过渡到社会主义,即过渡到生产资料公有和按劳分配。”社会主义社会,就是马克思所说的“从前者变为后者的革命转变时期”,即从资本主义社会过渡到共产主义社会的必经的历史阶段。

社会主义社会和共产主义社会是共产主义的两个发展阶段。我们通常所说的社会主义社会就是共产主义的低级阶段,而共产主义社会则是在社会主义社会的基础上发展起来的共产主义的高级阶段。

在社会主义社会,建立了生产资料的公有制,消灭了人对人的剥削制度,生产的目的是为了满足社会和人民日益增长的需要,国民经济是有计划按比例高速度地发展的,马克思列宁主义是社会的指导思想的理论基础。正如列宁指出的:“共产主义社会就是土地、工厂都是公共的,实行共同劳动,——这就是共产主义。”因此,在社会主义社会,“既然生产资料已成为公有豺产,那末‘共产主义’这个名词在这里也是可以用的,只要不忘记这还不是完全的共产主义。”

社会主义社会还不是完全成熟的共产主义,同完全成熟了的共产主义社会相比,又有着质的差别。正如马克思所说的,社会主义社会“在各方面,在经济、道德和精神方面都还带着它脱胎出来的那个旧社会的痕迹”。列宁指出:“无论在自然界或在社会中,实际生活随时随地都使我们看到新事物中有旧的残余”,马克思并不是随便地把旧社会的痕迹塞到社会主义社会中去的,“而是抓住了从资本主义脱胎出来的社会里那种在经济上和政治上不可避免的东西”。社会主义社会就是已经产生的共产主义因素和依然存在的资本主义痕迹的对立统一体。社会主义社会发展的客观进程,就是共产主义因素的不断发展壮大和资本主义因素的不断削弱以至最后消灭。因此,在社会主义社会和共产主义社会之间,就必然存在着许多重大差别。诸如,在社会主义社会里,还存在着生产资料公有化程度不同的两种公有制形式以及私有制的残余;还存在着三大差别,存在着商品制度、按劳分配,存在着资产阶级法权;生产力还没有高度发展,社会产品还没有达到极大丰富的程度;还存在着资产阶级和其他剥削阶级的思想影响以及旧社会的传统观念,等等。总之,社会主义社会和共产主义社会的根本差别,就在于社会主义社会还存在着阶级和阶级斗争,存在着社会主义同资本主义两条道路的斗争,存在着资本主义复辟的危险性,以及还存在着帝国主义和社会帝国主义进行颠覆和侵略的威胁。

由此可见,社会主义社会和共产主义社会是同一社会形态的两个不同发展阶段,前者是后者的必要准备,后者则是前者发展的必然趋势。在社会主义历史阶段,无产阶级就是要坚持无产阶级专政下的继续革命,不断地解决生产关系和生产力、上层建筑和经济基础之间的矛盾,发展共产主义因素,清除资本主义痕迹,粉碎资产阶级的一切反抗和破坏,高速度地发展社会生产力,最后消灭一切阶级和阶级差别,以实现共产主义的崇高理想。

\section{从社会主义到共产主义,必须经历深刻的社会革命}

\subsection{由社会主义向共产主义转变必须具备一定的条件}

马克思在讲到共产主义社会时,曾经概括地指出:“在共产主义社会高级阶段上,在迫使人们奴隶般地服从分工的情形已经消失,从而脑力劳动和体力劳动的对立也随之消失之后;在劳动已经不仅仅是谋生的手段,而且本身成了生活的第一需要之后;在随着个人的全面发展生产力也增长起来,而集体财富的一切源泉都充分涌流之后,——只有在那个时候,才能完全超出资产阶级法权的狭隘眼界,社会才能在自己的旗帜上写上:各尽所能,按需分配!”

根据马克思主义关于科学共产主义的理论和国际共产主义运动的经验教训,实现共产主义必须具备以下一些基本条件:

彻底消灭一切阶级和阶级差别,包括工农之间、城乡之间、脑力劳动与体力劳动之间的本质差别,以及资产阶级法权。共产主义社会是消灭一切阶级和阶级差别的社会。列宁指出:“为了完全消灭阶级,不仅要推翻剥削者即地主和资本家,不仅要废除他们的所有制,而且要废除任何生产资料私有制,要消灭城乡之间、体力劳动者和脑力劳动者之间的差别。”为了彻底消灭一切阶级和阶级差别,必须以阶级斗争为纲,坚持无产阶级专政下的继续革命,认真贯彻执行党在整个社会主义历史阶段的基本路线,加强无产阶级对资产阶级的全面专政,加速实现工业和农业的现代化,用现代化的技术装备武装农业,彻底改变农业落后于工业、乡村落后于城市的状况,实现工农群众知识化和知识分子劳动化,在较高的水平上普及文化教育,造就全面发展的共产主义劳动者。

毛主席的“五·七指示”,要求各行各业,凡是有条件的,都要以一业为主,兼学别样,都要学工、学农、学军事、学政治、学文化,都要批判资产阶级。“五·七”道路,是消灭旧社会遗留下来的三大差别以及资产阶级法权的一条光辉道路。无产阶级文化大革命以来,在毛主席革命路线的指引下,蓬蓬勃勃地生长和发展起来的社会主义新生事物,对于逐步缩小三大差别和限制资产阶级法权以至最后予以消灭,有着重要意义和作用。

实现生产资料的共产主义全民所有制,使之成为社会唯一的经济基础。共产主义生产关系的基础,只能是生产资料的单一的共产主义全民所有制。为了实现生产资料的共产主义全民所有制,不仅要消灭私有制的残余,而且要把社会主义经济的领导权牢牢地掌握在真正的马克思主义者和广大劳动群众的手里,限制所有制方面存在的资产阶级法权,巩固和发展社会主义的全民所有制和集体所有制,使社会主义集体所有制逐步地由小到大、由低级向高级地提高到社会主义全民所有制,然后再由社会主义全民所有制过渡到共产主义全民所有制。我国人民创造的人民公社,就是解决这个过渡的一种适宜的社会组织形式。

随着共产主义全民所有制的建立,人们在生产中的相互关系以及分配关系,必然要发生变化。与此相适应,社会在组织和管理生产、核算和分配劳动、调拨和交换产品的形式上,也都会发生一系列的变化。商品和货币关系也将退出历史舞台。

高度发展社会生产力,使社会产品极大丰富,实现“各尽所能,按需分配”的原则。要实现共产主义社会,必须极大地增强人们利用和控制自然界的能力,大大提高社会劳动生产率,把社会生产力的发展推进到一个崭新的水平。在马克思主义路线指引下,用革命统帅生产,高度发展社会生产力,将为生产关系的不断变革,逐步缩小以至最后消灭三大差别创造必要的物质条件;将使全体社会成员在改造自然、改造社会的过程中,不断地改造人们自身;也将造成社会产品的极大丰富,为实行“各尽所能,按需分配”的共产主义原则,提供雄厚的物质保证。因此,社会生产力的高度发展是实现共产主义社会的重要物质条件。我们热心于共产主义事业,就必须以阶级斗争为纲,“抓革命,促生产”,迅速地发展社会生产力。

培养人民群众的共产主义思想觉悟和道德品质,树立共产主义劳动态度,自党地发挥劳动积极性和创造性。高度的共产主义觉悟和道德品质,是实现共产主义社会的思想基础。这是因为,向共产主义转变不仅是人们改造自然界的革命运动,尤其重要的还是彻底改造人们的精神面貌和社会关系的伟大运动。毛主席指出:“世界到了全人类都自觉地改造自己和改造世界的时候,那就是世界的共产主义时代。”只有当全体人民都能自觉地用马克思主义、列宁主义、毛泽东思想,在长期的斗争中彻底克服旧思想和习惯势力的影响,用共产主义思想把自己武装起来的时候,才能为共产主义而忘我地、不计报酬地进行劳动;才能自觉地遵守社会纪律和公共道德,抵制错误倾向,进一步发展共产主义生产关系;才能摆脱资产阶级法权的束缚,用正确的态度对待共产主义的分配原则。因而,极大地提高人们的共产主义觉悟和道德品质是实现共产主义社会的最重要条件之一。而要创造这个条件,就必须把政治思想战线上的社会主义革命进行到底,坚持不懈地进行社会主义和共产主义教育,批判修正主义,批判资产阶级,同传统的所有制关系和传统的观念实行最彻底的决裂,直到彻底消灭资产阶级思想影响为止。

在全世界范围内彻底消灭帝国主义、社会帝国主义和一切反动派,消灭一切人剥削人的制度,使国家自行消亡。共产主义社会,是国家消亡了的社会。而无产阶级专政的国家只有在彻底消灭一切阶级和阶级差别以后,才会因完成了自己的历史使命而自行消亡。但是,社会主义是不能单独在一国取得最后胜利的。正如毛主席指出的:“按照列宁主义的观点,一个社会主义国家的最后胜利,不但需要本国无产阶级和广大人民群众的努力,而且有待于世界革命的胜利,有待于在整个地球上消灭人剥削人的制度,使整个人类都得到解放。因此,轻易地说我国革命的最后胜利,是错误的,是违反列宁主义的,也是不符合事实的。”因为,当人剥削人的制度还在世界上存在的时候,国外的资本主义阴风总会不时地吹到社会主义国家里来,社会主义国家也就不可能最终完成消灭一切阶级和阶级差别的任务,就始终存在着资本主义复辟的危险性,存在着帝国主义和社会帝国主义进行颠覆和侵略的威胁。在这种情况下,社会主义国家的无产阶级专政的对内、对外职能就不仅不能削弱,不会消亡,而且必须进一步加强。因此,这就要求首先走上社会主义道路的无产阶级,必须在本国深入进行社会主义革命、加速社会主义建设的同时,履行自己应尽的国际主义义务,大力支援世界无产阶级和革命人民反对帝国主义、现代修正主义和各国反动派的革命斗争,争取世界革命的完全胜利。只有到那时,“阶级消灭了,作为阶级斗争的工具的一切东西,政党和国家机器,将因其丧失作用,没有需要,逐步地衰亡下去,完结自己的历史使命,而走到更高级的人类社会。”

\subsection{坚持无产阶级专政下的继续革命,是实现共产主义的必由之路}

社会主义向共产主义转变,是共产主义因素逐步成长壮大和资本主义因素逐步消灭的过程。这无论在生产关系的变化方面,或是在上层建筑的变化方面,都将是一种质的飞跃,是一场极其深刻的社会革命。这个革命是在无产阶级同资产阶级之间尖锐激烈的阶级斗争中进行的,社会主义是前进到共产主义,还是倒退到资本主义,将取决于阶级斗争的结局。因此,要实现共产主义,就必须以阶级斗争为纲,坚持无产阶级专政,在一切领域、在革命发展的一切阶段对资产阶级实行全面专政,深入地进行社会主义革命,把无产阶级专政下的继续革命进行到底。

伟大领袖毛主席在反对现代修正主义的斗争中,根据马克思列宁主义的基本原理,总结了无产阶级专政的历史经验,对马克思列宁主义的无产阶级专政学说和科学共产主义的理论作了重大的发展。

毛主席阐明了社会主义社会的基本矛盾及其特点,批判了那种认为社会主义社会没有矛盾的形而上学观点,从而论证了杜会主义社会必然发展到共产主义社会的客观依据。

毛主席阐明了社会主义社会还存在着阶级和阶级斗争、存在着产生资本主义和资产阶级的土壤和条件、存在着资本主义复辟的危险性,分析了产生修正主义的社会基础和阶级根源,论证了社会主义革命的性质、对象、任务和前途。从而,提出了无产阶级专政下继续革命的理论,为我们党制定了在整个社会主义历史阶段的基本路线,有力地批判了反动的唯生产力论和阶级斗争熄灭论,并创造了无产阶级文化大革命这个在社会主义条件下无产阶级反对资产阶级和一切剥削阶级、反对党内走资本主义道路当权派的斗争形式。

毛主席概括了社会主义革命和社会主义建设的重要内容,提出了开展阶级斗争、生产斗争和科学实验三大革命运动,以胜利地进行社会主义革命和社会主义建设而逐步向共产主义转变。

毛主席肯定了我国人民所创造的人民公社这种组织形式,论证了人民公社的性质、特点及其发展趋势,为在实践中进一步继续解决向共产主义转变的具体组织形式问题奠定了牢固的理论基础。

毛主席把马克思列宁主义的不断革命论和革命发展阶段论正确地结合起来,指明了要在一定的时期内和一定条件下保持社会主义生产关系和上层建筑的基本方面的相对稳定,但又必须积极扶持社会主义新生事物,及时地对生产关系和上层建筑中不适应生产力和经济基础发展需要的某些环节进行变革和调整,把各条战线的社会主义革命不断引向深入,以推动社会主义社会向前发展。从而,阐明了从社会主义社会转变到共产主义社会的进程和步骤。

向共产主义转变,是最全面、最深刻的社会革命,是空前激烈、复杂、尖锐的阶级斗争。无产阶级要把这个革命进行到底,使社会主义前进到共产主义,关键就在于要有一条马克思主义的路线,始终以阶级斗争为纲,坚持党的基本路线,坚持对资产阶级的全面专政,坚持无产阶级专政下的继续革命。

苏修叛徒集团竭力歪曲马克思列宁主义的科学共产主义的原理,胡说什么“共产主义思想体系是世界上最人道的思想体系”,共产主义是“所有人都可以得到的,盛满了体力劳动和精神劳动产品的一盘餐”,是“一盘土豆烧牛肉的好菜”,声称“无产阶级专政在苏联已经不再是必要的”,“在二十年内我们将基本上建成共产主义社会”。刘少奇亦步亦趋,也极力歪曲和丑化共产主义,胡说什么共产主义就是“擦胭脂,抹口红,讲生活”,“面包蘸白糖”。林彪甚至搬出孔老二那一套,胡说什么共产主义就是“公产主义”,“大道之行也,天下为公”是“共产主义之原始的思想”。他们这种否认社会革命、否认阶级斗争、否认无产阶级专政的谬论,是对马克思列宁主义的无耻背叛,是对无产阶级专政学说和科学共产主义的极大歪曲。

列宁在批判叛徒考茨基的谬论时,曾明确指出,“机会主义恰巧在最主要之点不承认有阶级斗争,即不承认在资本主义向共产主义过渡的时期,在推翻资产阶级并完全消灭资产阶级的时期有阶级斗争。”“向前发展,即向共产主义发展,必须经过无产阶级专政,决不能走别的道路”。苏修叛族集团完全背叛了列宁的教导,推行了一条反革命修正主义路线,在“全面建设共产主义”的幌子下,把列宁主义的故乡、世界上第一个社会主义国家蜕变为社会帝国主义国家,全面复辟了资本主义。这就清楚地说明了他们的“共产主义”是彻头彻尾的假共产主义,是地地道道的真资本主义,是用来掩盖他们向无产阶级猖狂进攻,大搞反革命复辟活动所采取的卑劣手段。今日之苏联资本主义全面复辟的严酷现实,充分地表明了这个叛徒集团是打着共产主义的旗号,干着复辟资本主义的勾当。他们的反革命本质,已越来越被人们所认识。


\section{为在全世界实现共产主义而奋斗}

共产主义在全世界的胜利是不可避免的,共产主义一定要实现。这是社会发展的客观规律,是当代世界基本矛盾发展的必然结果。但是,阶级敌人决不会甘心失败,共产主义也不会自动到来。实现共产主义,必须依靠无产阶级和广大劳动群众进行长期的、艰巨的斗争。

我国在毛主席的革命路线指引下,二十多年来,社会主义革命已经取得了伟大的胜利。但是,至今还仍然存在着阶级和阶级斗争,存在着三大差别和资产阶级法权,存在着产生资本主义和资产阶级的土壤和条件。因而社会主义革命还远远没有结束,无产阶级同资产阶级的阶级斗争还将长期继续下去。无产阶级和革命人民要沿着社会主义道路继续前进,坚持不懈地革资产阶级的命,这必然会遇到剥削阶级顽强的反抗。在共产党内部,除了一些混进来的阶级异己分子以外,还有一些思想至今还停止在民主革命阶段的资产阶级民主派,批资产阶级法权他们有反感,害怕社会主义革命革到自己头上,触动他们所喜欢的资产阶级法权和一切旧的传统观念。他们竭力推行修正主义路线,反对社会主义新生事物,反对限制并维护和强化资产阶级法权,维护和扩大资产阶级赖以产生和存在的基础,实际上代表党内外的新旧资产阶级,成为党内走资本主义道路的当权派。社会总是要前进的,无产阶级革命事业决不能停滞不进。走资派妄图阻挡历史前进,搞复辟倒退,完全是痴心妄想。

我们必须“认真看书学习,弄通马克思主义”,提高阶级斗争和路线斗争觉悟,认清社会主义革命时期的主要危险是修正主义,最危险的是党内的走资派。积极扫除前进道路上的障碍,为夺取社会主义革命和社会主义建设事业的新胜利、为实现共产主义而斗争。

我们必须以阶级斗争为纲,坚持党在整个社会主义历史阶段的基本路线,坚持无产阶级对资产阶级的全面专政,批判修正主义,批判资产阶级,限制资产阶级法权,促进社会主义生产关系和上层建筑不断趋于完善。

我们必须坚持“鼓足干劲,力争上游,多快好省地建设社会主义”的总路线,坚持“抓革命、促生产、促工作、促战备”的伟大方针,发扬“独立自主、自力更生、艰苦奋斗、勤俭建国”的革命精神,促进生产力的不断发展,尽快地把我国建成为一个具有现代农业.现代工业、现代国防、现代科学技术的社会主义强国。

我们必须积极扶持具有共产主义因素的社会主义新生事物,同各种压抑、反对和破坏新生事物成长的行为作坚持不懈的斗争,使它们在与衰亡着的资本主义因素的斗争中不断地发展和壮大。“致们应当缜密地研究新的幼芽,极仔细地对待它们,尽力帮助它们成化,并‘照管’这些嫩弱的幼芽”,做社会主义新生事物的促进派。

我们必须坚持无产阶级政治挂帅,积极地宣传共产主义思想,提倡共产主义劳动态度,发扬共产主义风格。但是,又要认真地执行党在现阶段的各项政策。党的正确政策,反映了社会主义经济发展的客观规律和人民群众的根本利益,是党的正确路线的具体体现。在积极进行共产主义思想教育的同时,认真执行党的各项具体经济政策,就能更好地调动广大群众的社会主义积极性,社会主义革命和社会主义建设事业就能得到更快的发展。

我们必须坚持无产阶级国际主义,“坚决、彻底、干净、全部地消灭大国主义”,同全世界无产阶级、被压迫人民和被压迫民族团结在一起,为反对美苏两个超级大国的霸权主义和强权政治,为打倒帝国主义、现代修正主义和各国反动派,为在地球上消灭人剥削人的制度,使整个人类都得到解放而共同奋斗。

当前国际形势的特点是“天下大乱”,这个“乱”是当代世界各种基本矛盾日益激化的表现。它加剧了腐朽的反动势力的瓦解和没落,促进了新生的人民力量的觉醒和壮大,推动着世界各国人民的革命斗争蓬勃发展。不论帝、修、反玩弄多少阴谋诡计,都不能阻止国家要独立、民族要解放、人民要革命的历史潮流奔腾向前。

“全世界马克思列宁主义者团结起来,全世界革命人民团结起来,打倒帝国主义,打倒现代修正主义,打倒各国反动派。一个没有帝国主义、没有资本主义、没有剥削制度的新世界,一定要建立起来。”毛主席的这个伟大号召,概括了国际共产主义运动总路线的基本内容,反映了无产阶级和劳动人民的共同要求,提出了无产阶级世界革命的战斗任务,指明了人类社会发展的光辉前程,永远激励着我们为实现共产主义的伟大理想而斗争。

我们的事业是伟大的,我们的力量是强大的,我们的旗帜——马克思主义、列宁主义、毛泽东思想是战无不胜的,共产主义一定要在全世界赢得彻底胜利。“这是最后的斗争,团结起来到明天,英特纳雄耐尔就一定要实现。”



\end{document}